% ----------------------------------------------------------
\chapter{Conclusões}
% ----------------------------------------------------------

Este capítulo tem como objetivo expor o progresso relacionado às Questões de Pesquisa e Desenvolvimento, apresentando uma análise de como estas vêm sendo respondidas até o presente momento; o andamento do Objetivo Geral e dos Objetivos Específicos estabelecidos no Capítulo 1; as Atividades Executadas, que correspondem às ações previstas para a primeira fase do Trabalho de Conclusão de Curso (TCC), acompanhadas de seus respectivos status; bem como os Próximos Passos, que orientam o planejamento dos autores para as etapas subsequentes, abrangendo o desenvolvimento da segunda fase do TCC. Por fim, apresenta-se um resumo do capítulo.

Ademais, é importante citar que o conteúdo deste trabalho foi submetido a uma revisão assistida por meio da ferramenta ChatGPT, com o objetivo de aprimorar a clareza, a correção gramatical e a fluidez textual. Esse processo buscou garantir que a redação final apresentasse maior precisão linguística e coesão, sem comprometer a essência das ideias e a originalidade da produção acadêmica.

\section{Evolução das Questões de Pesquisa e Desenvolvimento}

Este trabalho apresenta, como questões de pesquisa e desenvolvimento, uma questão principal e questões secundárias, que orientam a investigação sobre as características necessárias em um aplicativo destinado a apoiar cuidadores de pessoas com Transtorno do Espectro Autista (TEA) no manejo da seletividade alimentar, por meio da sugestão de alternativas e substituições alimentares, sendo elas:

\begin{itemize}
    \item \textbf{Questão principal:} Quais são as características necessárias em um aplicativo móvel que possa apoiar cuidadores de pessoas com TEA no manejo da seletividade alimentar, por meio da sugestão de trocas alimentares viáveis e personalizadas?
    \item O que é seletividade alimentar no contexto de TEA?
    \item Quais as características de tratamentos usados neste contexto?
    \item Quais tecnologias envolvendo aplicativos de software têm sido usadas neste contexto?
\end{itemize}

\vspace{0.5em}

As questões secundárias foram extensivamente exploradas no Capítulo 2 deste trabalho, o Referencial Teórico. Nessa seção, buscou-se consolidar o entendimento sobre o Transtorno do Espectro Autista, aprofundar o conhecimento sobre a seletividade alimentar como um dos grandes desafios enfrentados por essa população e mapear as abordagens terapêuticas e multidisciplinares existentes.

A partir da investigação conduzida, especialmente no que tange aos tratamentos e às tecnologias atualmente empregadas, uma constatação fundamental emergiu: a carência de soluções tecnológicas acessíveis e específicas para o problema da seletividade alimentar. Embora a importância do manejo alimentar seja amplamente reconhecida na literatura, poucas ferramentas digitais se propõem a oferecer suporte prático e personalizado aos cuidadores no dia a dia.

É precisamente nesta lacuna que o presente trabalho se insere. A evidente falta de alternativas tecnológicas validou a necessidade de projetar e desenvolver uma solução inovadora. O desenvolvimento do aplicativo proposto surge, assim, como uma resposta direta à problemática identificada, com o objetivo de preencher essa lacuna e oferecer uma ferramenta de apoio funcional. Por meio da prototipação e do desenvolvimento do aplicativo, buscamos responder objetivamente à questão principal de pesquisa, materializando as características levantadas como essenciais em uma solução tecnológica viável e centrada no usuário.

\section{Andamento dos Objetivos}

O Objetivo Geral deste trabalho se baseia no desenvolvimento de um aplicativo móvel como ferramenta de apoio tecnológico voltada a cuidadores de pessoas com Transtorno do Espectro Autista (TEA) que apresentam seletividade alimentar, com o objetivo de promover o bem-estar por meio da sugestão personalizada de alternativas alimentares.

De forma semelhante às Questões de Pesquisa e Desenvolvimento, o objetivo geral apresenta, até o momento, um cumprimento parcial, visto que somente poderá ser considerado alcançado após a conclusão do desenvolvimento dos itens especificados no Backlog do Produto.

Para alcançar o propósito geral deste trabalho — o desenvolvimento de um aplicativo de apoio a cuidadores de pessoas com TEA no manejo da seletividade alimentar — foram traçados objetivos específicos que delinearam as etapas do projeto, desde a sua concepção teórica até a implementação prática. A consecução destes objetivos garantiu uma abordagem metódica e focada.

\begin{itemize}
    \item \textbf{Realizar uma pesquisa aprofundada sobre a seletividade alimentar} no contexto do Transtorno do Espectro Autista (TEA). Esta etapa, já concluída, foi crucial para estabelecer o estado da arte e compreender as nuances do problema que a solução proposta busca endereçar.

    \item \textbf{Estruturar a fundamentação teórica e o planejamento do produto de software.} Também concluída, esta fase permitiu consolidar o conhecimento adquirido na pesquisa e traduzi-lo em um escopo e uma arquitetura bem definidos para o aplicativo, detalhando seus requisitos funcionais e não funcionais.

    \item \textbf{Desenvolver a regra de negócio que serve como núcleo (*core*) da aplicação.} A conclusão desta etapa, fundamentada em todo o estudo realizado, resultou na lógica central da ferramenta, capaz de gerar sugestões personalizadas de trocas alimentares e constituindo seu principal diferencial.

    \item \textbf{Iniciar a implementação do aplicativo.} Esta é a etapa atual do projeto, na qual o planejamento e a lógica de negócio são efetivamente transformados em um produto tangível. O trabalho agora se concentra em traduzir os requisitos em código e construir as primeiras versões da ferramenta, que serão, em um momento posterior, submetidas à validação junto ao Product Owner (PO).
\end{itemize}

\begin{table}[h!]
\centering
\label{tab:metas_tcc1}
\caption{Resumo das Metas e Status de Execução do TCC 1.}
    \caption*{Fonte: Autor, 2025.}
\begin{tabular}{l|l}
\hline
\textbf{Atividade} & \textbf{Status} \\
\hline
Definir Tema & Concluída \\
Formular Proposta & Concluída \\
Contextualizar Problema & Concluída \\
Levantamento Bibliográfico & Concluída \\
Definir Metodologia & Concluída \\
Definir Backlog & Concluída \\
Desenvolver Protótipo da Aplicação & Em andamento (Versão MVP) \\
Revisar TCC1 & Concluída \\
Apresentar TCC1 & A Em Andamento \\
\hline
\end{tabular}
\end{table}

\vspace{1em}

\newpage

\section{Próximas Metas}

A conclusão desta etapa do trabalho representa um marco fundamental. Ela estabelece as bases teóricas e o planejamento necessários para avançar para a fase de implementação do projeto. O foco, a partir de agora, se volta para as atividades práticas do Trabalho de Conclusão de Curso 2, com o objetivo de transformar o conceito em um produto de software funcional e validado.

O ciclo de desenvolvimento do aplicativo será a meta central. As funcionalidades, já detalhadas em nosso backlog, serão construídas de forma incremental. Após a implementação de cada conjunto de funcionalidades, uma fase rigorosa de testes será conduzida. Estes testes visam assegurar a estabilidade do código, a performance da aplicação e a ausência de erros, garantindo um alto padrão de qualidade técnica.

Paralelamente, um processo de validação contínua será executado. O aplicativo será apresentado à nossa Product Owner em intervalos planejados. O feedback coletado em cada uma dessas sessões será crucial para o refinamento do produto. As sugestões e críticas serão analisadas pela equipe, que planejará e implementará as melhorias necessárias. Este ciclo de desenvolvimento, teste e validação se repetirá, garantindo que o produto evolua de maneira alinhada às expectativas e necessidades do usuário final.

A documentação será uma atividade transversal a todo o processo. O foco será registrar o percurso do desenvolvimento. Para isso, adotaremos os princípios da metodologia Scrum como guia. As atividades de cada sprint, as decisões tomadas, os desafios encontrados e os resultados obtidos serão cuidadosamente documentados. Isso criará um registro detalhado e transparente da evolução do projeto, que será a base para a redação do Trabalho de Conclusão de Curso 2.

Para assegurar a viabilidade de todas essas metas, foi elaborado um cronograma de execução detalhado. Este cronograma organiza as tarefas do backlog em uma linha do tempo realista, distribuindo o esforço de desenvolvimento de forma equilibrada. O início das atividades de implementação está previsto para o segundo semestre letivo de 2025.

\begin{table}[h!]
\centering
\label{tab:metas_tcc2}
\caption{Planejamento de Próximas Etapas para o TCC 2.}
    \caption*{Fonte: Autor, 2025.}
\begin{tabular}{l|l}
\hline
\textbf{Próxima Etapa / Atividade} & \textbf{Status} \\
\hline
Realizar correções apontadas na avaliação & A Fazer \\
Desenvolvimento da aplicação & A Fazer \\
Validação com Product Owner & A Fazer \\
Revisar documentação do TCC 2 & A Fazer \\
Apresentação do TCC 2 & A Fazer \\
\hline
\end{tabular}
\end{table}

\vspace{1em}

\begin{table}[H]
\centering
\caption{Cronograma Previsto de Desenvolvimento e Entregas (TCC 2).}
\label{tab:cronograma_tcc2}
\resizebox{\textwidth}{!}{%
\begin{tabular}{|l|c|c|c|c|c|c|c|c|c|}
\hline
\textbf{} & \multicolumn{2}{c|}{\textbf{Agosto}} & \multicolumn{2}{c|}{\textbf{Setembro}} & \multicolumn{2}{c|}{\textbf{Outubro}} & \multicolumn{2}{c|}{\textbf{Novembro}} & \textbf{Dezembro} \\
\cline{2-10}
\textbf{Épico / Fase} & \textbf{1ª Q} & \textbf{2ª Q} & \textbf{1ª Q} & \textbf{2ª Q} & \textbf{1ª Q} & \textbf{2ª Q} & \textbf{1ª Q} & \textbf{2ª Q} & \textbf{1ª Quinzena} \\
\hline
\textbf{1. Gestão de Usuários} & \cellcolor{planned} & & & & & & & & \\
\hline
\textbf{2. Cadastro de Supervisionados} & & \cellcolor{planned} & & & & & & & \\
\hline
\textbf{3. Avaliação Alimentar e Sensorial} & & & \cellcolor{planned} & \cellcolor{planned} & & & & & \\
\hline
\textbf{4. Geração de Perfil e Relatórios} & & & & & \cellcolor{planned} & \cellcolor{planned} & & & \\
\hline
\textbf{5. Testes, Refinamento e Validação} & & & & & & & \cellcolor{planned} & \cellcolor{planned} & \\
\hline
\textbf{6. Finalização TCC 2 e Apresentação} & & & & & & & & & \cellcolor{planned} \\
\hline
\end{tabular}%
}
\end{table}

\vspace{0.5em}

\section{Resumo do Capítulo}

Ao final desta primeira etapa do Trabalho de Conclusão de Curso, pode-se afirmar que os objetivos específicos iniciais foram satisfatoriamente alcançados. Foi realizado um robusto levantamento bibliográfico que fundamentou a contextualização do problema e permitiu responder às questões de pesquisa teóricas sobre o Transtorno do Espectro Autista, a seletividade alimentar e as abordagens terapêuticas existentes. Com base nesse embasamento, foram definidas a metodologia de desenvolvimento, a arquitetura tecnológica do sistema e o backlog detalhado do produto, além de um protótipo de alta fidelidade do projeto.

A próxima etapa deste trabalho, correspondente ao TCC 2, focará na concretização e validação da solução proposta. As atividades subsequentes incluem a implementação das correções apontadas, a continuidade do desenvolvimento da aplicação para além da versão MVP e a validação das funcionalidades com o Product Owner. O cumprimento dessas metas permitirá a análise dos resultados obtidos, a fim de alcançar o objetivo geral deste Trabalho de Conclusão de Curso: o desenvolvimento de uma aplicação móvel funcional e embasada, que sirva como ferramenta de apoio para cuidadores e pessoas com TEA no manejo personalizado da seletividade alimentar.