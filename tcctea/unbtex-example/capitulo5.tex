% ----------------------------------------------------------
\chapter{Conclusões}
% ----------------------------------------------------------

\section{Introdução}

Após o desenvolvimento, o aplicativo se propõe a ser uma solução para a limitação da diversidade alimentar em indivíduos com Transtorno do Espectro Autista e seletividade.

Neste capítulo, serão abordados o cumprimento dos objetivos específicos, as fragilidades mapeadas no projeto e as propostas sugeridas para a evolução do software.

\subsection{Objetivos Específicos}

Os objetivos específicos, que foram definidos previamente ao processo de desenvolvimento, foram cumpridos ao longo do processo de planejamento e de construção do software proposto. 

Identificamos os conceitos necessários ao desenvolviemnto do aplicativo de software através de reuniões com a PO, que nos explicou os aspectos nessesários para que pudéssemos implementar a lógica das trocas alimentares dentro do contexto estabelecido. 

Estabelecemos uma ferramenta de software para apoiar os cuidadores e os próprios indivíduos que possuem TEA e seletividade alimentar, através da criação de um aplicativo que atendesse aos requisitos que foram elicitados mediante entrevistas com a PO.

Por fim, o software será disponibilizado para que a PO possa estabelecer um estudo de caso para avaliar adequação do produto de software aos fins no qual ele se dedica.

\subsection{Processo de desenvolvimento}

O processo definido foi, em geral, cumprido, ainda que com ajustes pontuais que se mostraram necessários durante o desenvolvimento. Um exemplo foi a adequação das branches de gestão de configuração, visto que esta primeira versão foi restrita a dois desenvolvedores e não houve implantação (deploy) dos lançamentos.

O método GQM possibilitou uma avaliação objetiva do andamento do projeto. As métricas utilizadas foram efetivas para validar as decisões adotadas e assegurar o cumprimento dos objetivos. Os dados coletados no processo formaram uma base concreta para o aprimoramento contínuo e a otimização da ferramenta.

Os testes unitários do back-end indicaram um nível satisfatório de cobertura e confiabilidade do código. Sua execução foi fundamental para atestar o funcionamento esperado das funcionalidades críticas e para identificar falhas antecipadamente, contribuindo para a robustez do sistema. Desta forma, o desenvolvimento manteve um padrão de qualidade consistente e alinhado às práticas de engenharia de software.

Por fim, a aplicação dos testes de usabilidade com a PO confirmou que as necessidades dos usuários finais, levantadas no início do processo, foram atendidas. Além disso, os testes forneceram percepções valiosas sobre usabilidade e navegação, que indicam caminhos para futuras melhorias no projeto.

\section{Fragilidades e Propostas de Evolução}

Apesar do resultado satisfatório, foram mapeados aspectos que se enquadram como fragilidades. Estes devem ser explorados com o propósito de reconhecer as limitações da proposta implementada e mapear futuras soluções para tornar o sistema mais robusto.

As fragilidades identificadas foram classificadas em duas categorias, as internas e as externas. As internas dizem respeito a potenciais vulnerabilidades do código, tais como uma API não otimizada, limitações do banco de dados ou falhas de segurança. As externas abordam limitações fora do escopo do código, que, no caso deste TCC, estão associadas à pesquisa e à utilização do software.


\subsection{Fragilidades Internas}

\begin{enumerate}
    \item \textbf{Porcentagem da cobertura de testes:} Apesar de uma boa cobertura de testes, o sistema não possui 100\% de cobertura, o que significa que algum erro ainda pode surgir durante o uso do app.

   \item \textbf{Recuperação de Credenciais Simplificada:} O fluxo de recuperação de conta (para "esqueci minha senha") foi implementado de forma simplificada. Ao invés de um sistema de redefinição baseado em token enviado por e-mail, o projeto utilizará um mecanismo de "palavra de segurança" definida pelo usuário no cadastro. Esta implementação é temporária. A abordagem foi adotada por não haver, no momento, um serviço de envio de e-mails configurado e uma infraestrutura de hospedagem definitiva. Embora funcional, este método é considerado menos robusto que os padrões de mercado e representa uma fragilidade na segurança da autenticação.

    \item \textbf{Hospedagem do software:} Não há nenhum serviço de hospedagem sendo utilizado para o software. Além disso, não há um instalador do aplicativo e isso limita o uso do software pois ele só pode ser utilizado localmente.

    \item \textbf{Gerenciamento de Estado no Aplicativo - Fontend:} O aplicativo pode apresentar fragilidades na gestão de estado, especialmente em cenários de uso offline. Se o usuário registrar uma refeição ou aceitação sem conexão, o sistema pode não ter mecanismos robustos para sincronizar esses dados de forma segura quando a conexão for reestabelecida, gerando potencial perda de dados ou inconsistências.
\end{enumerate}


\subsection{Fragilidades Externas}

\begin{enumerate}
    \item \textbf{Limitação da lista de alimentos:} O software opera com uma lista pré-definida de alimentos e seus respectivos aspectos sensoriais. No entanto, essa lista é finita. Tal característica representa uma limitação, pois um usuário que recuse exaustivamente as sugestões eventualmente esgotará as opções disponíveis. Nesse cenário, o aplicativo não será mais capaz de gerar novas propostas.
    
    \item \textbf{Viés de Registro do Usuário:} A eficácia do sistema depende inteiramente da precisão e da consistência com que o usuário insere os dados de aceitação alimentar. A subjetividade na avaliação do usuário, como a interpretação do que seria aceitar bem um alimento, ou o esquecimento de registrar refeições podem gerar um perfil sensorial impreciso para o indivíduo, ocasionando sugestões de troca inadequadas.
    
    \item \textbf{Fatores Comportamentais Não Mapeados:} O software foca primariamente nos aspectos sensoriais dos alimentos. Contudo, a recusa alimentar no TEA é multifatorial e pode envolver o contexto da refeição, fatores emocionais, ou a forma de apresentação do prato. O aplicativo, em sua versão atual, não captura essas variáveis, o que limita a eficácia de suas sugestões.
\end{enumerate}


\subsection{Proposta de Evolução}

Uma série de sugestões foi mapeada com o propósito de documentação para um futuro trabalho de evolução do aplicativo. A lista com as propostas para futuras implementações é apresentada a seguir.

\begin{enumerate}
    \item \textbf{Criação de um novo tipo de conta:} Atualmente, o software possui dois tipos de conta: cuidador e indivíduo com TEA e seletividade alimentar. No entanto, identificou-se a oportunidade para a criação de um terceiro tipo, destinado ao profissional da área da saúde. Este perfil teria acesso aos dados dos indivíduos com TEA, permitindo a realização de acompanhamentos recorrentes.

    \item \textbf{Utilização de IA:} O uso de Inteligência Artificial pode aprimorar as sugestões de troca alimentar. A tecnologia permitiria uma análise mais refinada dos padrões sensoriais dos alimentos já consumidos por indivíduos com TEA, com o objetivo de gerar propostas mais assertivas.

    \item \textbf{Módulo de Relatórios de Progresso:} Para materializar os avanços e facilitar a comunicação com profissionais, sugere-se a implementação de um módulo de relatórios. O sistema poderia gerar gráficos simples sobre a evolução da aceitação alimentar do indivíduo, como o número de novos alimentos aceitos por período ou a frequência de consumo.

    \item \textbf{Orientações de Preparo e Receitas:} A aceitação de um novo alimento muitas vezes depende da forma como ele é preparado. O aplicativo poderia ser expandido para incluir um módulo de receitas e orientações de preparo focadas em aspectos sensoriais.

    \item \textbf{Hospedagem e Publicação nas Lojas:} A aplicação encontra-se em ambiente de desenvolvimento local. Para que a solução cumpra seu papel social e alcance efetivamente o público-alvo, é fundamental realizar a publicação do aplicativo nas lojas oficiais (Google Play Store e Apple App Store). Além disso, propõe-se a migração da infraestrutura do sistema para um serviço de hospedagem em nuvem.

    \item \textbf{Registro do Software:} Visando a segurança jurídica e a proteção da propriedade intelectual do trabalho desenvolvido, propõe-se a realização do registro do programa de computador junto ao Instituto Nacional da Propriedade Industrial (INPI). Essa etapa é essencial para assegurar os direitos autorais sobre o código-fonte e oficializar a autoria da solução tecnológica antes de sua ampla distribuição.
\end{enumerate}

\subsection{Conclusões finais}

O presente trabalho alcançou seu objetivo principal ao desenvolver e validar uma ferramenta digital voltada para o auxílio de cuidadores e pessoas com TEA no manejo da seletividade alimentar no contexto do Transtorno do Espectro Autista. A solução proposta alcançou os objetivos ao oferecer um meio gratuito e sistematizado para o registro e a sugestão de alimentos com base em perfis sensoriais. Os testes e validações realizados demonstraram que a aplicação atende aos requisitos funcionais estabelecidos, oferecendo uma interface coerente com as necessidades do público-alvo e facilitando a tomada de decisão no ambiente domiciliar.

Além da implementação técnica, este estudo permitiu a identificação de desafios e oportunidades no uso de tecnologia assistiva para o contexto alimentar. As limitações mapeadas durante o processo, como a necessidade de expansão da base de dados e a dependência de registros manuais, servem agora como um roteiro claro para a continuidade da pesquisa. A arquitetura desenvolvida fornece a fundação necessária para as futuras evoluções sugeridas, como a integração com profissionais de saúde e o uso de inteligência artificial, consolidando o potencial do software como um recurso de apoio efetivo na rotina das famílias.

O processo de desenvolvimento do aplicativo proporcionou aprendizados importantes, tanto no aspecto técnico quanto no metodológico. A aplicação prática de conceitos de engenharia de software, aliada à gestão ágil e ao uso de métricas (GQM), mostrou-se fundamental para contornar as limitações de recursos e manter o foco na entrega de valor. Além disso, a interação com os desafios reais da implementação, desde a estruturação da arquitetura até a validação da interface, reforçou a importância de uma abordagem centrada no usuário. Essa experiência consolidou o entendimento de que a eficácia de uma solução tecnológica para a saúde depende não apenas da robustez do código, mas de sua capacidade de se integrar à rotina de quem a utiliza.

Além disso, a elaboração deste trabalho acadêmico fomentou uma compreensão sobre a metodologia de pesquisa científica. A estruturação rigorosa do estudo, desde a definição do problema até a análise dos resultados, proporcionou uma visão crítica indispensável para a formação profissional. Simultaneamente, o aprofundamento nos fundamentos do Transtorno do Espectro Autista foi essencial. O estudo detalhado sobre as nuances da seletividade alimentar e o impacto sensorial na rotina dos indivíduos permitiu que a solução tecnológica fosse desenhada com a necessária empatia e embasamento teórico, garantindo que o software atendesse às especificidades do seu público-alvo.