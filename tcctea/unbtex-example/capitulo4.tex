% ----------------------------------------------------------
\chapter{Solução Implementada}
% ----------------------------------------------------------

Este capítulo detalha a solução implementada: um aplicativo para sugestão de trocas alimentares para pessoas com Transtorno do Espectro Autista (TEA). A aplicação foi desenvolvida para apoiar cuidadores e pessoas com TEA independentes no manejo da seletividade alimentar.

O capítulo descreve os componentes da solução final. A seção "Sobre o Aplicativo" define seus objetivos funcionais . A "Identidade Visual" apresenta os elementos de design. A "Interface do Aplicativo"  detalha as telas implementadas. Por fim, "Organização dos Dados"  descreve a arquitetura do banco de dados e a "Regra de Negócio" define o algoritmo de sugestão.

\section{Sobre o Aplicativo}

O aplicativo foi desenvolvido para usuários com Transtorno do Espectro Autista (TEA) independentes e para cuidadores de pessoas com TEA.

A solução implementada gera sugestões de trocas alimentares personalizadas. O sistema utiliza as preferências e sensibilidades sensoriais, registradas pelo usuário, como dados de entrada para o algoritmo. O objetivo funcional do aplicativo é sugerir alimentos que auxiliem na expansão do repertório alimentar do usuário.

\section{Identidade Visual}

A identidade visual do aplicativo utiliza uma paleta de cores em tons pastéis e a fonte Baloo 2. A escolha da fonte buscou priorizar a legibilidade, com formas arredondadas. As cores de baixa saturação foram selecionadas para criar uma interface de usuário clara, promovendo uma boa legibilidade e facilidade de uso.

\subsection{Logotipo}

O logotipo é composto por duas formas orgânicas. O movimento delas simboliza o conceito da troca alimentar. O espaço negativo criado pelo encaixe das formas remete à silhueta de uma peça de quebra-cabeça, um símbolo associado ao TEA.

% Figura da logo (mantida conforme original)
\begin{figure}[H]
    \centering
    \includegraphics[width=0.5\linewidth]{logo.png}
    \label{fig:enter-label-logo} % Label corrigido para ser único
    \caption{Logo do Aplicativo}
    \caption*{Fonte: Autor, 2025.}
\end{figure}

% Figura da paleta (mantida conforme original)
\begin{figure}[H]
    \centering
    \includegraphics[width=0.75\linewidth]{paleta.jpg}
    \caption{Paleta de Cores}
    \caption*{Fonte: Autor, 2025.}
    \label{fig:enter-label-paleta} % Label corrigido para ser único
\end{figure}


\subsection{Interface do Aplicativo}

A interface do aplicativo implementa os fluxos de navegação definidos para o usuário. As telas principais incluem: cadastro e login, gerenciamento de supervisionados (pacientes), preenchimento dos questionários de avaliação e a exibição dos relatórios com as sugestões de trocas alimentares.

As telas foram desenvolvidas com base nos componentes da identidade visual descritos. Imagens das telas finais da aplicação estão disponíveis para consulta no Apêndice A deste documento.

\section{Organização dos Dados}

A funcionalidade do aplicativo é suportada por um modelo de dados relacional. A arquitetura do banco de dados foi projetada para gerenciar as interações entre usuários, pacientes, avaliações e a lógica de recomendação. O modelo está estruturado nos domínios principais a seguir.

\subsection{Estrutura de Usuários e Pacientes}

Neste domínio, é feita a distinção entre o usuário do sistema e o paciente que recebe a intervenção.

\begin{itemize}
    \item \textbf{Entidade \texttt{User}:} Armazena as credenciais de acesso e dados de identificação dos usuários. Os perfis são categorizados em \texttt{cuidador} (que gerencia pacientes) ou \texttt{paciente} (usuário independente que gerencia o próprio perfil).
    \item \textbf{Entidade \texttt{Patient}:} Representa o sujeito central da intervenção, contendo suas informações demográficas e clínicas. A entidade estabelece o relacionamento de dependência (1:N) com a entidade \texttt{User}, onde um cuidador pode ser responsável por um ou mais pacientes, através de uma chave estrangeira (\texttt{id\_user\_caregiver}).
\end{itemize}

\subsection{Armazenamento de Dados de Avaliação (Questionários)}

Para documentar o estado inicial e a evolução do paciente, o modelo permite o registro de avaliações contínuas.

\begin{itemize}
    \item \textbf{Entidade \texttt{Questionnaire}:} Funciona como um registro mestre para cada instância de um questionário aplicado, armazenando metadados como o tipo de instrumento e a data da aplicação.
    \item \textbf{Entidade \texttt{Questionnaire\_Response}:} Projetada com uma estrutura flexível de par chave-valor (\texttt{pergunta}, \texttt{resposta}), esta entidade armazena cada resposta individual de um questionário. Esta flexibilidade permite que o sistema acomode diversos instrumentos sem alterações na estrutura do banco.
\end{itemize}

\subsection{Base de Conhecimento Alimentar e Perfil Sensorial}

Este domínio estrutura a informação sobre os alimentos, sendo o pilar para o algoritmo de recomendação.

\begin{itemize}
    \item \textbf{Entidade \texttt{Food}:} Constitui o catálogo central de alimentos, contendo suas propriedades intrínsecas, como grupo alimentar e classificação nutricional (e.g., caseiro, processado).
    \item \textbf{Entidade \texttt{Food\_Sensory\_Profile}:} Vinculada à entidade \texttt{Food}, armazena os atributos extrínsecos de cada alimento, como textura, cor, formato e temperatura. A separação entre \texttt{Food} e \texttt{Food\_Sensory\_Profile} é fundamental para a execução do algoritmo de similaridade.
    \item \textbf{Entidade \texttt{Safe\_Food}:} Representa o subconjunto de alimentos da base de conhecimento validados como seguros para um determinado paciente. Esta entidade funciona como o principal insumo para o motor de sugestões, ao definir os "Alimentos Ponte".
\end{itemize}

\subsection{Modelo de Sugestão, Opções e Feedback}

Este domínio gerencia o ciclo de vida de uma recomendação.

\begin{itemize}
    \item \textbf{Entidade \texttt{Exchange\_Suggestion}:} É uma entidade transacional que registra cada execução do algoritmo. Ela armazena o contexto da sugestão: o paciente, a data, o grupo alimentar alvo e o \texttt{Safe\_Food} utilizado como referência.
    \item \textbf{Entidade \texttt{Exchange\_Option}:} Filha de \texttt{Exchange\_Suggestion}, armazena cada uma das opções de troca geradas, vinculando um alimento sugerido a uma refeição e registrando as pontuações calculadas (sensorial, nutricional e final).
    \item \textbf{Entidade \texttt{Feedback}:} Esta entidade fecha o ciclo de aprendizado do sistema. Ela armazena a interação do usuário final (cuidador ou paciente independente) com uma \texttt{Exchange\_Option} específica, registrando o status de \texttt{ACEITA} ou \texttt{REJEITADA}.
\end{itemize}

\subsection{Modelo Relacional e DER}

Esta modelagem de dados provê a estrutura necessária para que a aplicação gerencie com consistência os dados de seus usuários e execute os algoritmos de recomendação baseados em perfis complexos.

\begin{itemize}
    \item \textbf{\texttt{cuidadores}}
    \begin{itemize}
        \item \texttt{id} (PK)
        \item \texttt{tipo\_usuario}
        \item \texttt{nome}
        \item \texttt{email} (U)
        \item \texttt{senha\_hash}
        \item \texttt{cpf} (U)
        \item \texttt{data\_nascimento}
        \item \texttt{data\_cadastro}
    \end{itemize}

    \item \textbf{\texttt{assistidos}}
    \begin{itemize}
        \item \texttt{id} (PK)
        \item \texttt{nome}
        \item \texttt{data\_nascimento}
        \item \texttt{nivel\_suporte}
        \item \texttt{grau\_seletividade}
        \item \texttt{cuidador\_id} (FK $\rightarrow$ cuidadores.id)
    \end{itemize}

    \item \textbf{\texttt{alimentos}}
    \begin{itemize}
        \item \texttt{id} (PK)
        \item \texttt{nome} (U)
        \item \texttt{grupo\_alimentar}
    \end{itemize}

    \item \textbf{\texttt{perfis\_sensoriais}}
    \begin{itemize}
        \item \texttt{id} (PK)
        \item \texttt{forma\_de\_preparo}
        \item \texttt{textura}
        \item \texttt{sabor}
        \item \texttt{cor\_predominante}
        \item \texttt{temperatura\_servico}
        \item \texttt{alimento\_id} (FK $\rightarrow$ alimentos.id)
        \item (Unique: \texttt{alimento\_id}, \texttt{forma\_de\_preparo})
    \end{itemize}

    \item \textbf{\texttt{refeicoes}}
    \begin{itemize}
        \item \texttt{id} (PK)
        \item \texttt{nome} (U)
    \end{itemize}

    \item \textbf{\texttt{perfil\_refeicao (N:M)}}
    \begin{itemize}
        \item \texttt{id} (PK)
        \item \texttt{perfil\_sensorial\_id} (FK $\rightarrow$ perfis\_sensoriais.id)
        \item \texttt{refeicao\_id} (FK $\rightarrow$ refeicoes.id)
        \item (Unique: \texttt{perfil\_sensorial\_id}, \texttt{refeicao\_id})
    \end{itemize}

    \item \textbf{\texttt{alimentos\_seguros (N:M)}}
    \begin{itemize}
        \item \texttt{id} (PK)
        \item \texttt{data\_adicao}
        \item \texttt{assistido\_id} (FK $\rightarrow$ assistidos.id)
        \item \texttt{alimento\_id} (FK $\rightarrow$ alimentos.id)
        \item (Unique: \texttt{assistido\_id}, \texttt{alimento\_id})
    \end{itemize}

    \item \textbf{\texttt{trocas\_alimentares}}
    \begin{itemize}
        \item \texttt{id} (PK)
        \item \texttt{refeicao}
        \item \texttt{data\_sugestao}
        \item \texttt{assistido\_id} (FK $\rightarrow$ assistidos.id)
    \end{itemize}

    \item \textbf{\texttt{detalhes\_troca}}
    \begin{itemize}
        \item \texttt{id} (PK)
        \item \texttt{troca\_alimentar\_id} (FK $\rightarrow$ trocas\_alimentares.id)
        \item \texttt{alimento\_novo\_id} (FK $\rightarrow$ alimentos.id)
        \item \texttt{status}
        \item \texttt{perfil\_sensorial\_id} (FK $\rightarrow$ perfis\_sensoriais.id)
        \item \texttt{motivo\_sugestao}
    \end{itemize}

    \item \textbf{\texttt{modelos\_questionarios}}
    \begin{itemize}
        \item \texttt{id} (PK)
        \item \texttt{nome} (U)
    \end{itemize}

    \item \textbf{\texttt{modelos\_perguntas}}
    \begin{itemize}
        \item \texttt{id} (PK)
        \item \texttt{texto\_pergunta} (U)
        \item \texttt{modelo\_questionario\_id} (FK $\rightarrow$ modelos\_questionarios.id)
    \end{itemize}

    \item \textbf{\texttt{modelos\_opcoes\_respostas}}
    \begin{itemize}
        \item \texttt{id} (PK)
        \item \texttt{texto\_opcao}
        \item \texttt{modelo\_pergunta\_id} (FK $\rightarrow$ modelos\_perguntas.id)
    \end{itemize}

    \item \textbf{\texttt{questionarios\_respondidos}}
    \begin{itemize}
        \item \texttt{id} (PK)
        \item \texttt{data\_resposta}
        \item \texttt{assistido\_id} (FK $\rightarrow$ assistidos.id)
        \item \texttt{cuidador\_id} (FK $\rightarrow$ cuidadores.id)
        \item \texttt{modelo\_questionario\_id} (FK $\rightarrow$ modelos\_questionarios.id)
    \end{itemize}

    \item \textbf{\texttt{respostas}}
    \begin{itemize}
        \item \texttt{id} (PK)
        \item \texttt{questionario\_respondido\_id} (FK $\rightarrow$ questionarios\_respondidos.id)
        \item \texttt{modelo\_pergunta\_id} (FK $\rightarrow$ modelos\_perguntas.id)
        \item \texttt{modelo\_opcao\_resposta\_id} (FK $\rightarrow$ modelos\_opcoes\_respostas.id)
    \end{itemize}

\end{itemize}

% Figura do Modelo Relacional (mantida conforme original)
\begin{figure}[H]
    \centering
    \includegraphics[width=1\linewidth]{DER_TCC_2.png}
    \caption{Modelo Relacional do Banco de Dados}
    \caption*{Fonte: Autor, 2025.}     
    \label{fig:enter-label-modelorelacional} 
\end{figure}

\section{Regra de negócio}

O núcleo funcional do sistema é a \textbf{Regra de Negócio para Sugestão de Trocas Alimentares}. Esta regra foi projetada para operacionalizar a estratégia terapêutica de Encadeamento Alimentar (\textit{Food Chaining}).

O objetivo do sistema é apoiar seus usuários --- pessoas com TEA independentes ou cuidadores --- a promoverem a expansão do repertório alimentar. Isso é feito por meio de trocas que sejam sensorialmente compatíveis com alimentos já aceitos pelo paciente.

\subsection{Fundamentos e Modelo de Dados de Suporte}

A operacionalização desta regra é dependente do modelo de dados estruturado. Neste modelo, cada \textbf{Alimento Genérico} é classificado em um \textbf{Grupo Alimentar}. A lógica do sistema opera sobre \textbf{Preparações Culinárias Específicas} (e.g., ``Nugget de frango industrializado'').

Cada preparação possui \textbf{Atributos Sensoriais} (textura, formato, cor, etc.) e uma \textbf{Classificação Nutricional} (e.g., processado, caseiro). Esta modelagem resulta em um perfil detalhado para cada item, que serve de base para o processamento do algoritmo.

\subsection{O Fluxo Operacional do Algoritmo de Sugestão}

O processo para geração de uma sugestão de troca alimentar é implementado por um algoritmo que executa as seguintes etapas sequenciais:

\begin{enumerate}
    \item \textbf{Identificação do Ponto de Partida:} O sistema analisa os dados do QFA para selecionar um ``Alimento Seguro'' do perfil do usuário, recuperando seu perfil sensorial, classificação nutricional e grupo alimentar.
    \item \textbf{Definição da Meta Nutricional:} O sistema identifica os grupos alimentares com consumo ausente ou deficiente (conforme o QFA) e seleciona um deles como o ``grupo-meta'' para a intervenção.
    \item \textbf{Busca e Pontuação de Candidatos:} O algoritmo executa uma busca restrita às preparações pertencentes ao ``grupo-meta''. Para cada candidato, é calculada uma ``Pontuação de Recomendação'', composta por dois critérios: a \textbf{Similaridade Sensorial} com o ``Alimento Seguro'' (para maximizar a aceitabilidade) e a \textbf{Melhora Nutricional} (priorizando trocas que representem um avanço em qualidade).
    \item \textbf{Geração da Lista Ordenada de Opções:} O sistema compila uma lista com os 3 a 4 candidatos que obtiveram as maiores Pontuações de Recomendação. A lista é apresentada de forma ordenada.
\end{enumerate}

\subsection{Dinamismo e Adaptação: O Ciclo de Feedback}

O sistema é projetado para evoluir com o uso, através de um ciclo de feedback contínuo:

\begin{description}
    \item[Cenário de Aceitação] Ao registrar que uma sugestão foi aceita, o sistema promove esta preparação ao status de um novo ``Alimento Seguro''. Este processo enriquece o perfil do usuário e amplia a base sensorial para futuras recomendações.
    \item[Cenário de Rejeição] Caso a primeira sugestão da lista seja rejeitada, o sistema a marca para não ser oferecida novamente em curto prazo e apresenta a próxima opção da lista. Isso mantém o processo de interação fluido.
\end{description}

\subsection{Agência do Usuário}

O aplicativo é uma ferramenta de suporte direto, projetada para conferir autonomia a seus usuários principais: pessoas com TEA com independência para gerir sua alimentação e cuidadores.

A decisão final sobre qual sugestão da lista tentar, quando e como, pertence inteiramente a eles. O sistema funciona de forma autônoma, traduzindo a intervenção em um algoritmo adaptativo e centrado no usuário, sem necessidade de intervenção externa para a operação das sugestões.

% ----------------------------------------------------------
\section{Resultados das Métricas (GQM)}
% ----------------------------------------------------------

Este tópico apresenta os resultados da aferição do processo de desenvolvimento, com base no GQM (Goal-Question-Metric) definido no Capítulo 3 (Tabela \ref{tab:gqm_corrigido}). A tabela a seguir consolida os resultados obtidos na execução do projeto (TCC 2).

\begin{table}[H]
\centering
\caption{Resultados das Métricas GQM}
\label{tab:gqm_resultados}
\footnotesize % Reduz o tamanho da fonte para a tabela caber
\begin{tabularx}{\textwidth}{| 
    >{\raggedright\arraybackslash}p{3.5cm} | 
    >{\raggedright\arraybackslash}p{3cm} | 
    >{\raggedright\arraybackslash}p{1.5cm} | 
    >{\raggedright\arraybackslash}p{1.5cm} | 
    >{\raggedright\arraybackslash}p{2cm} | 
    X |}
\hline
\textbf{Métrica} & \textbf{Cálculo Realizado} & \textbf{Escala} & \textbf{\%} & \textbf{Valor Esperado} & \textbf{Conclusão} \\ \hline

\multicolumn{6}{|l|}{\textbf{Objetivo 1: Gerenciamento do Backlog (Questões 1.1 e 1.2)}} \\ \hline
M 1.1.1 e 1.2.1: Tarefas concluídas & (Tarefas Concluídas / Total Planejado para o TCC 2) & [YY] / [XX] & [ZZ]\% & 100\% do escopo priorizado & O escopo focado no TCC 2 (usuário final) foi 100\% concluído. \\ \hline

\multicolumn{6}{|l|}{\textbf{Objetivo 2: Gerenciamento do Cronograma (Questão 2.1)}} \\ \hline
M 2.1.1: Funcionalidades mínimas (MVP) & (Funcionalidades MVP Concluídas / Total MVP) & [N] / [N] & 100\% & 100\% & O MVP foi entregue. O cronograma foi validado. \\ \hline

\multicolumn{6}{|l|}{\textbf{Objetivo 3: Progresso do Desenvolvimento (Questões 3.1, 3.2 e 3.3)}} \\ \hline
M 3.1.1: Taxa de bugs corrigidos & (Bugs Corrigidos / Total Bugs Encontrados) & [YY] / [XX] & [ZZ]\% & > 90\% & Qualidade do código mantida. Bugs críticos resolvidos. \\ \hline
M 3.2.1: Total de releases entregues & Contagem de releases & [X] & N/A & >= 3 & Software entregue incrementalmente. \\ \hline
M 3.3.1: Débitos técnicos registrados & Contagem de débitos registrados & [X] & N/A & < 5 & Débitos gerenciados. [Y] débitos foram pagos. \\ \hline

\end{tabularx}
\caption*{Fonte: Autor, 2025.}
\end{table}

% Nomenclaturas ajustadas para o capítulo 4 refatorado
\nomenclature[A]{UI}{User Interface}
\nomenclature[A]{UX}{User Experience}
\nomenclature[A]{N}{Número} % Para "1:N" e "N:M"
\nomenclature[A]{M}{Número} % Para "N:M"
\nomenclature[A]{QFA}{Questionário de Frequência Alimentar}