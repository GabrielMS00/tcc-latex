% ----------------------------------------------------------
\chapter{Solução Implementada}
% ----------------------------------------------------------

Este capítulo detalha a solução implementada: um aplicativo para sugestão de trocas alimentares para pessoas com Transtorno do Espectro Autista (TEA), desenvolvido para apoiar cuidadores e pessoas com TEA independentes no manejo da seletividade alimentar.

\section{Sobre o Aplicativo}

O aplicativo foi desenvolvido para usuários com Transtorno do Espectro Autista (TEA) independentes e para cuidadores de pessoas com TEA.

A solução implementada gera sugestões de trocas alimentares personalizadas, utilizando as preferências e sensibilidades sensoriais registradas pelo usuário como dados de entrada para o algoritmo. O objetivo funcional do aplicativo é sugerir alime

\section{\textit{Backlog} do produto}

Conforme introduzido no Capítulo 3, o Backlog do Produto é um artefato central em metodologias ágeis, funcionando como uma lista priorizada de todas as funcionalidades e requisitos do projeto. Para o desenvolvimento da solução proposta neste trabalho, foi criado um backlog específico que serviu como guia para a equipe de desenvolvimento.

Este backlog foi populado com itens na forma de Histórias de Usuário, detalhando as necessidades dos usuários finais. A priorização desses itens foi fundamental para definir o escopo da versão atual do projeto, garantindo que as funcionalidades de maior valor e impacto fossem desenvolvidas primeiro. Portanto, o sistema apresentado neste capítulo é o resultado tangível dos itens que foram selecionados, planejados e executados a partir do topo do nosso backlog.

As Histórias de Usuário foram utilizadas para descrever todas as funcionalidades do sistema sob a perspectiva de quem o utiliza. Adotando o formato padrão de mercado — "Como um [ator], eu quero [realizar uma ação] para que [possa obter um benefício]" —, conseguimos manter o foco na entrega de valor real. Cada história representava um requisito de negócio ou uma necessidade do usuário que o software deveria satisfazer. Em contrapartida, as Tarefas Técnicas foram criadas para registrar trabalhos essenciais que não entregam uma nova funcionalidade necessariamente visível ao usuário final, mas são cruciais para a saúde, qualidade e viabilidade do projeto.



\subsection*{Backlog do Aplicativo (Priorização MoSCoW)}

O backlog foi estruturado para atender aos dois perfis de usuário finais: o Cuidador (que gerencia assistidos) e o Usuário Independente (pessoa com TEA que gerencia a própria alimentação).

\begin{itemize}
    \item \textbf{US01 (Must Have, 3 pts)}: Como usuário (cuidador ou independente), quero criar uma conta para acessar as funcionalidades do aplicativo.
    
    \item \textbf{US02 (Must Have, 2 pts)}: Como usuário, quero realizar login para acessar meus dados de forma segura.

    \item \textbf{US03 (Must Have, 3 pts)}: Como cuidador, quero cadastrar múltiplos assistidos para gerenciar as preferências alimentares de cada um individualmente.

    \item \textbf{US04 (Must Have, 3 pts)}: Como usuário independente, quero configurar meu próprio perfil com dados pessoais e nível de suporte.

    \item \textbf{US05 (Must Have, 3 pts)}: Como usuário, quero registrar os "Alimentos Seguros" (preferências e aceitação atual) para alimentar a base de dados do algoritmo.

    \item \textbf{US06 (Must Have, 5 pts)}: Eu, como sistema, devo processar os "Alimentos Seguros" e identificar um grupo alimentar alvo para sugerir trocas.

    \item \textbf{US07 (Must Have, 5 pts)}: Eu, como sistema, devo gerar uma lista de sugestões de trocas alimentares baseada na similaridade sensorial com os alimentos já aceitos.

    \item \textbf{US08 (Must Have, 3 pts)}: Como usuário, quero visualizar a lista de sugestões de troca ordenadas pela compatibilidade sensorial.

    \item \textbf{US09 (Should Have, 2 pts)}: Como usuário, quero registrar o feedback (aceitou/recusou) sobre uma sugestão para que o sistema aprenda e melhore as próximas recomendações.

    \item \textbf{US10 (Must Have, 3 pts)}: Como usuário, quero visualizar um relatório com o histórico das trocas alimentares realizadas (aceitas e rejeitadas) para acompanhar a evolução da dieta.

    \item \textbf{US11 (Should Have, 2 pts)}: Como usuário, quero alterar minha senha de acesso para garantir a segurança da conta.
\end{itemize}

\subsection*{Épicos do Produto}

Os requisitos foram agrupados em épicos que representam os grandes módulos funcionais da solução implementada.

\begin{itemize}
    \item \textbf{Épico 1: Gestão de Identidade e Acesso}
    \begin{itemize}
        \item (Must Have) \textbf{US01}: Criação de conta (Cuidador e Independente).
        \item (Must Have) \textbf{US02}: Autenticação (Login).
        \item (Should Have) \textbf{US11}: Alteração de senha.
        \item (Must Have) \textbf{TK01}: Implementar autenticação JWT e controle de sessão.
    \end{itemize}

    \vspace{0.3cm}

    \item \textbf{Épico 2: Gerenciamento de Perfis e Assistidos}
    \begin{itemize}
        \item (Must Have) \textbf{US03}: Cadastro de múltiplos assistidos (perfil Cuidador).
        \item (Must Have) \textbf{US04}: Configuração de auto-perfil (perfil Independente).
        \item (Must Have) \textbf{TK02}: Modelagem do banco de dados para suportar relacionamento 1:N (Cuidador-Assistidos).
    \end{itemize}

    \vspace{0.3cm}

    \item \textbf{Épico 3: Mapeamento do Repertório Alimentar e Sensorial}
    \begin{itemize}
        \item (Must Have) \textbf{US05}: Registro de Alimentos Seguros e preferências sensoriais.
        \item (Must Have) \textbf{TK03}: Popular banco de dados com a base de alimentos e seus atributos sensoriais detalhados (textura, sabor, cor) para viabilizar o algoritmo.
    \end{itemize}

    \vspace{0.3cm}

    \item \textbf{Épico 4: Sistema de Recomendação e Análise de Evolução}
    \begin{itemize}
        \item (Must Have) \textbf{US06}: Identificação automática do grupo-meta para intervenção.
        \item (Must Have) \textbf{US07}: Algoritmo de recomendação por similaridade sensorial.
        \item (Must Have) \textbf{US08}: Visualização das sugestões de troca.
        \item (Must Have) \textbf{US09}: Registro de feedback (Aceitação/Recusa) e atualização automática dos Alimentos Seguros.
        \item (Should Have) \textbf{US10}: Relatório de histórico das trocas realizadas.
        \item (Must Have) \textbf{TK04}: Implementação da lógica de pontuação ponderada (Similaridade Sensorial vs. Melhora Nutricional) no serviço de sugestões.
    \end{itemize}
\end{itemize}


\section{Identidade Visual}

A identidade visual do aplicativo utiliza uma paleta de cores em tons pastéis e a fonte Baloo 2. A escolha da fonte buscou priorizar a legibilidade por meio de formas arredondadas, enquanto as cores de baixa saturação foram selecionadas para criar uma interface de usuário clara, promovendo assim a facilidade de uso.

\subsection{Logotipo}

O logotipo é composto por duas formas orgânicas, cujo movimento simboliza o conceito da troca alimentar. O espaço negativo criado pelo encaixe das formas remete à silhueta de uma peça de quebra-cabeça, um símbolo associado ao TEA.

% Figura da logo (mantida conforme original)
\begin{figure}[H]
   \centering
   \includegraphics[width=0.5\linewidth]{logo.png}
   \label{fig:enter-label-logo} % Label corrigido para ser único
   \caption{Logo do Aplicativo}
   \caption*{Fonte: Autor, 2025.}
\end{figure}

% Figura da paleta (mantida conforme original)
\begin{figure}[H]
   \centering
   \includegraphics[width=0.75\linewidth]{paleta.jpg}
   \caption{Paleta de Cores}
   \caption*{Fonte: Autor, 2025.}
   \label{fig:enter-label-paleta} % Label corrigido para ser único
\end{figure}


\subsection{Interface do Aplicativo}

A interface do aplicativo foi projetada para guiar o usuário de forma intuitiva através dos principais fluxos de uso: autenticação, gestão de assistidos e o processo de troca alimentar.

O primeiro ponto de contato é o fluxo de acesso. Caso o usuário ainda não possua conta, ele utiliza a tela de \textbf{Cadastro} (Figura \ref{fig:tela_cadastro}), onde fornece seus dados básicos. Para usuários recorrentes, o acesso é feito via tela de \textbf{Login} (Figura \ref{fig:tela_login}), garantindo a segurança das informações pessoais.

Após a autenticação, o usuário é direcionado para a \textbf{Home do Cuidador} (Figura \ref{fig:tela_home}), que atua como o painel central de navegação. Nesta área, é possível acessar as configurações de conta através da tela de \textbf{Perfil} (Figura \ref{fig:tela_perfil}) e, principalmente, gerenciar os indivíduos sob seus cuidados na tela de \textbf{Cadastro de Assistido} (Figura \ref{fig:tela_cadastro_assistido}). É nesta etapa que se definem os dados críticos como nível de suporte e grau de seletividade.

O núcleo funcional do sistema — a sugestão de trocas — inicia-se com a coleta de dados. O cuidador preenche o \textbf{Questionário Alimentar} (Figura \ref{fig:tela_questionario}), registrando o repertório atual do assistido. Com base nesses dados, o sistema processa as informações e apresenta as opções na tela de \textbf{Sugestão de Troca} (Figura \ref{fig:tela_sugestao}), exibindo alternativas sensorialmente compatíveis.

Para fechar o ciclo de aprendizado do algoritmo, o aplicativo solicita o retorno do usuário na tela de \textbf{Feedback} (Figura \ref{fig:tela_feedback}), onde é registrado se a sugestão foi aceita ou recusada, refinando as futuras recomendações apresentadas na listagem de \textbf{Refeições Disponíveis} (Figura \ref{fig:tela_refeicoes}).

As imagens detalhadas de todas as telas mencionadas encontram-se disponíveis para consulta no Apêndice \ref{apendice_telas}.

\subsection{Interface do Aplicativo}

A interface do aplicativo foi desenvolvida com base nos requisitos levantados e na identidade visual definida. Antes da implementação final, foi elaborado um protótipo de média fidelidade para a validação inicial dos fluxos de navegação, o qual pode ser visualizado no Apêndice \ref{apendice_media_fidelidade}.

Além disso, o protótipo interativo de média fidelidade, que serviu de base para o desenvolvimento do \textit{front-end}, está disponível para acesso público através do seguinte endereço: \url{https://www.figma.com/design/ZPEHj7ke48KWDV5uJfHKMw/Prototipo_App_TCC_TEA?node-id=0-1&t=bTELW0yh2FSIr6Hp-1}.

\section{Estrutura do Código da Aplicação}

A estrutura do código-fonte da aplicação reflete a arquitetura em 3 camadas (3-Tier). Este padrão de \textit{design} divide a aplicação em três unidades lógicas separadas para garantir a separação de responsabilidades: a Camada de Apresentação (a interface do usuário), a Camada de Negócio (o servidor com a lógica) e a Camada de Dados (o banco de dados).

O projeto foi organizado em um formato \textit{monorepo}, onde o código-fonte do cliente (front-end) e do servidor (back-end) coexistem no mesmo repositório, mas estão desacoplados em diretórios distintos (\texttt{/app-tea} e \texttt{/backend}) para implementar fisicamente essa separação.

O código-fonte completo da solução implementada está disponível no repositório do projeto e está hospedado na plataforma GitHub, no seguinte endereço: \url{https://github.com/GabrielMS00/Aplicativo_TCC_TEA}. 

\subsection{Camada de Apresentação (Front-end)}

O front-end está contido no diretório \texttt{/app-tea} e consiste em um aplicativo móvel desenvolvido em React Native com o framework Expo. Esta camada é a interface com a qual o usuário interage. Sua estrutura interna é organizada da seguinte forma:

\begin{itemize}
    \item \textbf{Roteamento (app/):} O sistema de navegação utiliza o \texttt{expo-router}, que adota uma abordagem de roteamento baseada em arquivos. Os diretórios dentro de \texttt{app/} definem as rotas da aplicação, como as rotas de autenticação \texttt{(auth)} e as rotas principais pós-login \texttt{(tabs)}.
    
    \item \textbf{Componentes (components/):} Contém os componentes de UI reutilizáveis, como botões, inputs e cards. A estilização desses componentes é feita com a biblioteca NativeWind.
    
    \item \textbf{Serviços de API (api/):} Abstrai a comunicação com o back-end. Um cliente \texttt{apiClient.ts} (baseado em Axios) é configurado, e serviços específicos (ex: \texttt{auth.ts}, \texttt{assistidos.ts}) exportam funções para consumir os \textit{endpoints}.
    
    \item \textbf{Gerenciamento de Estado (context/):} Utiliza a Context API do React para gerenciamento de estado global, como o contexto de autenticação (\texttt{AuthContext.tsx}), que provê os dados do usuário para a aplicação.
\end{itemize}

\subsection{Camada de Negócio e Dados (Back-end)}

O back-end está contido no diretório \texttt{/backend} e consiste na API RESTful desenvolvida em Node.js com o framework Express.js. Esta implementação agrupa tanto a Camada de Negócio quanto a Camada de Dados:

\begin{itemize}
    \item \textbf{Rotas (src/api/routes/):} Define os \textit{endpoints} da API. O arquivo \texttt{index.js} centraliza os diferentes arquivos de rota (ex: \texttt{sugestaoRoutes.js}, \texttt{authRoutes.js}), que mapeiam os verbos HTTP (GET, POST, etc.) para os \textit{controllers} correspondentes.
    
    \item \textbf{Controladores (src/api/controllers/):} Responsáveis por receber as requisições (Request) e formular as respostas (Response). Eles validam os dados de entrada e orquestram a execução da lógica de negócio, acionando os \textit{services}.
    
    \item \textbf{Serviços (src/services/) - Camada de Negócio:} Contêm a lógica de negócio principal do sistema. Por exemplo, \texttt{sugestaoService.js} implementa o algoritmo da Regra de Negócio (comparação sensorial, pontuação), enquanto \texttt{processamentoQuestionarioService.js} lida com o processamento das respostas dos questionários.
    
    \item \textbf{Acesso a Dados (src/api/models/ e config/db.js) - Camada de Dados:} A camada de dados é implementada por \textit{models} que executam consultas SQL diretas no banco de dados PostgreSQL, cuja conexão é gerenciada pelo arquivo \texttt{db.js}.
    
    \item \textbf{Schema do Banco (migrations/):} A estrutura (schema) do banco de dados é gerenciada de forma versionada através de arquivos de migração (ex: \texttt{...create-table-cuidadores.js}), utilizando a ferramenta \texttt{node-pg-migrate}.
\end{itemize}

\subsection{Ambiente e Orquestração}

Finalmente, a plataforma Docker é utilizada para padronizar e gerenciar o ambiente de execução. O \texttt{Dockerfile} define a imagem de contêiner para o back-end Node.js. O arquivo \texttt{docker-compose.yml}, na raiz do projeto, orquestra a inicialização conjunta da Camada de Negócio (serviço \texttt{api}) e da Camada de Dados (serviço \texttt{postgres}), garantindo um ambiente de desenvolvimento reprodutível e isolado.


\section{Organização dos Dados}

A funcionalidade do aplicativo é suportada por um modelo de dados relacional. Para isso, a arquitetura do banco de dados foi projetada para gerenciar as interações entre usuários, pacientes, avaliações e a lógica de recomendação. O modelo está estruturado nos domínios principais a seguir.

\subsection{Estrutura de Usuários e Pacientes}

Neste domínio, é feita a distinção entre o usuário do sistema e o paciente que recebe a intervenção.

\begin{itemize}
   \item \textbf{Entidade \texttt{User}:} Armazena as credenciais de acesso e dados de identificação dos usuários. Os perfis são categorizados em \texttt{cuidador} (que gerencia pacientes) ou \texttt{paciente} (usuário independente que gerencia o próprio perfil).
   \item \textbf{Entidade \texttt{Patient}:} Representa o sujeito central da intervenção, contendo suas informações demográficas e clínicas. A entidade estabelece o relacionamento de dependência (1:N) com a entidade \texttt{User}, onde um cuidador pode ser responsável por um ou mais pacientes, através de uma chave estrangeira (\texttt{id\_user\_caregiver}).
\end{itemize}

\subsection{Armazenamento de Dados de Avaliação (Questionários)}

Para documentar o estado inicial e a evolução do paciente, o modelo permite o registro de avaliações contínuas.

\begin{itemize}
   \item \textbf{Entidade \texttt{Questionnaire}:} Funciona como um registro mestre para cada instância de um questionário aplicado, armazenando metadados como o tipo de instrumento e a data da aplicação.
   \item \textbf{Entidade \texttt{Questionnaire\_Response}:} Projetada com uma estrutura flexível de par chave-valor (\texttt{pergunta}, \texttt{resposta}), esta entidade armazena cada resposta individual, permitindo que o sistema acomode diversos instrumentos sem alterações na estrutura do banco.
\end{itemize}

\subsection{Base de Conhecimento Alimentar e Perfil Sensorial}

Este domínio estrutura a informação sobre os alimentos, sendo o pilar para o algoritmo de recomendação.

\begin{itemize}
   \item \textbf{Entidade \texttt{Food}:} Constitui o catálogo central de alimentos, contendo suas propriedades intrínsecas, como grupo alimentar e classificação nutricional (e.g., caseiro, processado).
   \item \textbf{Entidade \texttt{Food\_Sensory\_Profile}:} Vinculada à entidade \texttt{Food}, armazena os atributos extrínsecos de cada alimento (textura, cor, etc.). A separação entre \texttt{Food} e \texttt{Food\_Sensory\_Profile} é fundamental para a execução do algoritmo de similaridade.
   \item \textbf{Entidade \texttt{Safe\_Food}:} Representa o subconjunto de alimentos validados como seguros para um paciente. Esta entidade funciona como o principal insumo para o motor de sugestões, ao definir os "Alimentos Ponte".
\end{itemize}

\subsection{Modelo de Sugestão, Opções e Feedback}

Este domínio gerencia o ciclo de vida de uma recomendação, desde sua geração até a resposta do usuário.

\begin{itemize}
   \item \textbf{Entidade \texttt{Exchange\_Suggestion}:} É uma entidade transacional que registra cada execução do algoritmo, armazenando o contexto da sugestão (paciente, data, grupo-meta e o \texttt{Safe\_Food} de referência).
   \item \textbf{Entidade \texttt{Exchange\_Option}:} Filha de \texttt{Exchange\_Suggestion}, armazena cada uma das opções de troca geradas, vinculando um alimento sugerido a uma refeição e registrando as pontuações calculadas.
   \item \textbf{Entidade \texttt{Feedback}:} Esta entidade fecha o ciclo de aprendizado do sistema, pois armazena a interação do usuário final (cuidador ou paciente) com uma \texttt{Exchange\_Option}, registrando o status de \texttt{ACEITA} ou \texttt{REJEITADA}.
\end{itemize}

\subsection{Modelo Relacional e DER}

Abaixo segue o modelo relacional implementado, que descreve textualmente a estrutura do banco de dados.

\begin{itemize}
   \item \textbf{\texttt{cuidadores}}
   \begin{itemize}
      \item \texttt{id} (PK)
      \item \texttt{tipo\_usuario}
      \item \texttt{nome}
      \item \texttt{email} (Unique)
      \item \texttt{senha\_hash}
      \item \texttt{cpf} (Unique)
      \item \texttt{data\_nascimento}
      \item \texttt{data\_cadastro}
      \item \texttt{palavra\_seguranca}
   \end{itemize}

   \item \textbf{\texttt{assistidos}}
   \begin{itemize}
      \item \texttt{id} (PK)
      \item \texttt{nome}
      \item \texttt{data\_nascimento}
      \item \texttt{nivel\_suporte}
      \item \texttt{grau\_seletividade}
      \item \texttt{cuidador\_id} (FK $\rightarrow$ cuidadores.id)
   \end{itemize}

   \item \textbf{\texttt{alimentos}}
   \begin{itemize}
      \item \texttt{id} (PK)
      \item \texttt{nome} (Unique)
      \item \texttt{grupo\_alimentar}
   \end{itemize}

   \item \textbf{\texttt{perfis\_sensoriais}}
   \begin{itemize}
      \item \texttt{id} (PK)
      \item \texttt{forma\_de\_preparo}
      \item \texttt{textura}
      \item \texttt{sabor}
      \item \texttt{cor\_predominante}
      \item \texttt{temperatura\_servico}
      \item \texttt{alimento\_id} (FK $\rightarrow$ alimentos.id)
      \item (Unique: \texttt{alimento\_id}, \texttt{forma\_de\_preparo})
   \end{itemize}

   \item \textbf{\texttt{refeicoes}}
   \begin{itemize}
      \item \texttt{id} (PK)
      \item \texttt{nome} (Unique)
   \end{itemize}

   \item \textbf{\texttt{perfil\_refeicao (N:M)}}
   \begin{itemize}
      \item \texttt{id} (PK)
      \item \texttt{perfil\_sensorial\_id} (FK $\rightarrow$ perfis\_sensoriais.id)
      \item \texttt{refeicao\_id} (FK $\rightarrow$ refeicoes.id)
      \item (Unique: \texttt{perfil\_sensorial\_id}, \texttt{refeicao\_id})
   \end{itemize}

   \item \textbf{\texttt{alimentos\_seguros (N:M)}}
   \begin{itemize}
      \item \texttt{id} (PK)
      \item \texttt{data\_adicao}
      \item \texttt{assistido\_id} (FK $\rightarrow$ assistidos.id)
      \item \texttt{alimento\_id} (FK $\rightarrow$ alimentos.id)
      \item (Unique: \texttt{assistido\_id}, \texttt{alimento\_id})
   \end{itemize}

   \item \textbf{\texttt{trocas\_alimentares}}
   \begin{itemize}
      \item \texttt{id} (PK)
      \item \texttt{refeicao}
      \item \texttt{data\_sugestao}
      \item \texttt{assistido\_id} (FK $\rightarrow$ assistidos.id)
   \end{itemize}

   \item \textbf{\texttt{detalhes\_troca}}
   \begin{itemize}
      \item \texttt{id} (PK)
      \item \texttt{troca\_alimentar\_id} (FK $\rightarrow$ trocas\_alimentares.id)
      \item \texttt{alimento\_novo\_id} (FK $\rightarrow$ alimentos.id)
      \item \texttt{status}
      \item \texttt{perfil\_sensorial\_id} (FK $\rightarrow$ perfis\_sensoriais.id)
      \item \texttt{motivo\_sugestao}
   \end{itemize}

   \item \textbf{\texttt{modelos\_questionarios}}
   \begin{itemize}
      \item \texttt{id} (PK)
      \item \texttt{nome} (Unique)
   \end{itemize}

   \item \textbf{\texttt{modelos\_perguntas}}
   \begin{itemize}
      \item \texttt{id} (PK)
      \item \texttt{texto\_pergunta} (Unique)
      \item \texttt{modelo\_questionario\_id} (FK $\rightarrow$ modelos\_questionarios.id)
   \end{itemize}

   \item \textbf{\texttt{modelos\_opcoes\_respostas}}
   \begin{itemize}
      \item \texttt{id} (PK)
      \item \texttt{texto\_opcao}
      \item \texttt{modelo\_pergunta\_id} (FK $\rightarrow$ modelos\_perguntas.id)
   \end{itemize}

   \item \textbf{\texttt{questionarios\_respondidos}}
   \begin{itemize}
      \item \texttt{id} (PK)
      \item \texttt{data\_resposta}
      \item \texttt{assistido\_id} (FK $\rightarrow$ assistidos.id)
      \item \texttt{cuidador\_id} (FK $\rightarrow$ cuidadores.id)
      \item \texttt{modelo\_questionario\_id} (FK $\rightarrow$ modelos\_questionarios.id)
   \end{itemize}

   \item \textbf{\texttt{respostas}}
   \begin{itemize}
      \item \texttt{id} (PK)
      \item \texttt{questionario\_respondido\_id} (FK $\rightarrow$ questionarios\_respondidos.id)
      \item \texttt{modelo\_pergunta\_id} (FK $\rightarrow$ modelos\_perguntas.id)
      \item \texttt{modelo\_opcao\_resposta\_id} (FK $\rightarrow$ modelos\_opcoes\_respostas.id)
   \end{itemize}

\end{itemize}

% Figura do Modelo Relacional (mantida conforme original)
% Figura em página paisagem (rotacionada 90 graus)
% Página em modo paisagem (Landscape)
% Página em modo paisagem
\begin{landscape}
    \centering % Centraliza verticalmente e horizontalmente na página
    \begin{figure} % Sem [H], deixe flutuar para ajustar na página
        \centering
        \includegraphics[width=0.9\linewidth]{DER_TCC_2_.png} 
        \caption{Diagrama Entidade-Relacionamento Banco de Dados}
        \caption*{Fonte: Autor, 2025.}
        \label{fig:enter-label-modelorelacional}
    \end{figure}
\end{landscape}

Esta modelagem de dados, representada textualmente pelo Modelo Relacional e visualmente pelo Diagrama Entidade-Relacionamento (Figura \ref{fig:enter-label-modelorelacional}), provê a estrutura necessária para que a aplicação gerencie com consistência os dados de seus usuários e execute os algoritmos de recomendação.

\section{Regra de negócio}

O núcleo funcional do sistema é a \textbf{Regra de Negócio para Sugestão de Trocas Alimentares}, projetada para operacionalizar a estratégia terapêutica de Encadeamento Alimentar (\textit{Food Chaining}). O objetivo do sistema é apoiar seus usuários — pessoas com TEA independentes ou cuidadores — a promoverem a expansão do repertório alimentar, o que é feito por meio de trocas que sejam sensorialmente compatíveis com alimentos já aceitos.

\subsection{Fundamentos e Modelo de Dados de Suporte}

A operacionalização desta regra é dependente do modelo de dados estruturado. Neste modelo, cada \textbf{Alimento Genérico} é classificado em um \textbf{Grupo Alimentar}, mas a lógica do sistema opera sobre \textbf{Preparações Culinárias Específicas} (e.g., ``Nugget de frango industrializado'').

Cada preparação possui \textbf{Atributos Sensoriais} (textura, formato, etc.) e uma \textbf{Classificação Nutricional} (e.g., processado, caseiro). Esta modelagem resulta em um perfil detalhado para cada item, que serve de base para o processamento do algoritmo.

\subsection{O Fluxo Operacional do Algoritmo de Sugestão}

O processo para geração de uma sugestão de troca alimentar é implementado por um algoritmo que executa as seguintes etapas sequenciais:

\begin{enumerate}
   \item \textbf{Identificação do Ponto de Partida:} O sistema analisa os dados do QFA para selecionar um ``Alimento Seguro'' do perfil do usuário, recuperando seu perfil sensorial e nutricional.
   \item \textbf{Definição da Meta Nutricional:} O algoritmo identifica os grupos alimentares com consumo ausente ou deficiente (conforme o QFA) e seleciona um deles como o ``grupo-meta'' para a intervenção.
   \item \textbf{Busca e Pontuação de Candidatos:} O algoritmo executa uma busca restrita às preparações pertencentes ao ``grupo-meta''. Para cada candidato, é calculada uma ``Pontuação de Recomendação'', composta por dois critérios: a \textbf{Similaridade Sensorial} com o ``Alimento Seguro'' e a \textbf{Melhora Nutricional}.
   \item \textbf{Geração da Lista Ordenada de Opções:} O sistema compila uma lista com os 3 a 4 candidatos que obtiveram as maiores Pontuações de Recomendação e a apresenta de forma ordenada.
\end{enumerate}

\subsection{Dinamismo e Adaptação: O Ciclo de Feedback}

O sistema é projetado para evoluir com o uso, através de um ciclo de feedback contínuo:

\begin{description}
   \item[Cenário de Aceitação] Ao registrar que uma sugestão foi aceita, o sistema promove esta preparação ao status de um novo ``Alimento Seguro''. Este processo enriquece o perfil do usuário e amplia a base sensorial para futuras recomendações.
   \item[Cenário de Rejeição] Caso a primeira sugestão da lista seja rejeitada, o sistema a marca para não ser oferecida novamente em curto prazo e apresenta a próxima opção da lista, mantendo o processo de interação fluido.
\end{description}

\section{Testes}

Com a conclusão do aplicativo, realizaram-se verificações para garantir o funcionamento correto de recursos específicos e da aplicação como um todo. O processo incluiu testes unitários, funcionais e de integração, descritos nas próximas seções.

\subsection{Testes Unitários}

Como indicado na metodologia, foi utilizado o Jest para realização dos testes unitários. Foram realizados 56 testes no Font-End e 25 no Back-End. A figura 4.4 e 4.5 indicam a cobertura de testes do código.

\begin{figure}[H]
   \centering
   \includegraphics[width=0.75\linewidth]{testes-back.png}
   \caption{Cobertura de testes do Back-End}
   \caption*{Fonte: Autor, 2025.}
   \label{fig:enter-label-paleta} % Label corrigido para ser único
\end{figure}

\begin{figure}[H]
   \centering
   \includegraphics[width=0.75\linewidth]{testes-front.jpeg}
   \caption{Cobertura de testes do Front-End}
   \caption*{Fonte: Autor, 2025.}
   \label{fig:enter-label-paleta} % Label corrigido para ser único
\end{figure}

% ----------------------------------------------------------
\section{Resultados das Métricas (GQM)}
% ----------------------------------------------------------

Este tópico apresenta os resultados da aferição do processo de desenvolvimento, com base no GQM (Goal-Question-Metric) definido no Capítulo 3 (Tabela \ref{tab:gqm_corrigido}). A tabela a seguir consolida os resultados obtidos na execução do projeto (TCC 2).

\begin{table}[H]
\centering
\caption{Resultados das Métricas GQM}
\label{tab:gqm_resultados}
\footnotesize % Reduz o tamanho da fonte para a tabela caber
\begin{tabularx}{\textwidth}{| 
    >{\raggedright\arraybackslash}p{3.2cm} | 
    >{\raggedright\arraybackslash}p{2.8cm} | 
    >{\raggedright\arraybackslash}p{2.0cm} | 
    >{\raggedright\arraybackslash}p{1.5cm} | 
    >{\raggedright\arraybackslash}p{1.8cm} | 
    X |}
\hline
\textbf{Métrica} & \textbf{Cálculo Realizado} & \textbf{Escala} & \textbf{Valor Obtido} & \textbf{Valor Esperado} & \textbf{Conclusão} \\ \hline

\multicolumn{6}{|l|}{\textbf{Objetivo 1: Gerenciamento do Backlog (Questões 1.1 e 1.2)}} \\ \hline
M 1.1.1 e 1.2.1: Tarefas concluídas do backlog & Contagem de tarefas (US + TK) finalizadas & Inteiro & 15 & 15 & O backlog planejado foi integralmente cumprido, atendendo às necessidades mapeadas. \\ \hline

\multicolumn{6}{|l|}{\textbf{Objetivo 2: Gerenciamento do Cronograma (Questão 2.1)}} \\ \hline
M 2.1.1: Funcionalidades mínimas do MVP & (Func. Entregues / Func. MVP Planejadas) & Percentual & 100\% & 100\% & O MVP foi entregue dentro do cronograma estipulado. \\ \hline

\multicolumn{6}{|l|}{\textbf{Objetivo 3: Progresso do Desenvolvimento (Questões 3.1, 3.2 e 3.3)}} \\ \hline
M 3.1: Taxa de bugs corrigidos & (Bugs Corrigidos / Total Bugs Encontrados) & Percentual & (16/17) 94\% & > 90\% & A qualidade foi mantida com a correção ágil das falhas. \\ \hline
M 3.2.1: Total de releases entregues & Contagem de releases/versões & Inteiro & 2 & >= 2 & Houve entregas incrementais e contínuas do software. \\ \hline
M 3.3.1: Débitos técnicos registrados & Contagem de débitos (To Do/Refactor) & Inteiro & 3 & < 5 & O endividamento técnico foi controlado e mantido em níveis abaixo do esperado. \\ \hline

\end{tabularx}
\caption*{Fonte: Autor, 2025.}
\end{table}


\section{Cronograma de Desenvolvimento}
O desenvolvimento da solução seguiu um cronograma estruturado em entregas incrementais, organizado para priorizar as funcionalidades essenciais do sistema. As atividades foram distribuídas ao longo do segundo semestre de 2025, conforme detalhado a seguir.

A fase inicial, na primeira quinzena de setembro, concentrou-se na implementação dos **Épicos 1 e 2**, estabelecendo as bases de gestão de identidade e o cadastro de perfis e assistidos. Em sequência, a segunda quinzena de setembro e a primeira de outubro foram dedicadas ao **Épico 3**, focado no mapeamento do repertório alimentar e sensorial, etapa fundamental para a entrada de dados no sistema.

Durante todo o mês de outubro, o desenvolvimento voltou-se para o **Épico 4**, o núcleo da solução, onde foram implementados o algoritmo de recomendação e os relatórios de evolução. O mês de novembro foi reservado para a garantia da qualidade: a primeira quinzena focou na validação da solução com os requisitos propostos, enquanto o restante do mês foi dedicado à execução de testes, refatoração de código e correções de erros. Por fim, a primeira quinzena de dezembro destinou-se à finalização da monografia e à preparação para a defesa.

\begin{table}[H]
\centering
\caption{Cronograma Executado de Desenvolvimento e Entregas (TCC 2).}
\label{tab:cronograma_tcc2_executado}
\resizebox{\textwidth}{!}{%
\begin{tabular}{|l|c|c|c|c|c|c|c|}
\hline
\textbf{} & \multicolumn{2}{c|}{\textbf{Setembro}} & \multicolumn{2}{c|}{\textbf{Outubro}} & \multicolumn{2}{c|}{\textbf{Novembro}} & \textbf{Dez} \\
\cline{2-8}
\textbf{Épico / Fase} & \textbf{1ª Q} & \textbf{2ª Q} & \textbf{1ª Q} & \textbf{2ª Q} & \textbf{1ª Q} & \textbf{2ª Q} & \textbf{1ª Q} \\
\hline
\textbf{1. Gestão de Identidade e Acesso} & \cellcolor{gray!30} & & & & & & \\
\hline
\textbf{2. Gerenciamento de Perfis e Assistidos} & & \cellcolor{gray!30} & & & & & \\
\hline
\textbf{3. Mapeamento do Repertório Alimentar} & & \cellcolor{gray!30} & \cellcolor{gray!30} & & & & \\
\hline
\textbf{4. Sistema de Recomendação e Análise} & & & \cellcolor{gray!30} & \cellcolor{gray!30} & & & \\
\hline
\textbf{5. Validação da Solução} & & & & & \cellcolor{gray!30} & & \\
\hline
\textbf{6. Testes e Refatoração} & & & & & \cellcolor{gray!30} & \cellcolor{gray!30} & \\
\hline
\textbf{7. Finalização TCC 2 e Apresentação} & & & & & & & \cellcolor{gray!30} \\
\hline
\end{tabular}%
}
\caption*{Fonte: Autor, 2025.}
\end{table}

% Nomenclaturas ajustadas para o capítulo 4 refatorado
\nomenclature[A]{UI}{User Interface}
\nomenclature[A]{UX}{User Experience}
\nomenclature[A]{N}{Número} % Para "1:N" e "N:M"
\nomenclature[A]{M}{Número} % Para "N:M"
\nomenclature[A]{QFA}{Questionário de Frequência Alimentar}