% ----------------------------------------------------------
\chapter{Introdução}
\label{cap:intr}
% ----------------------------------------------------------

Este capítulo apresenta a contextualização do trabalho, focado no desenvolvimento de uma aplicação móvel para apoiar cuidadores de pessoas com Transtorno do Espectro Autista (TEA) que possuem seletividade alimentar e também os próprios indivíduos que possuem TEA. O objetivo é desenvolver um software que auxilie na sugestão de trocas alimentares. A aplicação analisará informações fornecidas pelo usuário para oferecer alternativas adequadas às preferências e necessidades da pessoa com TEA. As seções seguintes detalham o problema, a justificativa do projeto, as questões de pesquisa e os objetivos que norteiam o desenvolvimento.

\section{Contextualização} \label{secao-contextualizacao}

A seletividade alimentar é um desafio prevalente entre crianças e adolescentes com Transtorno do Espectro Autista (TEA), com impacto na qualidade de vida dos indivíduos e de seus cuidadores. Estudos indicam que até 84\% das crianças com TEA apresentam padrões alimentares restritivos, como recusa de grupos alimentares, rigidez a texturas e cores, e consumo nutricional limitado \cite{sharp2018}. Tais comportamentos elevam o risco de deficiências nutricionais, como baixos níveis de cálcio e proteínas, e associam-se a comorbidades gastrointestinais e transtornos de comportamento \cite{leader2020}.

Embora existam abordagens terapêuticas eficazes, como as baseadas em princípios analítico-comportamentais \cite{taylor2017, ulloa2020}, sua aplicação no cotidiano representa um desafio para os cuidadores. As barreiras incluem dificuldade de acesso a orientação especializada, escassez de recursos práticos e insegurança na escolha de alimentos.

Os desafios alimentares no TEA também geram repercussões emocionais nos cuidadores, como ansiedade, estresse e sentimentos de impotência, derivados do manejo de recusas alimentares e da preocupação com a nutrição dos assistidos \cite{guller2024}.

Neste contexto, o trabalho propõe o desenvolvimento de um aplicativo móvel como ferramenta de apoio aos cuidadores e aos próprios portadores de TEA. A aplicação coletará informações por meio de formulários, cujo método será descrito no capítulo de Metodologia, para sugerir alternativas alimentares compatíveis. O objetivo é oferecer um recurso funcional e baseado em evidências científicas, apresentadas no Referencial Teórico, que amplie a autonomia dos usuários e contribua para a melhoria da rotina alimentar.

\section{Justificativa}

A pesquisa bibliográfica que fundamenta este trabalho revela uma lacuna no uso de tecnologias de software para apoiar os usuários na rotina alimentar de pessoas com TEA. Intervenções como estratégias de reforço ou orientações nutricionais personalizadas são, geralmente, restritas a ambientes clínicos e nem sempre disponíveis à maioria das famílias \cite{vazquez2019, ulloa2020}.

Mesmo com acompanhamento profissional, a aplicação das recomendações no dia a dia enfrenta barreiras. Fatores como falta de tempo, excesso de informações não sistematizadas e o estresse associado às recusas alimentares dificultam a implementação eficaz das orientações \cite{guller2024}.

A tecnologia móvel, já integrada à rotina de muitas famílias, oferece um meio para disponibilizar conhecimento e sistematizar o manejo alimentar. A ferramenta proposta busca promover a autonomia dos cuidadores e incentivar a diversificação alimentar de forma gradual.

Portanto, o projeto se justifica pela proposta de uma ferramenta digital que busca ampliar o conjunto de alimentos que são consumidos pelos portadores de TEA com seletividade alimentar, tudo isso através de uma contribuição tecnológica para auxiliar na área social.

\section{Questões de Pesquisa e Desenvolvimento}

Este Trabalho de Conclusão de Curso tem como objetivo principal desenvolver um aplicativo móvel que ofereça suporte aos seus usuários, auxiliando na identificação de alternativas alimentares mais adequadas e promovendo a diversificação da dieta dos indivíduos que pssuem TEA e seletividade alimentar. A proposta visa proporcionar uma solução tecnológica funcional, gratuita e disponível para os sistemas Android e iOS, baseada em dados coletados para subsidiar decisões alimentares no ambiente domiciliar.

Durante o processo de desenvolvimento do aplicativo, pretende-se responder à seguinte questão de pesquisa principal:

\begin{itemize}
\item Quais são as características necessárias em um aplicativo móvel que possa apoiar cuidadores de pessoas com TEA no manejo da seletividade alimentar, por meio da sugestão de trocas alimentares viáveis e personalizadas?
\end{itemize}

Além da questão central, este trabalho busca explorar as seguintes questões secundárias, que darão suporte teórico e prático à concepção da solução proposta:

\begin{itemize}
\item O que é seletividade alimentar no contexto de TEA?
\item Quais as características de tratamentos usados neste contexto?
\item Quais tecnologias envolvendo aplicativos de software têm sido usadas neste contexto?
\end{itemize}

Essas questões nortearão tanto a fundamentação teórica quanto as decisões de projeto envolvidas no desenvolvimento do aplicativo, assegurando que a solução proposta esteja alinhada às necessidades reais dos cuidadores e fundamentada em evidências técnico-científicas.

\section{Objetivos}

Esta seção apresenta o objetivo geral e os objetivos específicos deste trabalho de TCC.

\subsection{Objetivo Geral}

Desenvolver um aplicativo móvel como ferramenta de apoio tecnológico voltada a cuidadores de pessoas com Transtorno do Espectro Autista (TEA) que apresentam seletividade alimentar e também aos próprios indivíduos com TEA e seletividade alimentar.

\subsection{Objetivos Específicos}

\begin{enumerate}
\item Identificar os conceitos necessários ao desenvolvimento do aplicativo de software almejado.
\item Estabelecer uma ferramenta de software para apoiar os usuários com a questão da seletividade alimentar no contexto de TEA.
\item Estabelecer estudos de caso para avaliar a adequação do produto de software aos fins a que se dedica.
\end{enumerate}

\section{Organização da Monografia}

Este Trabalho de Conclusão de Curso está organizado nos seguintes capítulos:

\begin{itemize}
\item \textbf{Capítulo 2 - Referencial Teórico:} apresenta os fundamentos teóricos do trabalho, especialmente em relação à seletividade alimentar e o Transtorno do Espectro Autista, além de tópicos mais técnicos de Engenharia de Software (ex. engenharia de requisitos, arquitetura de software, metodologias de desenvolvimento, testes e qualidade);
\item \textbf{Capítulo 3 - Metodologias:} apresenta os aspectos metodológicos sobre o levantamento bibliográfico, o desenvolvimento do software e a análise de resultados;
\item \textbf{Capítulo 4 - Solução Implementada:} descreve em detalhes a solução que foi implementada por meio do desenvolvimento do um software dedicado;
\item \textbf{Capítulo 5 - Conclusão:} apresenta os resultados alcançados na primeira etapa do TCC, bem como retoma questionamentos e objetivos para conferir uma visão geral de como os mesmos foram tratados até o momento. Por fim, aborda detalhes sobre os próximos passos desse trabalho de software;
\end{itemize}

% Definição da nomenclatura que irá para a lista de siglas e abreviações

\nomenclature[A]{TEMAC}{Teoria do Enfoque Meta Analítico Consolidado}