
\chapter{Referencial Teórico}
%
Este capítulo subdivide-se em dois grandes grupos de conceitos teóricos. Inicialmente, o foco se dá nos conceitos envolvidos com TEA e seletividade alimentar; tratamentos atualmente usados e suas características; além das tecnologias envolvendo aplicativos de software que têm sido usadas para tanto. Um segundo ponto focal deste capítulo reside em aspectos de engenharia de software voltados ao desenvolvimento do aplicativo almejado. Para esse primeiro foco, foram usados princípios de mapeamentos sistemáticos de literatura expostos ao longo das próximas seções.

Este trabalho foi realizado com o apoio de uma profissional da área de nutrição com experiência no manejo de pacientes deste contexto, Paula Uessugue é nutricionista referência em nutrição materno-infantil no DF, com 20 anos de atuação acadêmica e clínica, mais de mil crianças atendidas, mestrado na área e reconhecimento público por sua contribuição em introdução alimentar, TEA e educação nutricional, além de ser professora e mentora de projetos acadêmicos.

%
\section{Seletividade Alimentar e TEA}
%

\subsection{Estabelecimento de conceitos teóricos}

Com o objetivo de embasar conceitualmente o desenvolvimento deste trabalho, foi realizada uma pesquisa bibliográfica exploratória com foco em estudos sobre seletividade alimentar em indivíduos com Transtorno do Espectro Autista (TEA).

A consulta foi realizada na base de dados Scopus, acessada por meio do portal \textit{Periódicos CAPES}. Foram elaboradas três estratégias de busca (expressão de busca), utilizando operadores booleanos e descritores combinados, a fim de garantir a abrangência e relevância dos resultados. Os descritores incluíram termos como \textit{“eating disorder”}, \textit{“food selectivity”}, \textit{“autism spectrum disorder”}, \textit{“technology”}, \textit{“software”} e \textit{“application”}.

\vspace{1em}

\begin{itemize}
    \item Expressão de busca final:
    
    \texttt{TITLE-ABS-KEY((Technology OR “Software application development” OR app) AND ("Food selectivity" OR "food refusal") AND ("autism spectrum disorder" OR tea OR “food exchange”))}
\end{itemize}

\vspace{1em}

Como critérios de seleção, foram considerados artigos publicados entre 2018 e 2024, escritos em inglês ou português, revisados por pares e com foco nas áreas de Medicina, Psicologia, Enfermagem, Neurociência e Sociologia. Foram excluídos trabalhos duplicados, artigos de revisão, resumos de conferência e publicações sem texto completo disponível.

A partir da expressão de busca, foram identificados 44 artigos relevantes, dos quais foram extraídas as principais contribuições teóricas sobre:

\begin{itemize}
    \item Caracterização de seletividade alimentar em pacientes portadores de TEA;
    \item Causas e implicações da seletividade alimentar;
    \item Lacunas na utilização de recursos tecnológicos no suporte ao tratamento alimentar;
\end{itemize}

\vspace{1em}

A expressão de busca, voltada à investigação de ferramentas tecnológicas, resultou em um número reduzido de publicações, indicando uma lacuna na literatura científica sobre o uso de aplicativos de software no contexto de seletividade alimentar em indivíduos com TEA. Os resultados são apresentados nas próximas seções que se seguem, procurando munir a equipe de desenvolvimento dos conceitos abordados na medida do necessário para o desenvolvimento do aplicativo de software almejado.

\subsection{Seletividade Alimentar no Contexto do TEA}
A seletividade alimentar é uma das características mais comuns observadas em pessoas com Transtorno do Espectro Autista (TEA) e representa um desafio constante para familiares, cuidadores e profissionais que atuam no suporte à alimentação. Em geral, crianças e adolescentes dentro do espectro tendem a apresentar padrões alimentares restritivos, marcados pela recusa frequente de determinados alimentos, preferência por marcas ou formas de preparo específicas e escolhas limitadas a certas cores, texturas ou temperaturas.

Essa condição está relacionada, em grande parte, a alterações no processamento sensorial, comuns no TEA, como hipersensibilidade ou hipossensibilidade a estímulos gustativos, olfativos e táteis. Esses fatores podem gerar desconforto ou aversão a alimentos que, para outras pessoas, seriam aceitáveis no dia a dia. Além disso, aspectos comportamentais, como apego a rotinas rígidas e dificuldade de aceitação de mudanças, contribuem para tornar a alimentação um momento desafiador e, muitas vezes, fonte de estresse para todos os envolvidos.

A consequência mais evidente da seletividade alimentar é o risco de uma dieta nutricionalmente desequilibrada, que pode resultar em deficiências de vitaminas, minerais e outros nutrientes importantes para o desenvolvimento saudável. A limitação alimentar persistente também impacta o convívio social, pois interfere em momentos como refeições em família ou atividades fora de casa.

\subsection{Transtorno do Espectro Autista}
O Transtorno do Espectro Autista (TEA) é uma condição do neurodesenvolvimento caracterizada por déficits na comunicação social e padrões repetitivos de comportamento, além de particularidades sensoriais. Essas características se manifestam em graus variados, o que justifica a utilização do termo “espectro”. Indivíduos com TEA frequentemente demonstram hipersensibilidade ou hipossensibilidade a estímulos sensoriais, como sons, luzes, texturas, cheiros e sabores, o que impacta diretamente diversos aspectos da vida cotidiana, incluindo a alimentação.

Diversos estudos destacam a relação entre essas sensibilidades e a seletividade alimentar. Estudos indicam que estímulos sensoriais específicos, como a textura crocante ou viscosa de certos alimentos, podem provocar reações de rejeição alimentar imediata em crianças com TEA \cite{ulloa2022}. Essa aversão vai além de uma preferência comum: trata-se de uma resposta sensorial intensa que pode desencadear comportamentos disruptivos, como choro, gritos ou evasão. Esses padrões de comportamento alimentar tornam o processo de alimentação um desafio significativo para familiares e profissionais de saúde.

Além disso, as dificuldades alimentares associadas ao TEA não estão necessariamente relacionadas à falta de apetite ou fome, mas sim à dificuldade de processar estímulos sensoriais complexos presentes nos alimentos \cite{bicer2020}. Dessa forma, a alimentação torna-se uma experiência estressante, que precisa ser gerida com estratégias específicas de apoio e compreensão do perfil sensorial do indivíduo.

\subsection{Tratamentos e Intervenções Comportamentais para Seletividade Alimentar no TEA}
O manejo da seletividade alimentar em indivíduos com Transtorno do Espectro Autista (TEA) exige uma abordagem estruturada, que considere os diferentes níveis de suporte necessários, o grau de seletividade apresentado e o uso de instrumentos de avaliação adequados para guiar o processo terapêutico.

Segundo o Manual Diagnóstico e Estatístico de Transtornos Mentais (DSM-5)\cite{americanpsychiatricassociation2013}, o TEA pode ser classificado em três níveis de suporte — variando de necessidade de apoio mínimo até apoio muito substancial — de acordo com a intensidade dos déficits de comunicação social e a presença de comportamentos restritivos.

Essa classificação tem impacto direto na forma como as intervenções alimentares são planejadas. Crianças que demandam maior suporte tendem a apresentar maior rigidez comportamental, menor tolerância a novos alimentos e resistência acentuada a mudanças na rotina alimentar. Assim, quanto mais elevado o nível de suporte, mais complexas e frequentes devem ser as estratégias de introdução de novos alimentos e reeducação nutricional, envolvendo acompanhamento contínuo de equipes multidisciplinares.

O grau de seletividade também varia amplamente entre indivíduos. Há casos em que a restrição alimentar é moderada, limitando-se à preferência por alguns grupos de alimentos, enquanto em quadros mais severos é comum a recusa quase total de categorias alimentares, com aceitação restrita a pouquíssimos itens. Tal comportamento seletivo compromete a variedade nutricional, podendo gerar carências de nutrientes, problemas de crescimento e até quadros paradoxais, como obesidade associada à desnutrição oculta, devido ao consumo excessivo de alimentos ultraprocessados e pobres em qualidade nutricional.

Para avaliar essas condições, são utilizadas escalas e instrumentos estruturados, como checklists, entrevistas com cuidadores e ferramentas padronizadas, a exemplo da Brief Autism Mealtime Behavior Inventory (BAMBI), que ajuda a identificar padrões de recusa, preferências específicas, frequência e intensidade de comportamentos durante as refeições. O uso dessas ferramentas possibilita monitorar a evolução da intervenção e planejar ajustes conforme a resposta da criança ao tratamento.

No que se refere às estratégias de intervenção, as abordagens baseadas nos princípios da Análise do Comportamento Aplicada (ABA) têm mostrado resultados consistentes na ampliação do repertório alimentar. Técnicas como reforço positivo, modelagem, dessensibilização sistemática, extinção de fuga e exposição gradual são aplicadas de forma individualizada, considerando as particularidades sensoriais de cada criança. O reforço positivo utiliza estímulos agradáveis, como elogios ou recompensas, para encorajar comportamentos desejáveis, como aceitar um novo alimento ou permanecer à mesa durante a refeição. A exposição repetida e gradual contribui para reduzir a rejeição inicial, promovendo o contato progressivo com alimentos que antes eram evitados.

Outro fator essencial para o sucesso dessas intervenções é o envolvimento da família. O trabalho conjunto entre profissionais, cuidadores e familiares possibilita criar um ambiente alimentar mais acolhedor, reforça a continuidade das práticas fora do ambiente clínico e reduz o estresse cotidiano relacionado à alimentação. Assim, a cooperação entre todos os envolvidos se torna indispensável para garantir uma intervenção eficaz, adaptada e sustentável ao longo do tempo.

\subsection{Uso de Tecnologia no Contexto}

Durante a pesquisa por tecnologias existentes, a solução que mais se aproximou de nossa proposta foi o aplicativo "Garfinho". Ele se posiciona como uma ferramenta de apoio à alimentação infantil, oferecendo planejamento de cardápios e receitas com foco em uma nutrição mais saudável. Contudo, uma análise aprofundada revela que seu propósito e público são distintos do nosso projeto. O "Garfinho" atende a um público geral, sem especialização nas complexidades do Transtorno do Espectro Autista (TEA). Sua abordagem não considera as questões sensoriais que são centrais na seletividade alimentar severa. Adicionalmente, seu modelo de negócio é baseado em uma assinatura paga, o que pode limitar o acesso para muitas famílias que já lidam com altos custos terapêuticos.

É para preencher essa lacuna que nosso aplicativo foi concebido. Diferente de uma solução genérica, nossa proposta é totalmente focada nas necessidades de pessoas com TEA e seus cuidadores. Em vez de impor dietas, o sistema opera com base em trocas alimentares personalizadas, partindo dos alimentos já aceitos pelo indivíduo para sugerir substituições graduais e seguras. Este método respeita o perfil de cada usuário, tornando o processo mais empático e com maior chance de sucesso. O objetivo é entregar uma ferramenta especializada, direcionada e mais acessível, oferecendo um suporte prático e específico que não encontramos no mercado atual.
\section{Engenharia de Software}
%

A Engenharia de Software é uma disciplina que aplica uma abordagem sistemática, disciplinada e quantificável ao desenvolvimento, operação e manutenção de software. Ela abrange um conjunto de métodos, ferramentas e procedimentos que visam produzir software de alta qualidade, que atenda aos requisitos do usuário, dentro do prazo e orçamento estabelecidos.

Segundo Pressman e Maxim (2021), a Engenharia de Software é um campo que evolui continuamente, buscando soluções para os desafios inerentes à construção de sistemas complexos. Suas principais áreas de atuação incluem:

\begin{itemize}

    \item \textbf{Processo de Software:} Define as atividades, tarefas e produtos de trabalho necessários para construir software. Modelos de processo como cascata, iterativos, incrementais e ágeis orientam o desenvolvimento.

    \item \textbf{Engenharia de Requisitos:} Foca na elicitação, análise, especificação e validação das necessidades dos usuários e das restrições do sistema.

    \item \textbf{Projeto de Software:} Envolve a criação da arquitetura, estrutura de dados e interfaces para os componentes do sistema.

    \item \textbf{Construção de Software:} Refere-se à codificação, testes unitários e depuração.

    \item \textbf{Teste de Software:} Garante que o software funcione corretamente e atenda aos requisitos, identificando defeitos.

    \item \textbf{Manutenção de Software:} Lida com as modificações necessárias após a entrega, incluindo correção de erros, melhorias e adaptações.

    \item \textbf{Gerência de Configuração de Software (GCS):} Controla as mudanças nos artefatos do projeto ao longo do tempo.

    \item \textbf{Gerência de Projetos de Software:} Planeja, organiza e controla os recursos para garantir que o projeto seja concluído com sucesso.

\end{itemize}

A aplicação dos princípios da Engenharia de Software é crucial para o sucesso de projetos de desenvolvimento, pois permite gerenciar a complexidade, garantir a qualidade, otimizar recursos e entregar valor aos usuários de forma consistente.

%
\section{Metodologias Ágeis}
%

As metodologias ágeis representam uma abordagem para o desenvolvimento de software que prioriza a flexibilidade, a colaboração, a entrega contínua de valor e a resposta rápida a mudanças. Surgiram como uma alternativa aos métodos tradicionais, que muitas vezes se mostravam rígidos e pouco adaptáveis a projetos com requisitos em evolução.

O Manifesto Ágil, publicado em 2001, estabeleceu os quatro valores fundamentais que norteiam essas metodologias:

\begin{itemize}

    \item Indivíduos e interações mais que processos e ferramentas.
    \item Software em funcionamento mais que documentação abrangente.
    \item Colaboração com o cliente mais que negociação de contratos.
    \item Responder a mudanças mais que seguir um plano.

\end{itemize}

Esses valores são complementados por doze princípios que detalham a forma de trabalho ágil, como a entrega frequente de software funcional, a colaboração diária entre desenvolvedores e clientes, a simplicidade e a auto-organização das equipes.

Dentre as diversas metodologias ágeis existentes, algumas das mais conhecidas incluem:

\begin{itemize}

    \item \textbf{Scrum:} Um framework iterativo e incremental que organiza o desenvolvimento em ciclos curtos e fixos (sprints), com papéis e eventos bem definidos (Product Owner, Scrum Master, Daily Scrum, Sprint Review, Sprint Retrospective).

    \item \textbf{Kanban:} Um método visual para gerenciar o fluxo de trabalho, utilizando quadros com colunas que representam os estágios das tarefas (a fazer, em andamento, feito), com foco na limitação do trabalho em progresso e na otimização do fluxo.

    \item \textbf{Extreme Programming (XP):} Uma metodologia que enfatiza a entrega contínua, testes frequentes, programação em pares, refatoração e feedback constante.

    \item \textbf{Lean Software Development:} Baseado nos princípios do Lean Manufacturing, foca na eliminação de desperdícios, na construção de qualidade e na entrega rápida.

\end{itemize}

As metodologias ágeis são amplamente adotadas na indústria de software devido à sua capacidade de promover maior adaptabilidade, reduzir riscos, aumentar a satisfação do cliente e melhorar a qualidade do produto final.

%
\section{Desenvolvimento Mobile}
%

O desenvolvimento mobile refere-se ao processo de criação de aplicativos de software para dispositivos móveis, como smartphones e tablets. Com a crescente ubiquidade desses aparelhos, a demanda por aplicações que atendam a diversas necessidades dos usuários tem impulsionado a evolução de tecnologias e abordagens específicas para esse segmento.

Existem três principais abordagens para o desenvolvimento mobile:

\begin{itemize}

    \item \textbf{Desenvolvimento Nativo:} Envolve a criação de aplicativos usando as linguagens de programação e ferramentas específicas de cada plataforma (por exemplo, Swift/Objective-C e Xcode para iOS; Java/Kotlin e Android Studio para Android). Aplicativos nativos oferecem o melhor desempenho, acesso total aos recursos do dispositivo e uma experiência de usuário alinhada com as diretrizes de cada sistema operacional. No entanto, exigem o desenvolvimento de bases de código separadas para iOS e Android, o que pode aumentar o tempo e o custo do projeto.

    \item \textbf{Desenvolvimento Híbrido:} Permite a criação de aplicativos que funcionam em múltiplas plataformas usando uma única base de código, geralmente com tecnologias web (HTML, CSS, JavaScript) encapsuladas em um contêiner nativo. Frameworks como Apache Cordova (PhoneGap) e Ionic são exemplos. Embora ofereçam agilidade no desenvolvimento e reuso de código, podem ter limitações de desempenho e acesso a recursos nativos em comparação com aplicativos puramente nativos.

    \item \textbf{Desenvolvimento Multiplataforma (Cross-Platform):} Aborda o desenvolvimento de aplicativos com uma única base de código que compila para código nativo ou se comunica com componentes nativos. Frameworks como React Native, Flutter e Xamarin se enquadram nesta categoria. Eles buscam combinar a eficiência de uma única base de código com o desempenho e a experiência de aplicativos nativos, utilizando pontes para acessar funcionalidades específicas do dispositivo.

\end{itemize}

A escolha da abordagem depende de fatores como o orçamento, o prazo, a complexidade do aplicativo, a necessidade de acesso a recursos nativos e a experiência da equipe de desenvolvimento. O desenvolvimento mobile continua a ser um campo dinâmico, com novas ferramentas e tendências surgindo constantemente para atender às demandas de um mercado em rápida expansão.

%
\section{Tecnologias de Desenvolvimento}
%

\subsection{React Native}

React Native é um framework de código aberto criado pelo Facebook para o desenvolvimento de aplicativos móveis nativos utilizando JavaScript e React. Ele permite que desenvolvedores construam interfaces de usuário ricas e performáticas para iOS e Android a partir de uma única base de código, o que acelera o processo de desenvolvimento e reduz a necessidade de equipes separadas para cada plataforma.

A principal característica do React Native é a capacidade de renderizar componentes da interface do usuário que são, de fato, componentes nativos da plataforma (como `UIView` no iOS ou `android.view` no Android), e não webviews. Isso resulta em aplicativos com desempenho e aparência nativos, proporcionando uma experiência de usuário fluida.

Benefícios do React Native incluem:

\begin{itemize}

    \item \textbf{Reuso de Código:} Grande parte do código JavaScript pode ser compartilhada entre as plataformas iOS e Android.
    \item \textbf{Hot Reloading e Fast Refresh:} Ferramentas que permitem visualizar as mudanças no código quase instantaneamente, agilizando o desenvolvimento.
    \item \textbf{Comunidade Ativa:} Uma vasta comunidade de desenvolvedores contribui com bibliotecas, ferramentas e suporte.
    \item \textbf{Acesso a Recursos Nativos:} Embora utilize JavaScript, é possível acessar APIs nativas do dispositivo quando necessário.

\end{itemize}

O React Native é amplamente utilizado por empresas de diversos portes, desde startups até grandes corporações, devido à sua eficiência e capacidade de entregar aplicativos de alta qualidade para múltiplas plataformas.

\subsection{Node.js}

Node.js é um ambiente de execução JavaScript de código aberto e multiplataforma, construído sobre o motor V8 do Google Chrome. Ele permite que desenvolvedores usem JavaScript para criar aplicações do lado do servidor (back-end), ferramentas de linha de comando e scripts, estendendo o uso da linguagem além do navegador.

A principal característica do Node.js é seu modelo de I/O não bloqueante e orientado a eventos, o que o torna extremamente eficiente e escalável para aplicações que lidam com muitas requisições simultâneas, como APIs RESTful, servidores de chat em tempo real e microsserviços.

Vantagens do Node.js:

\begin{itemize}

    \item \textbf{Performance:} Graças ao motor V8 e ao modelo assíncrono, o Node.js é muito rápido na execução de código JavaScript.
    \item \textbf{Ecossistema NPM:} Possui o maior ecossistema de bibliotecas de código aberto do mundo (npm - Node Package Manager), facilitando o desenvolvimento e a integração de funcionalidades.
    \item \textbf{JavaScript Full-Stack:} Permite que desenvolvedores utilizem a mesma linguagem (JavaScript) tanto no front-end (com React, Angular, Vue) quanto no back-end, simplificando o aprendizado e a troca de contexto.
    \item \textbf{Escalabilidade:} Ideal para construir aplicações escaláveis e de alta concorrência.

\end{itemize}

Node.js é uma escolha popular para o desenvolvimento de APIs, microsserviços, aplicações em tempo real e qualquer sistema que precise de alta performance e processamento assíncrono.

\subsection{Expo}

Expo é um framework e plataforma de código aberto que simplifica o desenvolvimento de aplicativos React Native. Ele fornece um conjunto de ferramentas e serviços que abstraem muitas das complexidades do desenvolvimento nativo, permitindo que os desenvolvedores se concentrem na lógica de negócios e na interface do usuário usando apenas JavaScript.

Com o Expo, é possível:

\begin{itemize}

    \item \textbf{Desenvolvimento Rápido:} Iniciar um projeto React Native sem a necessidade de configurar ambientes nativos (Xcode, Android Studio).
    \item \textbf{Testes Simplificados:} Testar aplicativos diretamente no dispositivo móvel ou em um emulador/simulador escaneando um QR code, sem a necessidade de compilações complexas.
    \item \textbf{Acesso a APIs Nativas:} Oferece uma vasta coleção de APIs nativas (câmera, localização, notificações, etc.) prontas para uso via JavaScript.
    \item \textbf{Over-the-Air (OTA) Updates:} Possibilita o envio de atualizações para o aplicativo sem a necessidade de submeter novas versões para as lojas de aplicativos.

\end{itemize}

O Expo é ideal para projetos que precisam de um desenvolvimento ágil e que não exigem acesso a módulos nativos muito específicos que não são suportados pelo Expo. Ele acelera significativamente o ciclo de desenvolvimento e implantação de aplicativos React Native.

\subsection{NativeWind}

NativeWind é uma biblioteca que traz a filosofia e a sintaxe do Tailwind CSS para o desenvolvimento de aplicativos React Native. O Tailwind CSS é um framework CSS "utility-first" que oferece uma vasta gama de classes utilitárias pré-definidas para construir interfaces de usuário diretamente no HTML (ou JSX, no caso do React Native), sem a necessidade de escrever CSS personalizado.

Com o NativeWind, os desenvolvedores podem aplicar estilos aos componentes React Native usando classes utilitárias do Tailwind, como `flex`, `pt-4`, `text-lg`, `bg-blue-500`, etc. Isso resulta em:

\begin{itemize}

    \item \textbf{Estilização Rápida:} Agiliza o processo de estilização, eliminando a necessidade de alternar entre arquivos JSX e folhas de estilo.
    \item \textbf{Consistência Visual:} Promove a consistência no design, utilizando um sistema de design baseado em tokens.
    \item \textbf{Manutenibilidade:} Facilita a manutenção, pois os estilos são aplicados diretamente onde são usados.
    \item \textbf{Otimização de Tamanho:} O NativeWind, assim como o Tailwind, pode ser configurado para "purificar" o CSS, removendo classes não utilizadas e otimizando o tamanho final do bundle.

\end{itemize}

NativeWind é uma excelente opção para equipes que já estão familiarizadas com Tailwind CSS ou que buscam uma abordagem de estilização mais rápida e consistente para seus projetos React Native.

\subsection{PostgreSQL}

PostgreSQL é um sistema gerenciador de banco de dados relacional (SGBDR) de código aberto, robusto, escalável e altamente extensível. É conhecido por sua confiabilidade, integridade de dados e conformidade com o padrão SQL. O PostgreSQL é frequentemente referido como "o banco de dados relacional de código aberto mais avançado do mundo".

Características e vantagens do PostgreSQL:

\begin{itemize}

    \item \textbf{Confiabilidade e Integridade:} Suporta transações ACID (Atomicidade, Consistência, Isolamento, Durabilidade), garantindo que os dados sejam sempre consistentes e seguros.
    \item \textbf{Extensibilidade:} Permite aos usuários definir seus próprios tipos de dados, operadores, funções agregadas e linguagens de procedimento.
    \item \textbf{Suporte a Dados Complexos:} Lida eficientemente com dados estruturados e não estruturados, incluindo JSON, XML, arrays, e tipos geométricos.
    \item \textbf{Comunidade Ativa:} Possui uma grande e ativa comunidade que contribui para seu desenvolvimento e oferece suporte.
    \item \textbf{Licença Permissiva:} Sua licença BSD permite o uso e modificação sem restrições significativas.

\end{itemize}

O PostgreSQL é uma escolha popular para uma vasta gama de aplicações, desde pequenos projetos até sistemas empresariais de grande escala, devido à sua capacidade de lidar com cargas de trabalho complexas e sua forte garantia de integridade de dados.

\subsection{Visual Studio Code}

Visual Studio Code (VS Code) é um editor de código-fonte leve, gratuito e multiplataforma desenvolvido pela Microsoft. Ele se tornou um dos editores de código mais populares entre desenvolvedores de diversas linguagens e tecnologias, incluindo JavaScript, TypeScript, Python, Java, C++, entre outras.

Principais características do VS Code:

\begin{itemize}

    \item \textbf{Extensibilidade:} Possui um vasto marketplace de extensões que adicionam funcionalidades, suporte a linguagens, temas, depuradores e ferramentas de desenvolvimento.
    \item \textbf{IntelliSense:} Oferece autocompletar inteligente baseado em tipos de variáveis, definições de funções e módulos importados.
    \item \textbf{Depuração Integrada:} Permite depurar o código diretamente no editor.
    \item \textbf{Controle de Versão Integrado:} Suporte nativo para Git, facilitando operações de commit, branch, merge, etc.
    \item \textbf{Terminal Integrado:} Um terminal de linha de comando embutido no editor.
    \item \textbf{Leve e Rápido:} Apesar de suas muitas funcionalidades, é conhecido por ser rápido e responsivo.

\end{itemize}

O VS Code é uma ferramenta versátil e poderosa que se adapta a diferentes fluxos de trabalho e preferências de desenvolvedores, tornando-o uma escolha excelente para o desenvolvimento de software moderno.

\subsection{GitHub}

GitHub é uma plataforma de hospedagem de código-fonte e arquivos com controle de versão usando Git. É a maior plataforma do mundo para desenvolvimento colaborativo de software, permitindo que milhões de desenvolvedores e equipes trabalhem juntos em projetos.

Funcionalidades chave do GitHub:

\begin{itemize}

    \item \textbf{Controle de Versão (Git):} Permite rastrear e gerenciar todas as alterações no código, facilitando a colaboração e o retorno a versões anteriores.
    \item \textbf{Repositórios:} Cada projeto é um repositório, que pode ser público ou privado.
    \item \textbf{Pull Requests:} Mecanismo para propor e revisar alterações no código antes de mesclá-las à branch principal.
    \item \textbf{Issues:} Ferramenta para rastrear bugs, funcionalidades e tarefas.
    \item \textbf{Projetos (Kanban Boards):} Quadros estilo Kanban para gerenciamento visual de tarefas.
    \item \textbf{Actions (CI/CD):} Ferramenta para automação de fluxos de trabalho, como integração contínua e entrega contínua.
    \item \textbf{Wiki e Páginas:} Para documentação do projeto.

\end{itemize}

O GitHub é essencial para equipes de desenvolvimento que buscam colaboração eficiente, rastreabilidade de código e automação de processos, sendo um padrão da indústria para o gerenciamento de projetos de software.

\subsection{Microsoft Teams}

Microsoft Teams é uma plataforma unificada de comunicação e colaboração desenvolvida pela Microsoft. Ela integra chat, reuniões de vídeo, armazenamento de arquivos (com integração com SharePoint e OneDrive) e integração de aplicativos, tudo em um único espaço de trabalho.

Recursos do Microsoft Teams:

\begin{itemize}

    \item \textbf{Chat e Canais:} Permite comunicação em tempo real em conversas individuais ou em canais de equipe, organizados por tópicos ou projetos.
    \item \textbf{Reuniões Online:} Funcionalidades completas para reuniões de vídeo e áudio, com compartilhamento de tela, gravação e transcrição.
    \item \textbf{Compartilhamento de Arquivos:} Facilita o compartilhamento e a coautoria de documentos.
    \item \textbf{Integrações:} Compatível com uma vasta gama de aplicativos e serviços de terceiros, além de toda a suíte Microsoft 365.
    \item \textbf{Segurança e Conformidade:} Oferece recursos avançados de segurança e privacidade para dados corporativos.

\end{itemize}

O Microsoft Teams é amplamente utilizado por empresas e instituições de ensino para melhorar a produtividade e a comunicação entre equipes, especialmente em ambientes de trabalho remoto ou híbrido.

% Nomenclaturas de Siglas e Abreviações (todas reunidas aqui)
\nomenclature[A]{TCC}{Trabalho de Conclusão de Curso}
\nomenclature[A]{TEA}{Transtorno do Espectro Autista}
\nomenclature[A]{BAMBI}{Inventário Breve de Comportamento Alimentar em Autismo}
\nomenclature[A]{ABA}{Análise do Comportamento Aplicada}
\nomenclature[A]{PO}{Product Owner} % Se "PO" for usado como sigla no texto (senão remova)
\nomenclature[A]{MoSCoW}{Must Have, Should Have, Could Have, Won't Have}
\nomenclature[A]{MVP}{Produto Mínimo Viável}
\nomenclature[A]{MVC}{Model-View-Controller} % Se "MVC" for usado como sigla no texto
\nomenclature[A]{RN}{React Native} % Se "RN" for usado como sigla no texto para React Native
\nomenclature[A]{API}{Interface de Programação de Aplicativos}
\nomenclature[A]{UI}{User Interface}
\nomenclature[A]{SCM}{Gestão de Configuração de Software}
\nomenclature[A]{CI}{Integração Contínua}
\nomenclature[A]{RDBMS}{Sistema de Gerenciamento de Banco de Dados Relacional}
\nomenclature[A]{NoSQL}{Não Apenas SQL}
\nomenclature[A]{IoT}{Internet das Coisas}
\nomenclature[A]{LOC}{Linhas de Código}
\nomenclature[A]{GQM}{Goal-Question-Metric}
\nomenclature[A]{XP}{Extreme Programming} % Se "XP" for usado como sigla no texto
\nomenclature[A]{pair programming}{Programação em par} % Se "pair programming" for abreviado
\nomenclature[A]{RN Testing Library}{React Native Testing Library} % Se "RN Testing Library" for abreviado
\nomenclature[A]{stakeholders}{Partes interessadas}