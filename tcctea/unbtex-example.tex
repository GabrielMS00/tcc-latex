%%
%% UnBTeX: A class for bachelor, master, and doctoral thesis at the
%% University of Brasilia (UnB), Brazil
%% Version 1.5.4 2025/01/10
%% Copyright (C) 2021-2025 by Henrique C. Ferreira <hcferreira@unb.br>
%%
%% This class file may be distributed and/or modified under the conditions
%% of the LaTeX Project Public License, either version 1.3 of this license
%% or (at your option) any later version. The latest version of this
%% license is in:
%% 
%%    https://www.latex-project.org/lppl.txt
%% 
%% and version 1.3 or later is part of all distributions of LaTeX version
%% 2005/12/01 or later.
%%
%% This file is a template for use with the UnBTex class
%% To compile the document you should call pdflatex, bibtex, pdflatex
%% 

\documentclass[
    % -- Opção da classe memoir -- https://www.ctan.org/pkg/memoir
    oneside, % Para imprimir somente na frente da folha
    %twoside, % Para imprimir na frente e no verso da folha
    % -- Opção da classe abntex2 -- https://www.ctan.org/pkg/abntex2
    sumario=tradicional, % Remova esta opção para sumário padrão ABNT
    % -- Selecione o idioma no qual o trabalho será escrito
    idioma=brazil, % Para texto principal em português
    %idioma=english, % Para texto principal em inglês  
    % -- Opções para as referências bibliográficas
    bib=alf, % Bibliografia nas normas da ABNT, estilo autor-data
    %bib=num, % Bibliografia nas normas da ABNT, estilo numérico
    refback, % Indica na bibliografia onde cada referência é citada
    % -- Selecione o estilo de numeração de figuras, tabelas, etc.
    numb=chap, % Numeração por capítulo
    %numb=abnt, % Numeração para o documento inteiro
    ]{unbtex}

% ---
% Pacotes básicos (Adicione outros pacotes necessários para o seu trabalho)
% ---
\usepackage{pdflscape}
\usepackage{rotating}
\usepackage{lscape} % Pacote para rotacionar tabelas (e outros objetos)
%\usepackage{pdflscape} % Pacote para rotacionar objetos e também a página do arquivo pdf
\usepackage{afterpage} % Evita quebra de página quando inserir uma tabela (ou outro objeto) rotacionada
% ---
\usepackage{float}
\definecolor{planned}{HTML}{D3D3D3} % Define a cor 'planned' como cinza claro
% ---
% Compila a nomenclatura
% ---
\makenomenclature
% ---

% ---
% Diretório das figuras
\graphicspath{{unbtex-example/figuras}}
% --- 

% ------------------------------------------------------------------------
% ------------------------------------------------------------------------
% Informações do trabalho
% ------------------------------------------------------------------------
% ------------------------------------------------------------------------

% ---
% Título
% ---
\titulo{Seletividade alimentar em portadores de TEA: Apoio tecnológico por software} % No idioma principal do texto
% Insira \\ caso queira forçar quebras de linha no título
% Não utilize caixa alta para o título do trabalho e nem das seções (com exceção de siglas)
% ---
\tituloestrangeiro{} % Escreva aqui título em português se o trabalho for escrito em inglês (caso contrário, deixe vazio)
% ---

% ---
% Autores
% ---
\autori[]{Caio Mesquita}{Vieira}% \autori[]{Nome}{Sobrenome}
% No caso de nomes como Carlos de Souza, utilize \autori[]{Carlos de}{Souza} (e não \autori[]{Carlos}{de Souza})
% ---
\autorii[]{Gabriel Marques de}{Souza} % Deixe os argumentos vazios se não tiver segundo autor
% ---

% ---
% Código Cutter para a ficha catalográfica
% Gerado a partir da entrada <Sobrenome, Nome> (do primeiro autor) no site https://www.tabelacutter.com/
% ---
\numerocutter{} % Para não imprimir o código na ficha catalográfica, deixe o argumento vazio
%\numerocutter{769} % Prencher o argumento do comando apenas com os números gerados
% ---

% ---
% Orientadores
% ---
\orientador[Orientador]{Prof. MSc.}{Ricardo Ajax Dias }{Kosloski} % Para alterar o gênero, basta trocar Orientador por Orientadora
% ---
\coorientador[Coorientadora]{Prof. Dra.}{Marília Miranda Forte}{Gomes} % Deixe os argumentos para nome e sobrenome vazios se não tiver coorientador 
% ---

% ---
% Informações do curso
% ---
\tipotrabalho{Trabalho de Conclusão de Curso} % Dissertação de Mestrado; Tese de Doutorado (em português, mesmo que o trabalho seja em inglês)
%\tipotrabalho{Tese de Doutorado}
% ---
\tipocurso{Engenharia de Software} % Nome do curso de graduação ou do programa de pós-graduação, em português
%\tipocurso{Programa de Pós-Graduação em Engenharia Elétrica}
% ---
% Texto que aparece na folha de rosto e na folha de aprovação
\preambulo{Trabalho de Conclusão de Curso submetido como requisito parcial para obtenção do grau de Engenheiro de Software.} 
%\preambulo{Tese de Doutorado submetida ao Programa de Pós-Graduação em Engenharia Elétrica da Universidade de Brasília como parte dos requisitos necessários para obtenção do grau de Doutor.}
% Consulte a secretaria/coordenação do curso para saber o que deve ser escrito no preâmbulo. Use português mesmo que o trabalho seja em inglês.
% ---
% Informação adicional para ser impressa na folha de rosto
\publicacao{} % Deixe o argumento vazio caso não haja
%\publicacao{Publicação PPGEE 201/23} % Também imprime as informações do trabalho no topo da página da ficha catalográfica
% ---

% ---
% Instituição
% ---
%\instituicao[Universidade de Brasília]{Faculdade de Tecnologia}{} % Use português mesmo que o trabalho seja em inglês
\instituicao[Universidade de Brasília]{FCTE}{Engenharia de Software}
% Caso queira incluir o departamento da unidade acadêmica
% ---

% ---
% Local e data da defesa
% ---
\local{Brasília}
\dia{18}
\mes{julho}
\ano{2025}
% ---

% ---
% Membros da banca
% ---
\membrodabancai{Prof. MSc. Ricardo Ajax Dias Kosloski ,\\ UnB/FCTE} % Membro 1 - Geralmente é o orientador
\membrodabancaifuncao{Orientador} % Em português, mesmo que o trabalho seja em inglês.
\membrodabancaii{Prof. Dra. Marília Miranda Forte Gomes ,\\ UnB/FCTE} % Membro 2
\membrodabancaiifuncao{Coorientadora}
\membrodabancaiii{Prof. Dr. Paula Uessugue ,\\ UnB/Darcy Ribeiro} % Membro 3
\membrodabancaiiifuncao{Examinador interno}
\membrodabancaiv{Prof. MSc. Cristiane Soares Ramos,\\ UnB/FCTE} % Deixe vazio se não tiver o quarto membro
\membrodabancaivfuncao{Examinador interno}
\membrodabancav{} % Deixe vazio se não tiver o quinto membro
\membrodabancavfuncao{Examinador externo}
% Comprimento da linha da assinatura (ajuste conforme necessidade)
\signlinewidth{9cm}
% ---

% ---
% Resumo em português
% ---
\begin{Resumo}
Este documento exemplifica a elaboração de trabalho acadêmico (trabalho de conclusão de curso, dissertação e tese) a partir da classe UnB\TeX, uma extensão da classe \abnTeX\ para a Universidade de Brasília (UnB). Além de apresentar comandos básico de \LaTeX\ para inclusão de equações, tabelas e figuras, o documento mostra como utilizar pacotes adotados pela classe UnB\TeX\ para gerar referências bibliográficas, listas símbolos, caixas para teoremas e algoritmos, dentre outros elementos úteis ou obrigatórios para trabalhos acadêmicos. Espera-se que este documento facilite o uso da classe UnB\TeX\ na elaboração de trabalhos de alta qualidade gráfica mesmo por usuários com pouca experiência em \LaTeX.
\end{Resumo}
% ---

% ---
% Resumo em inglês
% ---
\begin{Abstract}
This document demonstrates the preparation of academic works (such as final papers, dissertations, and theses) using the UnB\TeX\ class, an extension of the \abnTeX\ class developed for the University of Brasília (UnB). In addition to introducing basic \LaTeX\ commands for the inclusion of equations, tables, and figures, the document shows how to utilize packages integrated with the UnB\TeX\ class to generate bibliographic references, lists of symbols, and formatted boxes for theorems and algorithms, among other essential or useful elements for academic writing. The goal is to simplify the use of the UnB\TeX\ class, enabling even users with minimal \LaTeX\ experience to produce visually high-quality academic documents.
\end{Abstract}
% ---

% ---
% Palavras-chave (defina no mínimo 3 e no máximo 5)
% ---
% Palavras-chave
\pchavei{Transtorno do Espectro Autista}
\kwordi{Autism Spectrum Disorder} % Assumindo que kwordi é para a versão em inglês (keyword)
\pchaveii{Seletividade Alimentar}
\kwordii{Food Selectivity} % Assumindo que kwordii é para a versão em inglês (keyword)
\pchaveiii{Aplicativo Mobile}
\kwordiii{Mobile Application} % Assumindo que kwordiii é para a versão em inglês (keyword)
% ---
\pchaveiv{palavra-chave 4} % Deixe vazio se não tiver
\kwordiv{keyword 4} % Deixe vazio se não tiver
\pchavev{} % Deixe vazio se não tiver
\kwordv{} % Deixe vazio se não tiver
% ---

% ---
% Agradecimentos
% ---
% Idioma usado nos agradecimentos (pode ser em português, mesmo que o trabalho seja em inglês)
\idiomaagradecimentos{brazil}
%\idiomaagradecimentos{english}

\begin{otherlanguage*}{\acklang}

% Primeiro autor
\begin{AgradecimentosAutorI}

Agradeço primeiramente à minha família, pelo amor, apoio incondicional e por acreditarem em mim em cada etapa dessa jornada acadêmica. Vocês foram meu alicerce nos momentos difíceis e minha inspiração para seguir em frente.

À minha companheira, por todo amor, paciência e compreensão nos dias em que precisei me ausentar para estudar. Sua presença e incentivo foram fundamentais para que eu pudesse chegar até aqui.

Aos meus professores e orientadores, por toda dedicação, paciência e pelos ensinamentos compartilhados ao longo do curso e durante a construção deste trabalho. Cada orientação contribuiu diretamente para meu crescimento pessoal e profissional.

Ao meu parceiro de TCC, pela parceria, comprometimento e companheirismo em todas as fases do projeto. Sua dedicação tornou esta caminhada mais leve, produtiva e motivadora.

Por fim, Agradeco à Deus, que sem ele não seríamos nada.

A todos que, direta ou indiretamente, contribuíram para a realização deste trabalho, meu muito obrigado.

\end{AgradecimentosAutorI}

% Segundo autor
\begin{AgradecimentosAutorII}

Agradeço, em primeiro lugar, à minha família, pelo apoio incondicional, pelo incentivo constante aos meus estudos e por toda a assistência prestada ao longo desta caminhada. Sem o amor, o cuidado e os sacrifícios de vocês, não seria possível chegar tão longe. Foi graças à força e aos esforços de cada um que este trabalho se tornou realidade.

Aos professores que contribuíram para a minha formação acadêmica na Universidade, deixo meus mais sinceros agradecimentos. Em especial, aos professores orientadores deste trabalho, pela dedicação, paciência e orientação valiosa em cada etapa deste percurso.

Por fim, deixo um agradecimento especial ao meu amigo e parceiro de TCC, cuja colaboração foi essencial para a concretização deste trabalho. Sua amizade, companheirismo e apoio nos desafios da vida fizeram toda a diferença para que pudéssemos chegar até aqui.

\end{AgradecimentosAutorII}

\end{otherlanguage*}
% ---

% ---
% Dedicatória
% ---

% Primeiro autor
\begin{DedicatoriaAutorI}
Dedico este trabalho a todos que, de alguma forma, fizeram parte da minha trajetória.
Aos meus familiares, pelo amor incondicional e apoio constante.
Aos amigos, pela companhia nos momentos difíceis e pelas risadas nos dias leves.
E a todos que acreditaram em mim mesmo quando eu duvidei.
Este trabalho é resultado de muitos esforços compartilhados. 
\end{DedicatoriaAutorI}


% Segundo autor
\begin{DedicatoriaAutorII}
Dedico este trabalho à minha família, pelo amor incondicional, incentivo constante e por sempre acreditarem em meu potencial ao longo de toda a minha trajetória acadêmica. Agradeço, especialmente, a Deus, por ter me concedido força, sabedoria e perseverança para chegar até aqui. 
\end{DedicatoriaAutorII}
% ---

% ---
% Epígrafe
% ---
\begin{Epigrafe}
\vspace*{\fill}
\begin{flushright}
    \textit{``Se você quer ser bem-sucedido,\\ precisa ter dedicação total, \\buscar seu último limite e dar o melhor de si.''\\
    (Ayrton Senna)}
\end{flushright}
\end{Epigrafe}
% ---

% ------------------------------------------------------------------------
% ------------------------------------------------------------------------
% Início do documento
% ------------------------------------------------------------------------
% ------------------------------------------------------------------------
\begin{document}

% ------------------------------------------------------------------------
% ELEMENTOS PRÉ-TEXTUAIS
% ------------------------------------------------------------------------
\pretextual
% ------------------------------------------------------------------------

% Insere capa e contracapa
\imprimircapa

% Insere folha de rosto
\imprimirfolhaderosto*

% Insere ficha bibliográfica
\fichacatalografica

% Insere folha de aprovação
\imprimirfolhadeaprovacao

% Insere dedicatória (elemento opcional)
\imprimirdedicatoria

% Insere agradecimentos (elemento opcional)
\imprimiragradecimentos

% Insere epígrafe (elemento opcional)
\imprimirepigrafe

% Insere resumos
\imprimirresumos

% Insere lista de figuras
\imprimirlistadefiguras

% Insere lista de quadros
%\imprimirlistadequadros

% Insere lista de tabelas
\imprimirlistadetabelas

% Insere lista de algoritmos
%\imprimirlistadealgoritmos

% Insere lista de códigos
%\imprimirlistadecodigos

% Insere a lista de abreviaturas e siglas e a lista de símbolos
\imprimirlistadesiglasesimbolos

% Insere o sumário
\imprimirsumario

% ------------------------------------------------------------------------
% ELEMENTOS TEXTUAIS
% ------------------------------------------------------------------------
\textualsimples % Cabeçalho com número da página e linha horizontal
%\textual % Cabeçalho com número da página, título do capítulo/seção e linha horizontal
% ------------------------------------------------------------------------

% Imprime uma página para agrupar um conjunto de capítulos (parte)
%\part{Nome da parte}

% Capítulo de introdução
% ----------------------------------------------------------
\chapter{Introdução}
\label{cap:intr}
% ----------------------------------------------------------

Este capítulo apresenta a contextualização do trabalho, focado no desenvolvimento de uma aplicação móvel para apoiar cuidadores de pessoas com Transtorno do Espectro Autista (TEA) que possuem seletividade alimentar e também os próprios indivíduos que possuem TEA. O objetivo é desenvolver um software que auxilie na sugestão de trocas alimentares. A aplicação analisará informações fornecidas pelo usuário para oferecer alternativas adequadas às preferências e necessidades da pessoa com TEA. As seções seguintes detalham o problema, a justificativa do projeto, as questões de pesquisa e os objetivos que norteiam o desenvolvimento.

\section{Contextualização} \label{secao-contextualizacao}

A seletividade alimentar é um desafio prevalente entre crianças e adolescentes com Transtorno do Espectro Autista (TEA), com impacto na qualidade de vida dos indivíduos e de seus cuidadores. Estudos indicam que até 84\% das crianças com TEA apresentam padrões alimentares restritivos, como recusa de grupos alimentares, rigidez a texturas e cores, e consumo nutricional limitado \cite{sharp2018}. Tais comportamentos elevam o risco de deficiências nutricionais, como baixos níveis de cálcio e proteínas, e associam-se a comorbidades gastrointestinais e transtornos de comportamento \cite{leader2020}.

Embora existam abordagens terapêuticas eficazes, como as baseadas em princípios analítico-comportamentais \cite{taylor2017, ulloa2020}, sua aplicação no cotidiano representa um desafio para os cuidadores. As barreiras incluem dificuldade de acesso a orientação especializada, escassez de recursos práticos e insegurança na escolha de alimentos.

Os desafios alimentares no TEA também geram repercussões emocionais nos cuidadores, como ansiedade, estresse e sentimentos de impotência, derivados do manejo de recusas alimentares e da preocupação com a nutrição dos assistidos \cite{guller2024}.

Neste contexto, o trabalho propõe o desenvolvimento de um aplicativo móvel como ferramenta de apoio aos cuidadores e aos próprios portadores de TEA. A aplicação coletará informações por meio de formulários, cujo método será descrito no capítulo de Metodologia, para sugerir alternativas alimentares compatíveis. O objetivo é oferecer um recurso funcional e baseado em evidências científicas, apresentadas no Referencial Teórico, que amplie a autonomia dos usuários e contribua para a melhoria da rotina alimentar.

\section{Justificativa}

A pesquisa bibliográfica que fundamenta este trabalho revela uma lacuna no uso de tecnologias de software para apoiar os usuários na rotina alimentar de pessoas com TEA. Intervenções como estratégias de reforço ou orientações nutricionais personalizadas são, geralmente, restritas a ambientes clínicos e nem sempre disponíveis à maioria das famílias \cite{vazquez2019, ulloa2020}.

Mesmo com acompanhamento profissional, a aplicação das recomendações no dia a dia enfrenta barreiras. Fatores como falta de tempo, excesso de informações não sistematizadas e o estresse associado às recusas alimentares dificultam a implementação eficaz das orientações \cite{guller2024}.

A tecnologia móvel, já integrada à rotina de muitas famílias, oferece um meio para disponibilizar conhecimento e sistematizar o manejo alimentar. A ferramenta proposta busca promover a autonomia dos cuidadores e incentivar a diversificação alimentar de forma gradual.

Portanto, o projeto se justifica pela proposta de uma ferramenta digital que busca ampliar o conjunto de alimentos que são consumidos pelos portadores de TEA com seletividade alimentar, tudo isso através de uma contribuição tecnológica para auxiliar na área social.

\section{Questões de Pesquisa e Desenvolvimento}

Este Trabalho de Conclusão de Curso tem como objetivo principal desenvolver um aplicativo móvel que ofereça suporte aos seus usuários, auxiliando na identificação de alternativas alimentares mais adequadas e promovendo a diversificação da dieta dos indivíduos que pssuem TEA e seletividade alimentar. A proposta visa proporcionar uma solução tecnológica funcional, gratuita e disponível para os sistemas Android e iOS, baseada em dados coletados para subsidiar decisões alimentares no ambiente domiciliar.

Durante o processo de desenvolvimento do aplicativo, pretende-se responder à seguinte questão de pesquisa principal:

\begin{itemize}
\item Quais são as características necessárias em um aplicativo móvel que possa apoiar cuidadores de pessoas com TEA no manejo da seletividade alimentar, por meio da sugestão de trocas alimentares viáveis e personalizadas?
\end{itemize}

Além da questão central, este trabalho busca explorar as seguintes questões secundárias, que darão suporte teórico e prático à concepção da solução proposta:

\begin{itemize}
\item O que é seletividade alimentar no contexto de TEA?
\item Quais as características de tratamentos usados neste contexto?
\item Quais tecnologias envolvendo aplicativos de software têm sido usadas neste contexto?
\end{itemize}

Essas questões nortearão tanto a fundamentação teórica quanto as decisões de projeto envolvidas no desenvolvimento do aplicativo, assegurando que a solução proposta esteja alinhada às necessidades reais dos cuidadores e fundamentada em evidências técnico-científicas.

\section{Objetivos}

Esta seção apresenta o objetivo geral e os objetivos específicos deste trabalho de TCC.

\subsection{Objetivo Geral}

Desenvolver um aplicativo móvel como ferramenta de apoio tecnológico voltada a cuidadores de pessoas com Transtorno do Espectro Autista (TEA) que apresentam seletividade alimentar e também aos próprios indivíduos com TEA e seletividade alimentar.

\subsection{Objetivos Específicos}

\begin{enumerate}
\item Identificar os conceitos necessários ao desenvolvimento do aplicativo de software almejado.
\item Estabelecer uma ferramenta de software para apoiar os usuários com a questão da seletividade alimentar no contexto de TEA.
\item Estabelecer estudos de caso para avaliar a adequação do produto de software aos fins a que se dedica.
\end{enumerate}

\section{Organização da Monografia}

Este Trabalho de Conclusão de Curso está organizado nos seguintes capítulos:

\begin{itemize}
\item \textbf{Capítulo 2 - Referencial Teórico:} apresenta os fundamentos teóricos do trabalho, especialmente em relação à seletividade alimentar e o Transtorno do Espectro Autista, além de tópicos mais técnicos de Engenharia de Software (ex. engenharia de requisitos, arquitetura de software, metodologias de desenvolvimento, testes e qualidade);
\item \textbf{Capítulo 3 - Metodologias:} apresenta os aspectos metodológicos sobre o levantamento bibliográfico, o desenvolvimento do software e a análise de resultados;
\item \textbf{Capítulo 4 - Solução Implementada:} descreve em detalhes a solução que foi implementada por meio do desenvolvimento do um software dedicado;
\item \textbf{Capítulo 5 - Conclusão:} apresenta os resultados alcançados na primeira etapa do TCC, bem como retoma questionamentos e objetivos para conferir uma visão geral de como os mesmos foram tratados até o momento. Por fim, aborda detalhes sobre os próximos passos desse trabalho de software;
\end{itemize}

% Definição da nomenclatura que irá para a lista de siglas e abreviações

\nomenclature[A]{TEMAC}{Teoria do Enfoque Meta Analítico Consolidado}

% Capítulo de fundamentos teóricos ou técnicos
\chapter{Referencial Teórico}
%
Este capítulo apresenta o referencial teórico dividido em dois grupos principais. O primeiro aborda os conceitos de TEA, seletividade alimentar, tratamentos e o uso de tecnologias de \textit{software} neste contexto, enquanto o segundo grupo foca nos conceitos de engenharia de \textit{software} que nortearam o desenvolvimento da aplicação. Para a definição do primeiro grupo, foram utilizados mapeamentos sistemáticos da literatura, detalhados nas seções seguintes.

Este trabalho foi realizado com o apoio de uma profissional da área de nutrição com experiência no manejo de pacientes deste contexto. Paula Uessugue é nutricionista referência em nutrição materno-infantil no DF, com 20 anos de atuação acadêmica e clínica e mais de mil crianças atendidas. Possui reconhecimento público por sua contribuição em introdução alimentar, TEA e educação nutricional, além de ser professora e mentora de projetos acadêmicos.

%
\section{Seletividade Alimentar e TEA}
%

\subsection{Estabelecimento de conceitos teóricos}

Para embasar este trabalho, realizou-se uma pesquisa bibliográfica exploratória sobre seletividade alimentar em indivíduos com Transtorno do Espectro Autista (TEA).

A consulta foi feita na base Scopus (via portal \textit{Periódicos CAPES}), onde foram usadas três estratégias de busca com operadores booleanos e descritores combinados, como \textit{“eating disorder”}, \textit{“food selectivity”}, \textit{“autism spectrum disorder”}, \textit{“technology”}, \textit{“software”} e \textit{“application”}.

\vspace{1em}

\begin{itemize}
    \item Expressão de busca final:
    
    \texttt{TITLE-ABS-KEY((Technology OR “Software application development” OR app) AND ("Food selectivity" OR "food refusal") AND ("autism spectrum disorder" OR tea OR “food exchange”))}
\end{itemize}

\vspace{1em}

Como critérios de seleção, foram considerados artigos publicados entre 2018 e 2024, escritos em inglês ou português, revisados por pares e com foco nas áreas de Medicina, Psicologia, Enfermagem, Neurociência e Sociologia. Foram excluídos trabalhos duplicados, artigos de revisão, resumos de conferência e publicações sem texto completo disponível.

A expressão de busca identificou 44 artigos, dos quais foram extraídas contribuições teóricas sobre:

\begin{itemize}
    \item Caracterização de seletividade alimentar em pacientes portadores de TEA;
    \item Causas e implicações da seletividade alimentar;
    \item Lacunas na utilização de recursos tecnológicos no suporte ao tratamento alimentar;
\end{itemize}

\vspace{1em}

A busca por ferramentas tecnológicas retornou poucas publicações, indicando uma lacuna na literatura sobre o uso de \textit{software} para seletividade alimentar no TEA. Os resultados são apresentados nas seções seguintes e fundamentam o desenvolvimento da aplicação.

\subsection{Seletividade Alimentar no Contexto do TEA}
A seletividade alimentar é uma característica comum em pessoas com Transtorno do Espectro Autista (TEA) e representa um desafio para familiares, cuidadores e profissionais. Indivíduos no espectro tendem a apresentar padrões alimentares restritivos, marcados pela recusa de alimentos, preferência por marcas ou preparos específicos e escolhas limitadas por cor, textura ou temperatura.

Essa condição está relacionada a alterações no processamento sensorial (hipersensibilidade ou hipossensibilidade gustativa, olfativa e tátil), fatores que podem gerar desconforto ou aversão a alimentos. Além disso, aspectos comportamentais, como rigidez em rotinas e dificuldade com mudanças, contribuem para tornar a alimentação uma fonte de \textit{stress}.

A consequência da seletividade alimentar é o risco de uma dieta desequilibrada, podendo resultar em deficiências de nutrientes. A limitação alimentar persistente também impacta o convívio social, pois interfere em refeições familiares ou atividades externas.

\subsection{Transtorno do Espectro Autista}
O Transtorno do Espectro Autista (TEA) é uma condição do neurodesenvolvimento caracterizada por déficits na comunicação social, padrões repetitivos de comportamento e particularidades sensoriais. Tais características se manifestam em graus variados, o que justifica o termo “espectro” \cite{americanpsychiatricassociation2013}. Indivíduos com TEA frequentemente demonstram hipersensibilidade ou hipossensibilidade a estímulos (sons, luzes, texturas, cheiros, sabores), impactando diretamente a vida cotidiana, incluindo a alimentação.

Diversos estudos destacam a relação entre essas sensibilidades e a seletividade alimentar, indicando que estímulos sensoriais específicos, como texturas, podem provocar reações de rejeição alimentar em crianças com TEA \cite{ulloa2022}. Essa aversão não é uma preferência comum, mas sim uma resposta sensorial intensa que pode desencadear comportamentos como choro, gritos ou evasão. Esses padrões tornam o processo de alimentação um desafio para familiares e profissionais.

Além disso, as dificuldades alimentares associadas ao TEA não estão relacionadas à fome, mas sim à dificuldade de processar estímulos sensoriais dos alimentos \cite{bicer2020}. Dessa forma, a alimentação torna-se uma experiência estressante, que precisa ser gerida com estratégias de apoio baseadas no perfil sensorial do indivíduo.

\subsection{Tratamentos e Intervenções Comportamentais para Seletividade Alimentar no TEA}
O manejo da seletividade alimentar em indivíduos com TEA exige uma abordagem estruturada, que considere os níveis de suporte, o grau de seletividade e o uso de instrumentos de avaliação para guiar o processo terapêutico.

Segundo o Manual Diagnóstico e Estatístico de Transtornos Mentais (DSM-5) \cite{americanpsychiatricassociation2013}, o TEA pode ser classificado em três níveis de suporte — variando de apoio mínimo até apoio muito substancial — de acordo com a intensidade dos déficits de comunicação social e a presença de comportamentos restritivos.

Essa classificação impacta o planejamento das intervenções. Crianças que demandam maior suporte tendem a apresentar maior rigidez comportamental e menor tolerância a novos alimentos. Assim, quanto mais elevado o nível de suporte, mais complexas devem ser as estratégias de introdução de alimentos e reeducação nutricional, envolvendo acompanhamento contínuo de equipes multidisciplinares.

O grau de seletividade também varia amplamente. Há casos em que a restrição é moderada, limitando-se a alguns grupos de alimentos, enquanto em quadros mais severos é comum a recusa de categorias alimentares, com aceitação restrita a pouquíssimos itens. Tal comportamento compromete a variedade nutricional, podendo gerar carências de nutrientes ou obesidade associada à desnutrição, devido ao consumo de alimentos ultraprocessados.

Para avaliar essas condições, são utilizadas escalas e instrumentos, como \textit{checklists}, entrevistas com cuidadores e ferramentas padronizadas, a exemplo da \textit{Brief Autism Mealtime Behavior Inventory} (BAMBI). O BAMBI ajuda a identificar padrões de recusa, preferências e intensidade de comportamentos durante as refeições. O uso dessas ferramentas possibilita monitorar a evolução da intervenção e planejar ajustes.

No que se refere às estratégias, as abordagens baseadas na Análise do Comportamento Aplicada (ABA) têm mostrado resultados na ampliação do repertório alimentar. Técnicas como reforço positivo, modelagem, dessensibilização sistemática, extinção de fuga e exposição gradual são aplicadas de forma individualizada. O reforço positivo, por exemplo, utiliza estímulos para encorajar comportamentos (como aceitar um novo alimento), enquanto a exposição repetida e gradual contribui para reduzir a rejeição inicial.

O envolvimento da família é outro fator nas intervenções. O trabalho conjunto entre profissionais e cuidadores possibilita criar um ambiente alimentar de reforço, dá continuidade às práticas fora do ambiente clínico e reduz o \textit{stress} cotidiano. Assim, a cooperação entre todos os envolvidos garante uma intervenção adaptada e sustentável.

\subsection{Uso de Tecnologia no Contexto}

Durante a pesquisa por tecnologias existentes, a solução que mais se aproximou da proposta foi o aplicativo "Garfinho". Ele se posiciona como uma ferramenta de apoio à alimentação infantil, oferecendo planejamento de cardápios e receitas. Contudo, uma análise revela que seu propósito e público são distintos. O "Garfinho" atende a um público geral, sem especialização nas complexidades do Transtorno do Espectro Autista (TEA), e sua abordagem não considera as questões sensoriais centrais na seletividade alimentar. Adicionalmente, seu modelo de negócio é baseado em assinatura paga, o que pode limitar o acesso.

Deste modo, para preencher essa lacuna que este aplicativo foi concebido. Diferente de uma solução genérica, a proposta é focada nas necessidades de pessoas com TEA e seus cuidadores. Contrapondo-se à imposição de dietas, o sistema opera com base em trocas alimentares personalizadas, partindo dos alimentos já aceitos pelo indivíduo para sugerir substituições graduais. Este método respeita o perfil de cada usuário, tornando o processo mais adaptado. O objetivo é entregar uma ferramenta especializada e direcionada para este contexto.

Como parte da fundamentação deste trabalho, foi conduzida uma revisão da literatura focada nos avanços científicos e tecnológicos para o manejo da seletividade alimentar no Transtorno do Espectro Autista (TEA). Este estudo, desenvolvido em coautoria pelos autores desta monografia e submetido à Revista Políticas Públicas \& Cidades \cite{Uessugue2025avancos}, analisou a literatura recente para identificar lacunas e tendências de pesquisa, destacando a necessidade de desenvolver novas ferramentas tecnológicas que auxiliem cuidadores e pacientes, uma vez que a literatura ainda é incipiente na aplicação de \textit{software} para este problema.

\section{Engenharia de \textit{Software}}
%

A Engenharia de \textit{Software} aplica uma abordagem sistemática, disciplinada e quantificável ao desenvolvimento, operação e manutenção de \textit{software}. Ela abrange um conjunto de métodos, ferramentas e procedimentos que visam produzir \textit{software} que atenda aos requisitos, dentro do prazo e orçamento.

Segundo Pressman e Maxim \cite{pressman2021}, a Engenharia de \textit{Software} é um campo em evolução que busca soluções para a construção de sistemas complexos. Suas principais áreas de atuação incluem:

\begin{itemize}
    \item \textbf{Processo de \textit{Software}:} Define as atividades, tarefas e produtos de trabalho necessários. Modelos de processo (cascata, ágeis) orientam o desenvolvimento.
    \item \textbf{Engenharia de Requisitos:} Foca na elicitação, análise, especificação e validação das necessidades dos usuários e das restrições do sistema.
    \item \textbf{Projeto de \textit{Software}:} Envolve a criação da arquitetura, estrutura de dados e interfaces para os componentes do sistema.
    \item \textbf{Construção de \textit{Software}:} Refere-se à codificação, testes unitários e depuração.
    \item \textbf{Teste de \textit{Software}:} Garante que o \textit{software} funcione e atenda aos requisitos, identificando defeitos.
    \item \textbf{Manutenção de \textit{Software}:} Lida com as modificações necessárias após a entrega, incluindo correção de erros, melhorias e adaptações.
    \item \textbf{Gerência de Configuração de \textit{Software} (GCS):} Controla as mudanças nos artefatos do projeto ao longo do tempo.
    \item \textbf{Gerência de Projetos de \textit{Software}:} Planeja, organiza e controla os recursos para garantir que o projeto seja concluído.
\end{itemize}

A aplicação dos princípios da Engenharia de \textit{Software} permite gerenciar a complexidade, garantir a qualidade, otimizar recursos e entregar valor aos usuários.

%
\section{Metodologias Ágeis}
%

As metodologias ágeis representam uma abordagem para o desenvolvimento de \textit{software} que prioriza a flexibilidade, a colaboração, a entrega de valor e a resposta rápida a mudanças. Elas surgiram como uma alternativa aos métodos tradicionais, que se mostravam rígidos e pouco adaptáveis a projetos com requisitos em evolução.

O Manifesto Ágil, publicado em 2001 \cite{beck2001manifesto}, estabeleceu os quatro valores que norteiam essas metodologias:

\begin{itemize}
    \item Indivíduos e interações mais que processos e ferramentas.
    \item \textit{Software} em funcionamento mais que documentação abrangente.
    \item Colaboração com o cliente mais que negociação de contratos.
    \item Responder a mudanças mais que seguir um plano.
\end{itemize}

Esses valores são complementados por doze princípios que detalham a forma de trabalho ágil, como a entrega frequente de \textit{software} funcional, a colaboração diária, a simplicidade e a auto-organização das equipes.

Dentre as diversas metodologias ágeis, algumas das mais conhecidas incluem:

\begin{itemize}
    \item \textbf{\textit{Scrum}:} Um \textit{framework} iterativo e incremental que organiza o desenvolvimento em ciclos curtos (\textit{sprints}), com papéis e eventos bem definidos \cite{schwaber2020scrum}.
    \item \textbf{\textit{Kanban}:} Um método visual para gerenciar o fluxo de trabalho, utilizando quadros com colunas que representam os estágios das tarefas, com foco na limitação do trabalho em progresso.
    \item \textbf{\textit{Extreme Programming} (XP):} Uma metodologia que enfatiza a entrega contínua, testes frequentes, programação em pares, refatoração e \textit{feedback} constante.
    \item \textbf{\textit{Lean Software Development}:} Baseado nos princípios do \textit{Lean Manufacturing}, foca na eliminação de desperdícios, na construção de qualidade e na entrega rápida.
\end{itemize}

As metodologias ágeis são amplamente adotadas na indústria devido à sua capacidade de promover maior adaptabilidade, reduzir riscos e melhorar a qualidade do produto final.

%
\section{Desenvolvimento \textit{Mobile}}
%

O desenvolvimento \textit{mobile} refere-se ao processo de criação de aplicativos para dispositivos móveis, como \textit{smartphones} e \textit{tablets}. Com a crescente ubiquidade desses aparelhos, a demanda por aplicações tem impulsionado a evolução de tecnologias e abordagens para esse segmento.

Existem três principais abordagens para o desenvolvimento \textit{mobile}:

\begin{itemize}
    \item \textbf{Desenvolvimento Nativo:} Envolve a criação de aplicativos usando as linguagens e ferramentas específicas de cada plataforma (por exemplo, Swift para iOS; Java/Kotlin para Android). Aplicativos nativos oferecem o melhor desempenho e acesso total aos recursos do dispositivo, mas exigem o desenvolvimento de bases de código separadas, o que pode aumentar o tempo e o custo.
    \item \textbf{Desenvolvimento Híbrido:} Permite a criação de aplicativos que funcionam em múltiplas plataformas usando uma única base de código (geralmente tecnologias \textit{web} como HTML, CSS, JavaScript) encapsuladas em um \textit{container} nativo. \textit{Frameworks} como Cordova ou Ionic são exemplos. Embora ofereçam agilidade, podem ter limitações de desempenho e acesso a recursos nativos.
    \item \textbf{Desenvolvimento Multiplataforma (\textit{Cross-Platform}):} Aborda o desenvolvimento com uma única base de código que compila para código nativo ou se comunica com componentes nativos. \textit{Frameworks} como React Native e Flutter se enquadram nesta categoria. Eles buscam combinar a eficiência de uma única base de código com o desempenho nativo.
\end{itemize}

A escolha da abordagem depende de fatores como o orçamento, o prazo, a complexidade do aplicativo e a necessidade de acesso a recursos nativos.

%
\section{Tecnologias de Desenvolvimento}
%

\subsection{React Native}

React Native é um \textit{framework} de código aberto criado pelo Facebook para o desenvolvimento de aplicativos móveis nativos utilizando JavaScript e React \cite{reactnative}. Ele permite que desenvolvedores construam interfaces para iOS e Android a partir de uma única base de código.

A principal característica do React Native é a capacidade de renderizar componentes da interface do usuário que são, de fato, componentes nativos da plataforma, e não \textit{webviews}. Isso resulta em aplicativos com desempenho e aparência nativos.

Benefícios do React Native incluem:
\begin{itemize}
    \item \textbf{Reuso de Código:} Grande parte do código JavaScript pode ser compartilhada entre as plataformas iOS e Android.
    \item \textbf{\textit{Hot Reloading} e \textit{Fast Refresh}:} Ferramentas que permitem visualizar as mudanças no código quase instantaneamente.
    \item \textbf{Comunidade Ativa:} Vasta comunidade de desenvolvedores que contribui com bibliotecas e suporte.
    \item \textbf{Acesso a Recursos Nativos:} Possibilita acessar APIs nativas do dispositivo quando necessário.
\end{itemize}

\subsection{Node.js}

Node.js é um ambiente de execução JavaScript de código aberto e multiplataforma, construído sobre o motor V8 do Google Chrome \cite{nodejs}. Ele permite que desenvolvedores usem JavaScript para criar aplicações do lado do servidor (\textit{back-end}).

A característica do Node.js é seu modelo de I/O não bloqueante e orientado a eventos, o que o torna eficiente para aplicações que lidam com muitas requisições simultâneas, como APIs \textit{RESTful} e microsserviços.

Vantagens do Node.js:
\begin{itemize}
    \item \textbf{\textit{Performance}:} Graças ao motor V8 e ao modelo assíncrono, o Node.js é rápido na execução de código JavaScript.
    \item \textbf{Ecossistema NPM:} Possui o maior ecossistema de bibliotecas de código aberto (npm), facilitando o desenvolvimento.
    \item \textbf{JavaScript \textit{Full-Stack}:} Permite que desenvolvedores utilizem a mesma linguagem (JavaScript) tanto no \textit{front-end} quanto no \textit{back-end}.
    \item \textbf{Escalabilidade:} Ideal para construir aplicações escaláveis e de alta concorrência.
\end{itemize}

\subsection{Expo}

Expo é um \textit{framework} e plataforma de código aberto que simplifica o desenvolvimento de aplicativos React Native \cite{expo}. Ele fornece um conjunto de ferramentas e serviços que abstraem complexidades do desenvolvimento nativo, permitindo que os desenvolvedores se concentrem na lógica de negócios usando apenas JavaScript.

Com o Expo, é possível:
\begin{itemize}
    \item \textbf{Desenvolvimento Rápido:} Iniciar um projeto React Native sem a necessidade de configurar ambientes nativos (Xcode, Android Studio).
    \item \textbf{Testes Simplificados:} Testar aplicativos diretamente no dispositivo móvel escaneando um QR code.
    \item \textbf{Acesso a APIs Nativas:} Oferece uma vasta coleção de APIs nativas (câmera, localização, etc.) prontas para uso via JavaScript.
    \item \textbf{\textit{Over-the-Air} (OTA) \textit{Updates}:} Possibilita o envio de atualizações para o aplicativo sem a necessidade de submeter novas versões para as lojas.
\end{itemize}
O Expo é indicado para projetos que precisam de um desenvolvimento ágil e que não exigem acesso a módulos nativos muito específicos que não são suportados pelo \textit{framework}.

\subsection{NativeWind}

NativeWind é uma biblioteca que traz a sintaxe do Tailwind CSS para o desenvolvimento React Native \cite{nativewind}. O Tailwind CSS é um \textit{framework} "\textit{utility-first}" que oferece classes utilitárias para construir interfaces diretamente no JSX, sem a necessidade de escrever CSS personalizado.

Com o NativeWind, os desenvolvedores podem aplicar estilos usando classes utilitárias do Tailwind (como `flex`, `pt-4`, `text-lg`). Isso resulta em:

\begin{itemize}
    \item \textbf{Estilização Rápida:} Agiliza o processo de estilização, eliminando a necessidade de alternar entre arquivos JSX e folhas de estilo.
    \item \textbf{Consistência Visual:} Promove a consistência no \textit{design}, utilizando um sistema baseado em \textit{tokens}.
    \item \textbf{Manutenibilidade:} Facilita a manutenção, pois os estilos são aplicados diretamente onde são usados.
    \item \textbf{Otimização de Tamanho:} O NativeWind pode ser configurado para "purificar" o CSS, removendo classes não utilizadas.
\end{itemize}

\subsection{PostgreSQL}

PostgreSQL é um sistema gerenciador de banco de dados relacional (SGBDR) de código aberto, conhecido por sua confiabilidade, integridade de dados e conformidade com o padrão SQL \cite{postgresql}.

Características e vantagens do PostgreSQL:
\begin{itemize}
    \item \textbf{Confiabilidade e Integridade:} Suporta transações ACID (Atomicidade, Consistência, Isolamento, Durabilidade).
    \item \textbf{Extensibilidade:} Permite aos usuários definir seus próprios tipos de dados, operadores e funções.
    \item \textbf{Suporte a Dados Complexos:} Lida eficientemente com dados estruturados e não estruturados, incluindo JSON, XML e \textit{arrays}.
    \item \textbf{Comunidade Ativa:} Possui uma grande comunidade que contribui para seu desenvolvimento.
    \item \textbf{Licença Permissiva:} Sua licença permite o uso e modificação sem restrições significativas.
\end{itemize}

\subsection{Visual Studio Code}

Visual Studio Code (VS Code) é um editor de código-fonte leve, gratuito e multiplataforma desenvolvido pela Microsoft \cite{vscode}. Tornou-se um dos editores mais populares entre desenvolvedores de diversas linguagens.

Principais características do VS Code:
\begin{itemize}
    \item \textbf{Extensibilidade:} Possui um vasto \textit{marketplace} de extensões que adicionam funcionalidades, suporte a linguagens e ferramentas.
    \item \textbf{\textit{IntelliSense}:} Oferece autocompletar inteligente baseado em tipos de variáveis e definições de funções.
    \item \textbf{Depuração Integrada:} Permite depurar o código diretamente no editor.
    \item \textbf{Controle de Versão Integrado:} Suporte nativo para Git.
    \item \textbf{Terminal Integrado:} Um terminal de linha de comando embutido no editor.
    \item \textbf{Leve e Rápido:} Conhecido por ser rápido e responsivo.
\end{itemize}

\subsection{GitHub}

GitHub é uma plataforma de hospedagem de código-fonte e arquivos com controle de versão usando Git \cite{github}. É a maior plataforma para desenvolvimento colaborativo de \textit{software}.

Funcionalidades chave do GitHub:
\begin{itemize}
    \item \textbf{Controle de Versão (Git):} Permite rastrear e gerenciar todas as alterações no código.
    \item \textbf{Repositórios:} Cada projeto é um repositório, que pode ser público ou privado.
    \item \textbf{\textit{Pull Requests}:} Mecanismo para propor e revisar alterações no código.
    \item \textbf{\textit{Issues}:} Ferramenta para rastrear \textit{bugs}, funcionalidades e tarefas.
    \item \textbf{Projetos (\textit{Kanban Boards}):} Quadros estilo \textit{Kanban} para gerenciamento visual de tarefas.
    \item \textbf{\textit{Actions} (CI/CD):} Ferramenta para automação de fluxos de trabalho, como integração e entrega contínua.
    \item \textbf{\textit{Wiki} e Páginas:} Para documentação do projeto.
\end{itemize}

\subsection{Microsoft Teams}

Microsoft Teams é uma plataforma unificada de comunicação e colaboração desenvolvida pela Microsoft \cite{msteams}. Ela integra \textit{chat}, reuniões de vídeo, armazenamento de arquivos e integração de aplicativos.

Recursos do Microsoft Teams:
\begin{itemize}
    \item \textbf{\textit{Chat} e Canais:} Permite comunicação em tempo real em conversas individuais ou em canais de equipe.
    \item \textbf{Reuniões \textit{Online}:} Funcionalidades completas para reuniões de vídeo e áudio, com compartilhamento de tela e gravação.
    \item \textbf{Compartilhamento de Arquivos:} Facilita o compartilhamento e a coautoria de documentos.
    \item \textbf{Integrações:} Compatível com uma vasta gama de aplicativos, além de toda a suíte Microsoft 365.
    \item \textbf{Segurança e Conformidade:} Oferece recursos avançados de segurança e privacidade para dados corporativos.
\end{itemize}

\subsection{Docker}

Docker é uma plataforma de código aberto que automatiza a implantação, o dimensionamento e o gerenciamento de aplicações dentro de \textit{containers} \cite{docker}. A tecnologia permite empacotar uma aplicação com todas as suas dependências — como bibliotecas, código e ambiente de execução — em uma unidade padronizada.

O objetivo principal do Docker é garantir a consistência entre múltiplos ambientes. Ele resolve o problema de "funciona na minha máquina" ao assegurar que o \textit{software} se comporte da mesma maneira em desenvolvimento, testes e produção.

Os principais benefícios do Docker são:
\begin{itemize}
    \item \textbf{Portabilidade:} Um \textit{container} pode ser executado em qualquer máquina que tenha o Docker instalado, independentemente do sistema operacional subjacente.
    \item \textbf{Isolamento:} Aplicações em \textit{containers} rodam de forma isolada, impedindo que uma aplicação interfira em outra.
    \item \textbf{Eficiência:} São mais leves que máquinas virtuais tradicionais, pois compartilham o \textit{kernel} do sistema operacional \textit{host}, resultando em inicialização rápida e menor consumo de recursos.
    \item \textbf{Reprodutibilidade:} O uso de um \textit{Dockerfile} permite definir a construção do ambiente de forma programática e versionável.
\end{itemize}

% Nomenclaturas de Siglas e Abreviações (todas reunidas aqui)
\nomenclature[A]{TCC}{Trabalho de Conclusão de Curso}
\nomenclature[A]{TEA}{Transtorno do Espectro Autista}
\nomenclature[A]{BAMBI}{Inventário Breve de Comportamento Alimentar em Autismo}
\nomenclature[A]{ABA}{Análise do Comportamento Aplicada}
\nomenclature[A]{PO}{Product Owner} % Se "PO" for usado como sigla no texto (senão remova)
\nomenclature[A]{MoSCoW}{Must Have, Should Have, Could Have, Won't Have}
\nomenclature[A]{MVP}{Produto Mínimo Viável}
\nomenclature[A]{MVC}{Model-View-Controller} % Se "MVC" for usado como sigla no texto
\nomenclature[A]{RN}{React Native} % Se "RN" for usado como sigla no texto para React Native
\nomenclature[A]{API}{Interface de Programação de Aplicativos}
\nomenclature[A]{UI}{User Interface}
\nomenclature[A]{SCM}{Gestão de Configuração de Software}
\nomenclature[A]{CI}{Integração Contínua}
\nomenclature[A]{RDBMS}{Sistema de Gerenciamento de Banco de Dados Relacional}
\nomenclature[A]{NoSQL}{Não Apenas SQL}
\nomenclature[A]{IoT}{Internet das Coisas}
\nomenclature[A]{LOC}{Linhas de Código}
\nomenclature[A]{GQM}{Goal-Question-Metric}
\nomenclature[A]{XP}{Extreme Programming} % Se "XP" for usado como sigla no texto
\nomenclature[A]{pair programming}{Programação em par} % Se "pair programming" for abreviado
\nomenclature[A]{RN Testing Library}{React Native Testing Library} % Se "RN Testing Library" for abreviado
\nomenclature[A]{stakeholders}{Partes interessadas}

% Capítulo com a proposta desenvolvida
% ----------------------------------------------------------
\chapter{Metodologias}
%

%
\section{Metodologia de pesquisa}
%

Para o desenvolvimento desta pesquisa, foram adotadas abordagens metodológicas que estruturam de forma sistemática os métodos de coleta, análise e interpretação dos dados, garantindo consistência e clareza em todas as etapas do estudo.

Em relação à abordagem de pesquisa, adota-se o método \textbf{qualitativo}, uma vez que ele possibilita compreender em profundidade as percepções e necessidades dos usuários diretos, contribuindo para a construção de um backlog que reflita fielmente essas demandas.

Quanto à natureza, caracteriza-se como uma \textbf{pesquisa aplicada}, pois visa não apenas gerar conhecimento teórico, mas também utilizá-lo para resolver um problema prático por meio do desenvolvimento de um software funcional, baseado em requisitos reais identificados durante o processo.

No que se refere aos objetivos, trata-se de uma \textbf{pesquisa exploratória}, já que busca aprofundar a compreensão do fenômeno investigado — as necessidades específicas do público-alvo — de modo a embasar a proposição de uma solução tecnológica adequada.

No que tange aos procedimentos técnicos, optou-se por uma combinação de \textbf{pesquisa bibliográfica}, \textbf{pesquisa de campo} e \textbf{estudo de caso}. A pesquisa bibliográfica fornecerá o embasamento teórico necessário para sustentar as decisões de desenvolvimento, enquanto a pesquisa de campo e o estudo de caso permitirão o contato direto com a realidade dos usuários, ampliando o entendimento de suas demandas específicas.

Por fim, como procedimento de coleta de dados, será utilizada a \textbf{observação}, ferramenta essencial para registrar comportamentos, interações e contextos de uso do sistema, possibilitando um diagnóstico mais preciso das necessidades que o software deverá atender.

\begin{figure}[H]
    \centering
    \includegraphics[width=0.75\linewidth]{metodologias_pesquisa.png}
    \caption{Metodologia da Pesquisa}
    \caption*{Fonte: Autor, 2025.}
    \label{fig:enter-label}
\end{figure}

%
\section{Metodologia de Desenvolvimento de Software}
%

\subsection{Ciclo de Vida}

No desenvolvimento deste software, tornou-se essencial, antes de qualquer implementação, compreender profundamente o contexto do problema e, a partir disso, selecionar um ciclo de vida adequado, que orientasse as etapas e atividades necessárias para a construção do sistema. A abordagem escolhida foi fundamentada nos princípios das metodologias ágeis, com ênfase no Scrum, devido à sua capacidade de estruturar o trabalho de forma iterativa, incremental e centrada na entrega contínua de valor.

Contudo, algumas adaptações foram realizadas para adequar o framework às características específicas do projeto. Uma dessas adaptações foi a não realização das daily meetings, reuniões diárias propostas pelo Scrum para promover alinhamento constante da equipe. Tal decisão ocorreu considerando a dinâmica enxuta do desenvolvimento, na qual a comunicação se deu predominantemente de forma assíncrona ou sob demanda, sempre que surgirem necessidades específicas.

Já a estrutura tradicional de papéis definida pelo Scrum — como Scrum Master, Development Team e Product Owner — foi adaptada para a realidade deste projeto. Em função do tamanho reduzido da equipe de desenvolvimento, optou-se por não designar um Scrum Master. Por outro lado, o papel de Product Owner (PO) foi mantido, sendo desempenhado pela Professora Paula Uessugue, que assume a responsabilidade de representar as necessidades dos usuários e priorizar os requisitos do produto.

Além das práticas do Scrum, este projeto incorporou também técnicas derivadas do Extreme Programming (XP), com o objetivo de aprimorar a qualidade do código e aumentar a produtividade. Dentre as práticas adotadas do XP, destacam-se a programação em pares (pair programming), que favorece a troca constante de conhecimento e a redução de erros; a refatoração contínua, voltada à melhoria incremental da estrutura do código; e a ênfase na simplicidade, buscando sempre soluções diretas e eficazes para os problemas encontrados. A adoção dessas práticas contribui para um desenvolvimento mais sustentável, com foco na manutenibilidade, qualidade e agilidade.

A figura 3.2 representa todo o ciclo de vida do projeto, estruturado com base nos artefatos e eventos definidos. A formação do backlog da sprint parte do backlog do produto e ocorre mediante uma priorização que considera o valor de negócio atribuído na etapa de levantamento e organização dos requisitos.

\begin{figure} [H]
    \centering
    \includegraphics[width=1\linewidth]{ciclo-de-vida-scrum.png}
    \caption{Ciclo de vida}
    \caption*{Fonte: Autor, 2025.}
    \label{fig:enter-label}
\end{figure}

\subsection{Artefatos}

Ao longo do desenvolvimento do projeto, fez-se indispensável a utilização de determinados artefatos derivados tanto do framework Scrum quanto da metodologia visual Kanban, com o objetivo de organizar, acompanhar e gerir o progresso das atividades de maneira eficiente e estruturada.

Entre os principais artefatos empregados, destaca-se o \textbf{Backlog do Produto}, que consiste em uma lista priorizada contendo todas as funcionalidades, melhorias e requisitos identificados para o sistema. Este backlog representa a visão global do produto, servindo como fonte principal para seleção das demandas que seriam trabalhadas em cada ciclo de desenvolvimento (Sprint).

Outro artefato essencial foi o \textbf{Backlog da Sprint}, que corresponde a um subconjunto do backlog do produto, contendo exclusivamente as tarefas e funcionalidades selecionadas para serem desenvolvidas durante uma sprint específica. Este documento é atualizado de forma contínua, refletindo o avanço das tarefas, as mudanças nas prioridades e eventuais ajustes identificados durante a execução.

\subsection{Fases de Desenvolvimento}

\begin{itemize}

    \item \textbf{Revisão e Planejamento da Sprint:} Ao final de cada ciclo semanal, foram discutidos os principais acontecimentos do ciclo, identificando os pontos positivos, os desafios enfrentados e as oportunidades de melhoria. Com base nessa análise, foi possível aprimorar as estimativas de esforço e a organização das tarefas futuras. Simultaneamente, foram definidos os itens do Backlog do Produto que foram priorizados para compor o Backlog da Sprint subsequente, sempre considerando a ordem de prioridade estabelecida. Quando necessário, ajustes também foram feitos no próprio backlog do produto, refletindo aprendizados e necessidades observadas nos incrementos anteriores.

    \item \textbf{Execução da Sprint:} Durante a sprint, o desenvolvimento foi concentrado nas funcionalidades planejadas no backlog da sprint. A comunicação entre os integrantes foi realizada predominantemente de maneira assíncrona, permitindo que eventuais dúvidas ou problemas fossem resolvidos conforme foram surgindo. 

    \item \textbf{Programação em Pares (Pair Programming):} Inspirada nas práticas do Extreme Programming (XP), a técnica de programação em pares foi adotada sempre que se mostrou pertinente ao longo das semanas de desenvolvimento. Essa prática consiste em dois desenvolvedores trabalhando simultaneamente sobre o mesmo trecho de código, colaborando na resolução de problemas, no compartilhamento de conhecimentos e na melhoria da qualidade do software. Esse método demonstrou-se especialmente útil na superação de dificuldades técnicas e na aceleração do desenvolvimento.

    \item \textbf{Testes:} Paralelamente à implementação das funcionalidades, foi conduzida a criação dos testes correspondentes, alinhados às boas práticas de desenvolvimento. Dependendo da natureza e da criticidade de cada funcionalidade, foram aplicados testes unitários, de integração ou de sistema. Essa estratégia visou assegurar que cada parte do software esteja funcionando corretamente, bem como garantir a robustez e a confiabilidade do aplicativo como um todo.

\end{itemize}

\vspace{1em}

%
\subsection{Requisitos}
%

Para que o projeto de software fosse desenvolvido de forma eficiente e alinhada às expectativas dos usuários, tornou-se fundamental, em um primeiro momento, realizar a identificação dos seus requisitos, ou seja, das especificações que definem as funcionalidades, comportamentos e restrições do sistema. Esses requisitos serviram como base para orientar todo o processo de desenvolvimento, garantindo que o produto final atendesse às necessidades para as quais foi concebido. Abaixo é detalhado como cada fase da engenharia de requisitos foi executada no contexto deste TCC.

\vspace{1em}

\textbf{Elicitação e análise}

\vspace{1em}

A etapa de elicitação dos requisitos foi conduzida por meio de uma entrevista estruturada com a professora Paula Uessugue, docente do curso de Nutrição da Universidade de Brasília (UnB), que também atua como Product Owner (PO) para o desenvolvimento do aplicativo. O principal objetivo dessa entrevista foi compreender, de maneira aprofundada, quais são as necessidades dos usuários no contexto de utilização do sistema, bem como identificar quais instrumentos e questionários seriam aplicados para o registro das informações dentro da aplicação.

Além disso, buscou-se entender de forma clara quais seriam os resultados esperados a partir do preenchimento desses questionários, tanto no que se refere ao processamento dos dados quanto à geração das sugestões alimentares. Também foi discutida a maneira mais adequada de apresentar esses resultados aos usuários na interface do aplicativo, garantindo clareza e alinhamento com as demandas práticas dos usuários.

\vspace{1em}

\textbf{Especificação}

\vspace{1em}

Após a realização das entrevistas, todas as informações coletadas foram organizadas e consolidadas na forma de requisitos, que foram devidamente documentados em um arquivo no formato Markdown, hospedado no repositório do projeto no GitHub. Esse documento serviu como base inicial para a formalização dos requisitos, sendo continuamente atualizado conforme surgiam novos entendimentos e validações ao longo do desenvolvimento. Assim sendo, este material foi estruturado como Backlog do Produto, que passou a desempenhar o papel central na gestão das funcionalidades, priorização de tarefas e acompanhamento do progresso do projeto.

\vspace{1em}

\textbf{Validação}

\vspace{1em}

Na fase de validação dos requisitos, foram aplicadas práticas recomendadas na engenharia de software, conforme proposto por \cite{sommerville1997}. Esse processo teve como objetivo assegurar que os requisitos definidos estivessem corretos, completos e alinhados com as necessidades do projeto. Dentre as atividades realizadas, destacam-se:

\begin{itemize}

    \item Análise de Consistência: Procedimento destinado a verificar se os requisitos estavam livres de conflitos, contradições, ambiguidades ou duplicidades que pudessem comprometer o desenvolvimento.

    \item Análise de Completude: Avaliação cujo foco foi assegurar que todos os comportamentos esperados do sistema, bem como suas funcionalidades essenciais, estivessem devidamente representados nos requisitos especificados.
    
\end{itemize}

\vspace{1em}

Durante esse processo, eventuais inconsistências ou lacunas identificadas foram imediatamente corrigidas, evitando assim que erros se propagassem para as etapas posteriores de desenvolvimento. Essa abordagem preventiva foi fundamental para reduzir retrabalho e garantir maior eficiência no ciclo de desenvolvimento do aplicativo.

%
\subsection{Arquitetura}
%

A presente seção tem como objetivo descrever a arquitetura adotada na solução proposta para o desenvolvimento deste aplicativo. De acordo com \cite{pressman2021}, a escolha do padrão arquitetural exerce influência direta sobre todo o ciclo de vida do software, uma vez que define a organização dos artefatos, os fluxos de dados, a modularização e a estrutura geral do sistema. A arquitetura escolhida orienta não apenas como os componentes são desenvolvidos, mas também como eles interagem entre si e evoluem ao longo do tempo.

Dentre os diversos estilos arquiteturais existentes, optou-se pela utilização do estilo Arquitetural Baseado em Componentes, devido às suas características que promovem uma alta coesão, baixo acoplamento, facilidade de manutenção, escalabilidade e forte aderência a projetos desenvolvidos com frameworks baseados em componentes, como é o caso do React Native. Para o modelo arquitetural, foi escolhida a arquitetura em 3 camadas. Essa estrutura ofereceu clareza na organização e no gerenciamento do projeto, além de ter facilitado o desenvolvimento e a manutenção dos componentes do software.
\subsection{Estilo Arquitetural Baseado em Componentes}

A Arquitetura Componentizada consiste na divisão do sistema em unidades menores chamadas de componentes, que são independentes, reutilizáveis e responsáveis por funcionalidades bem definidas dentro da aplicação. Cada componente encapsula sua lógica, sua interface e seu estado, comunicando-se com os demais por meio de interfaces e propriedades bem definidas.

De acordo com \cite{pressman2021}, esse modelo arquitetural favorece a construção de sistemas mais robustos, modulares e de fácil manutenção, uma vez que qualquer alteração realizada em um componente não impacta diretamente os demais, desde que a interface de comunicação seja preservada.

No contexto do aplicativo proposto, os componentes serão organizados da seguinte maneira:

\begin{itemize}

    \item \textbf{Componentes de Interface (UI Components):} São responsáveis pela construção da interface gráfica do aplicativo, incluindo botões, formulários, listas, cards, menus e demais elementos visuais. Estes componentes são desenvolvidos utilizando React Native em conjunto com o framework Expo, além do auxílio do NativeWind para estilização baseada em Tailwind CSS.

    \item \textbf{Componentes de Página (Screens):} Representam cada uma das telas do aplicativo, sendo compostos pela combinação de múltiplos componentes de interface e pela integração com os dados e funcionalidades necessárias. Por exemplo, a tela de cadastro de preferências alimentares ou a tela de exibição das sugestões geradas.

    \item \textbf{Componentes Funcionais (Hooks e Serviços)::} Abrangem funções responsáveis por tratar regras específicas, manipulação de estado, requisições assínronas e conexão com serviços de dados. Incluem também hooks personalizados que encapsulam lógicas como validações de formulários, controle de estado e consumo de APIs.
    
\end{itemize}

\vspace{1em}

Essa organização permitiu que cada parte do sistema fosse desenvolvida, testada e evoluída de maneira independente, promovendo maior agilidade no desenvolvimento, facilidade de manutenção e possibilidade de reutilização de código em diferentes contextos do aplicativo. Além disso, a arquitetura componentizada é altamente compatível com metodologias ágeis, como Scrum e práticas do Extreme Programming (XP), pois favorece entregas incrementais e contínuas.

Dessa forma, a arquitetura adotada neste projeto buscou equilibrar simplicidade estrutural, desempenho e escalabilidade, além de proporcionar uma base sólida para a evolução contínua do sistema. Abaixo, na Figura 3.3 há uma representação esquemática da estrutura de diretórios adotada para o projeto, seguindo o modelo do estilo arquitetural componentizado.

\begin{figure} [H]
    \centering
    \includegraphics[width=0.5\linewidth]{arquitetura-componentizada.png}
    \caption{Estrutura da arquitetura baseada em componentes}
    \caption*{Fonte: Autor, 2025.}
    \label{fig:enter-label}
\end{figure}

Essa estrutura favoreceu o isolamento de responsabilidades, o reaproveitamento de componentes e a escalabilidade do projeto, alinhando-se às boas práticas de desenvolvimento de aplicações móveis com React Native e Expo. Além disso, facilitou o trabalho colaborativo da equipe, uma vez que cada parte do sistema pode ser desenvolvida, testada e mantida de forma independente, sem comprometer a integridade do restante do código.

\subsection{Arquitetura Em 3 Camadas}

A Arquitetura em 3 Camadas fundamenta-se na segmentação da aplicação em três níveis lógicos e independentes: a camada de apresentação, a camada de lógica de negócio e a camada de dados. Essa estrutura promove o desacoplamento das responsabilidades, permitindo que cada componente seja desenvolvido, mantido e escalado de forma autônoma, sem criar dependências rígidas entre a interface e o armazenamento. De acordo com \cite{pressman2016}, esse modelo favorece a manutenibilidade e a flexibilidade do software, pois possibilita que alterações em uma camada específica, como a atualização da interface do usuário, ocorram sem impactar necessariamente às regras de negócio ou a estrutura do banco de dados.

\subsection{Desenvolvimento de Software}

O processo de desenvolvimento da aplicação foi sustentado por um conjunto de tecnologias modernas e amplamente utilizadas na indústria de software, escolhidas por sua eficiência, ampla documentação e alinhamento com os objetivos do projeto. A seguir, são descritas as principais ferramentas e ambientes adotados durante a construção do sistema.

\begin{itemize}

    \item \textbf{React Native:} O React Native é um framework criado pelo Facebook que permite o desenvolvimento de aplicativos móveis nativos utilizando a linguagem JavaScript e o paradigma de componentes do React. Sua principal vantagem está na possibilidade de desenvolver aplicações para Android e iOS a partir de uma única base de código, o que reduz custos e acelera o tempo de entrega. Segundo \cite{kuederle2018}, o React Native oferece uma arquitetura flexível, baseada em componentes reutilizáveis, o que promove maior manutenibilidade e escalabilidade dos projetos.

    \item \textbf{Node.js:} O Node.js é um ambiente de execução para código JavaScript no lado do servidor, baseado no motor V8 do Google Chrome. Destaca-se por sua natureza event-driven e assíncrona, o que o torna altamente eficiente para aplicações que demandam escalabilidade e manipulação intensiva de I/O (entrada e saída). Além disso, trata-se de uma tecnologia de código aberto, amplamente suportada pela comunidade e adotada em larga escala por grandes empresas. Conforme a documentação oficial \cite{nodejs}, o modelo orientado a eventos e não bloqueante do Node.js o torna leve e eficiente, sendo ideal para aplicações de dados intensivos em tempo real que rodam em dispositivos distribuídos.

    \item \textbf{NativeWind:} O NativeWind é uma biblioteca que permite a utilização do utilitário Tailwind CSS dentro de projetos desenvolvidos com React Native. Essa abordagem facilita a criação de interfaces coesas e responsivas por meio da aplicação de classes utilitárias diretamente nos componentes da interface. A principal vantagem do NativeWind é a redução da complexidade na estilização, além da maior legibilidade e reuso do código, o que contribui para a agilidade no desenvolvimento e a consistência visual da aplicação.

    \item \textbf{Visual Studio Code:} O Visual Studio Code (VS Code) é um editor de código-fonte leve, gratuito e multiplataforma, com suporte nativo a diversas linguagens de programação. Sua extensibilidade por meio de plugins torna-o uma ferramenta versátil, adequada para o desenvolvimento com React Native, Node.js e outras tecnologias modernas. De acordo com \cite{spinellis2021}, o VS Code representa uma solução moderna e eficiente para ambientes de desenvolvimento integrados em equipes ágeis.

    \item \textbf{Docker:} O Docker é uma plataforma de conteinerização que permite isolar processos em ambientes independentes. Neste projeto, a ferramenta foi utilizada especificamente para a sustentação e o gerenciamento da instância do banco de dados. Essa abordagem permitiu a configuração de um ambiente de dados padronizado e isolado, dispensando a instalação direta do SGBD na máquina local dos desenvolvedores. O uso de contêineres para serviços específicos otimizou o consumo de recursos e garantiu a consistência do ambiente de desenvolvimento, evitando conflitos de versões e dependências.

    \item \textbf{PostgreSQL:} O PostgreSQL é um sistema gerenciador de banco de dados objeto-relacional (SGBD-OR) de código aberto, reconhecido por sua robustez, estabilidade e conformidade com os padrões ACID (Atomicidade, Consistência, Isolamento e Durabilidade). No projeto, ele atuou na camada de dados, sendo responsável pelo armazenamento seguro e estruturado das informações dos usuários, perfis sensoriais e histórico alimentar.
\end{itemize}

\vspace{1em}

\subsection{Gerência de configuração de software}

A gerência de configuração de software (SCM) é uma disciplina fundamental dentro da engenharia de software, responsável por estabelecer processos, diretrizes e práticas voltadas ao controle sistemático das modificações realizadas no sistema ao longo do seu ciclo de desenvolvimento. Sua aplicação visa assegurar que todas as alterações sejam devidamente rastreadas, documentadas e integradas de forma controlada.

Essa prática torna-se essencial para evitar problemas como a perda de informações sobre o histórico de mudanças, a ocorrência de retrabalho e a geração de inconsistências entre diferentes versões do sistema desenvolvidas simultaneamente pela equipe. Segundo \cite{sommerville2007}, a ausência de um controle eficiente sobre a configuração do software pode comprometer significativamente a integridade do projeto e dificultar sua manutenção e evolução.

\subsection{Repositório}

Para o armazenamento e controle do código-fonte do projeto, foi utilizado um repositório na plataforma GitHub, que centraliza todo o desenvolvimento da aplicação. O repositório foi configurado como privado, garantindo que apenas os membros da equipe de desenvolvimento e o orientador tivessem acesso aos artefatos do projeto. A escolha pelo GitHub se deu, principalmente, devido à familiaridade da equipe com a ferramenta, além de seus recursos robustos de controle de versão, colaboração e rastreamento de alterações, os quais são fundamentais para manter a integridade e a organização do desenvolvimento ao longo de todo o ciclo de vida do software.

\subsection{Política de Branchs}

A estratégia adotada para a organização das branches no repositório foi inspirada no modelo Git Flow, proposto por Vincent Driessen \cite{driessen2010}, amplamente utilizado por equipes de desenvolvimento devido à sua robustez na gestão de versões e controle de mudanças. Contudo, para atender às necessidades específicas deste projeto, foram realizadas algumas adaptações na estrutura tradicional do Git Flow, incluindo a adição das branches release e docs, as quais não fazem parte da proposta original, mas foram fundamentais para o fluxo de trabalho adotado.

A definição e a função de cada branch ficaram estabelecidas da seguinte maneira:

\begin{itemize}

    \item \textbf{Main:} Contém a versão mais estável e consolidada do projeto, pronta para deploy ou entrega. Qualquer alteração nesta branch reflete uma versão oficialmente liberada do software.

    \item \textbf{Develop:} Serve como branch de integração, onde são reunidas todas as novas funcionalidades e correções desenvolvidas antes de serem promovidas à versão estável. É, portanto, uma versão de desenvolvimento que reflete o estado atual da construção do sistema.

    \item \textbf{Feature:} Branch criada a partir da develop, destinada ao desenvolvimento de uma nova funcionalidade, melhoria específica ou implementação de algum item do backlog. Após finalizada e testada, é integrada novamente à develop.

    \item \textbf{Hotfix:} Utilizada exclusivamente para correções emergenciais em produção. Quando um problema crítico é identificado na branch main, essa branch é criada para resolver rapidamente o problema e, após a correção, é integrada tanto na main quanto na develop.

    \item \textbf{Release:} Branch criada a partir da develop quando uma versão do sistema está próxima de ser finalizada. Nela são feitas correções menores, ajustes finais e preparação para o lançamento. Após a finalização, ela é mesclada tanto na main quanto na develop.

    \item \textbf{Docs:} Branch dedicada exclusivamente à documentação do projeto. Permite que a documentação evolua de forma independente do desenvolvimento do código, facilitando atualizações, correções e melhorias contínuas nos materiais de apoio, como o README, Wiki ou documentos técnicos.

\end{itemize}

\vspace{1em}

Essa abordagem híbrida proporcionou uma organização eficiente do fluxo de trabalho, assegurando maior controle sobre os diferentes estágios de desenvolvimento, manutenção da qualidade do código e facilidade na gestão de entregas e documentação. A introdução das branches release e docs contribuirá significativamente para atender às demandas específicas deste projeto acadêmico, tornando o processo mais organizado, rastreável e alinhado às boas práticas de desenvolvimento de software.

\subsection{Política de commits}

Para garantir clareza, rastreabilidade e organização no histórico de desenvolvimento do projeto, foi adotada uma política de padronização nas mensagens de commit. Essa prática visa facilitar a compreensão das alterações realizadas, tanto para os desenvolvedores envolvidos quanto para eventuais colaboradores futuros, além de contribuir para a manutenção, revisão e evolução do código de forma eficiente.

O formato utilizado segue o seguinte padrão: \textbf{<tipo> <descrição breve e objetiva>}. Nessa estrutura, o tipo indica a natureza da alteração realizada no código, enquanto a descrição resume, de forma clara, o que foi modificado. A seguir, são descritos os tipos definidos e suas respectivas finalidades:

\begin{itemize}

    \item \textbf{feat:} Indica a implementação de uma nova funcionalidade no sistema.

    \item \textbf{fix:} Refere-se à correção de bugs ou problemas identificados.

    \item \textbf{docs:} Usado para alterações relacionadas à documentação, como atualizações no README ou em arquivos de suporte.

    \item \textbf{style:} Alterações puramente estéticas ou de formatação, como ajustes de identação, remoção de espaços, alterações de estilo, que não afetam a lógica do código.

    \item \textbf{refactor:} Aplicado em mudanças na estrutura do código que não alteram seu comportamento externo, como melhorias internas, reorganização de funções ou métodos.

    \item \textbf{test:} Relacionado à criação, modificação ou melhoria de testes automatizados.

    \item \textbf{chore:} Utilizado para tarefas de manutenção e configuração do projeto, que não estão diretamente ligadas ao desenvolvimento de funcionalidades ou correção de erros. Exemplos incluem atualizações de dependências, ajustes em scripts ou arquivos de configuração.
    
\end{itemize}

\vspace{1em}

Exemplos de commits seguindo essa convenção:

\begin{itemize}

    \item \textbf{feat:} adicionar tela de cadastro de usuários

    \item \textbf{fix:} corrigir erro no comportamento do botão de login

    \item \textbf{docs:} atualizar README com instruções de execução

    \item \textbf{style:} ajustar espaçamentos e identação nos componentes de tela

    \item \textbf{refactor:} reorganizar funções do serviço de autenticação

    \item \textbf{test:} adicionar testes unitários para o componente de sugestões

    \item \textbf{chore:} atualizar dependências do projeto no package.json
    
\end{itemize}

\vspace{1em}

A adoção desta política contribui significativamente para a manutenção de um histórico limpo, organizado e semântico, alinhando-se às boas práticas de desenvolvimento de software recomendadas por autores como \cite{pressman2021}. Além disso, essa abordagem favorece a utilização futura de ferramentas de integração contínua, geração de changelogs automáticos e auditoria de modificações no código-fonte.

%
\section{Métricas}
%

Esta seção tem como finalidade descrever o processo de coleta e análise de métricas, com ênfase em fornecer subsídios para o acompanhamento administrativo e o gerenciamento do cumprimento do cronograma de entregas planejadas para o projeto.

\subsection{GQM}

Para apoiar esse processo, adotou-se a estratégia GQM (Goal-Question-Metric), proposta por \cite{basili1994}, que estabelece um método sistemático para definir e organizar a medição em projetos de software. O GQM estrutura o monitoramento em três níveis: objetivos, questões e métricas.

A aplicação do método ocorreu por meio das seguintes etapas:

\begin{itemize}

    \item Definição dos objetivos principais, que destacam os focos prioritários da medição dentro do contexto do projeto.

    \item Formulação de perguntas, que detalham cada objetivo e norteiam a investigação sobre o progresso e o desempenho das atividades.

    \item Determinação das métricas, especificando os dados a serem coletados para responder de forma clara às questões formuladas.

    \item Coleta, verificação e análise dos dados, garantindo que os registros sejam válidos, confiáveis e úteis para apoiar a tomada de decisão.
    
\end{itemize}

\vspace{1em}

A utilização dessa abordagem possibilitou uma avaliação estruturada do andamento do desenvolvimento, identificando pontos de conformidade com as metas traçadas, além de sinalizar oportunidades de ajustes e melhorias nos processos.

No âmbito desta pesquisa, o GQM foi especialmente direcionado para o acompanhamento das atividades de desenvolvimento de software, uma etapa central e de alta criticidade no projeto. Assim, foi realizada uma declaração formal dos objetivos de medição, sintetizada na Tabela 3.1, que detalha cada meta definida e as métricas correspondentes, permitindo um monitoramento claro e objetivo de todos os aspectos relevantes para a gestão do progresso do trabalho.


\begin{table}[H]
\centering
% Usamos 'tabularx' com a largura do texto (\textwidth) para garantir que a tabela não ultrapasse a página.
% O tipo de coluna 'X' permite que o texto seja quebrado em múltiplas linhas de forma automática.
% O comando \raggedright evita espaçamento excessivo entre palavras em colunas estreitas.
\begin{tabularx}{\textwidth}{| >{\raggedright\arraybackslash}p{4cm} | >{\raggedright\arraybackslash}X | >{\raggedright\arraybackslash}X |}
\hline
\textbf{Objetivos de Medição} & \textbf{Descrição do Objetivo} & \textbf{Questões e Métricas} \\ \hline

\textbf{Objetivo 1:} Gerenciar e acompanhar o progresso do cumprimento dos requisitos do sistema a partir do backlog
&
\textbf{Analisar} o cumprimento dos requisitos do sistema
\textbf{com o propósito de} entender o progresso do desenvolvimento em relação ao que foi planejado.
\textbf{Em relação ao} backlog do produto.
\textbf{Do ponto de vista do} gerente de projeto e do cliente.
\textbf{No contexto do} desenvolvimento do sistema.
&
\textbf{Questão 1.1:} O backlog atende às necessidades do usuário? \newline
\textbf{Métrica 1.1.1:} Número de tarefas concluídas do backlog \vspace{0.3cm} \newline
\textbf{Questão 1.2:} O backlog foi cumprido? \newline
\textbf{Métrica 1.2.1:} Número de tarefas concluídas do backlog \\ \hline

\textbf{Objetivo 2:} Gerenciar e acompanhar o cronograma do desenvolvimento do sistema
&
\textbf{Analisar} o andamento do cronograma
\textbf{com o propósito de} verificar a aderência aos prazos estabelecidos.
\textbf{Em relação ao} escopo do projeto.
\textbf{Do ponto de vista do} gerente de projetos e dos stakeholders.
\textbf{No contexto do} planejamento e execução do projeto.
&
\textbf{Questão 2.1:} O cronograma está realista para o escopo do projeto? \newline
\textbf{Métrica 2.1.1:} Porcentagem de funcionalidades mínimas implementadas no MVP. \\ \hline

\textbf{Objetivo 3:} Gerenciar e acompanhar o progresso do desenvolvimento do sistema
&
\textbf{Analisar} o processo de desenvolvimento
\textbf{com o propósito de} avaliar a produtividade da equipe e a qualidade do produto.
\textbf{Em relação à} entrega de valor e à saúde do código.
\textbf{Do ponto de vista da} equipe de desenvolvimento e do líder técnico.
\textbf{No contexto das} sprints e do ciclo de vida do software.
&
\textbf{Questão 3.1:} Qual a produtividade média da equipe? \newline
\textbf{Métrica 3.1.1:} Taxa de bugs corrigidos (bugs corrigidos / total de bugs encontrados). \newline
\textbf{Métrica 3.1.2:} Número de tarefas concluídas \vspace{0.3cm} \newline
\textbf{Questão 3.2:} Quantas versões do produto foram desenvolvidas? \newline
\textbf{Métrica 3.2.1:} Total de releases entregues \vspace{0.3cm} \newline
\textbf{Questão 3.3:} Quantas tarefas estão relacionadas a débitos técnicos? \newline
\textbf{Métrica 3.3.1:} Número de débitos técnicos registrados \\ \hline
\end{tabularx}
\caption{Objetivos, questões e métricas definidas pelo método GQM}
\caption*{Fonte: Autor, 2025.}
\label{tab:gqm_corrigido}
\end{table}

%
\section{Testes}
%

A realização de testes durante o processo de desenvolvimento foi um fator essencial para assegurar que o sistema estivesse funcionando conforme os requisitos definidos, de maneira confiável, eficiente e com qualidade. Com base nisso, foram adotadas estratégias de verificação sistemática por meio da aplicação de três tipos principais de testes: testes unitários, testes de integração e testes de sistema.

\subsection{Testes unitários}

Os testes unitários foram utilizados devido à sua praticidade e agilidade na execução. Esse tipo de teste é focado na validação de unidades isoladas do sistema — como funções e componentes — permitindo detectar falhas de forma precoce e com baixo custo. Por suas particularidades, os testes unitários contribuíram para garantir que pequenas porções da lógica de negócio estejam funcionando corretamente e que os requisitos estejam sendo devidamente atendidos.

Para a implementação dos testes unitários, foi utilizada a ferramenta Jest, um framework amplamente consolidado na comunidade JavaScript, conhecido por sua facilidade de configuração, performance otimizada e abrangente suporte à análise de cobertura de código.

Segundo a documentação oficial, o Jest foi inicialmente desenvolvido pelo Facebook com foco em testes de aplicações React, mas atualmente é utilizado em projetos variados de front-end e back-end, tornando-se uma solução robusta e versátil. Além disso, o Jest se destaca por executar testes de forma paralela, o que proporciona ganhos significativos de desempenho em projetos com grande volume de código.

A escolha pelo Jest se justifica, portanto, não apenas pela sua compatibilidade com o ecossistema do React Native, mas também por sua ampla adoção pela comunidade, extensa documentação e facilidade na identificação de problemas. Tais características o tornam uma ferramenta adequada para o contexto deste projeto, que exige confiabilidade, agilidade e suporte técnico durante as fases de desenvolvimento e validação.

\subsection{Testes de integração}

Os testes de integração desempenharam um papel essencial na verificação da interação entre diferentes módulos do sistema, assegurando que as funcionalidades implementadas de forma isolada operem corretamente quando combinadas. Esse tipo de teste visa identificar falhas na comunicação entre componentes, como inconsistências de dados, falhas de autenticação ou problemas na persistência das informações.

No contexto deste projeto, os testes de integração foram empregados principalmente para verificar operações críticas que envolvem múltiplos elementos do sistema, como, por exemplo, o fluxo de cadastro de um usuário, no qual a comunicação entre o front-end, o back-end e o banco de dados é fundamental para o correto funcionamento da funcionalidade.

Para a realização dos testes, foi utilizado o Insomnia, uma ferramenta open-source amplamente utilizada para a realização de requisições HTTP no padrão REST. O Insomnia permitiu a simulação de chamadas a endpoints da API da aplicação, oferecendo uma interface intuitiva para o envio de dados, análise das respostas e verificação dos comportamentos esperados.

A escolha pelo Insomnia se deu por sua facilidade de uso, suporte a diferentes métodos HTTP e recursos como histórico de requisições, organização por coleções e visualização clara das respostas da API. Tais características o tornam especialmente útil durante a fase de testes de integração, permitindo identificar com precisão eventuais falhas de comunicação entre os componentes da aplicação.

\subsection{Testes Funcionais e de Interface (UI)}

Além dos testes unitários e de integração, foram aplicados testes funcionais e de interface com o objetivo de validar o comportamento do sistema do ponto de vista do usuário final. Esses testes asseguram que as funcionalidades descritas nos requisitos estejam sendo executadas corretamente quando acionadas por meio da interface gráfica, e que a interação com os componentes visuais esteja ocorrendo conforme o esperado.

Os testes funcionais verificam se o sistema atende aos seus propósitos e realiza as tarefas previstas — como o envio de formulários, navegação entre telas, e a geração de sugestões alimentares com base nos dados inseridos. Já os testes de interface (também chamados de testes de UI) focam na resposta visual e comportamental dos elementos da aplicação, como botões, campos de entrada, mensagens de erro e feedbacks visuais, garantindo que estejam responsivos e alinhados com as diretrizes de usabilidade.

Para essa finalidade, foi empregada ferramentas como a React Native Testing Library, que é uma extensão da popular Testing Library adaptada para aplicações desenvolvidas em React Native. Essa biblioteca ofereceu um conjunto de utilitários para simular interações do usuário e verificar a presença, comportamento e acessibilidade dos componentes da interface. Segundo \cite{domingues2020}, a adoção de bibliotecas que promovem testes baseados na perspectiva do usuário contribui significativamente para o aumento da confiabilidade da aplicação e melhora da experiência de uso.

A escolha por realizar testes funcionais e de interface também se justifica pela natureza do público-alvo da aplicação — cuidadores de pessoas com Transtorno do Espectro Autista (TEA) —, o que exige interfaces claras, acessíveis e com feedbacks visuais bem definidos. A validação contínua dessas interações é, portanto, indispensável para assegurar que o sistema seja compreensível e eficaz em sua proposta.

%
\section{Tecnologias}
%

\subsection{Gestão do projeto}

\begin{itemize}

    \item \textbf{GitHub:} O GitHub é uma plataforma baseada em computação em nuvem voltada ao desenvolvimento colaborativo de software. Ele oferece uma série de funcionalidades que auxiliam no gerenciamento de projetos, entre as quais se destacam: o controle de versão distribuído, por meio da integração com o sistema Git; a criação e organização de issues; a geração e distribuição de releases (versões estáveis do sistema); e o uso de quadros no estilo Kanban, que permitem o acompanhamento visual do progresso das atividades.
    
\end{itemize}

\vspace{1em}

Conforme apresentado na subseção 3.2.4.0.2 — Política de Branches — o repositório do projeto, hospedado no GitHub, foi estruturado com seis branches distintas. Cada uma delas foi definida para atender a demandas específicas ao longo do ciclo de desenvolvimento, garantindo organização, rastreabilidade e melhor gerenciamento das entregas, essa organização é mostrada na Figura 3.4.

\begin{figure}[H]
    \centering
    \includegraphics[width=0.5\linewidth]{organização-github.png}
    \caption{Organização das branches - GitHub}
    \caption*{Fonte: Autor, 2025.}
    \label{fig:enter-label}
\end{figure}

\subsection{Ferramentas de comunicação}

\begin{itemize}

    \item \textbf{Microsoft Teams:} O Microsoft Teams é uma plataforma de comunicação corporativa que integra funcionalidades de mensagens instantâneas, reuniões virtuais, armazenamento de arquivos e integração com ferramentas do pacote Microsoft 365, como Word, Excel e PowerPoint. No contexto deste projeto, o Microsoft Teams foi escolhido como principal ferramenta de comunicação entre os membros da equipe de desenvolvimento, o Product Owner (PO) e o orientador. Tal escolha se justifica pela disponibilidade gratuita da plataforma aos estudantes da Universidade de Brasília (UnB), além da centralização de mensagens, agendamento de reuniões, compartilhamento de documentos e histórico de interações, o que contribui significativamente para a organização e acompanhamento das etapas do projeto.
    
\end{itemize}

\vspace{1em}

\subsection{Persistência de dados}

A estratégia de persistência de dados foi definida para assegurar a consistência, integridade e escalabilidade do sistema. A modelagem considerou a sensibilidade das informações registradas, como os dados alimentares de pessoas com Transtorno do Espectro Autista (TEA). Essa estruturação prévia permitiu uma implementação coerente dos mecanismos de armazenamento e acesso, fundamentada nas melhores práticas de engenharia de dados \cite{silberschatz2011}.

\subsection{PostgreSQL}

O armazenamento dos dados foi implementado utilizando o PostgreSQL. Este sistema gerenciador de banco de dados relacional (SGBD) foi escolhido por sua robustez, conformidade com o padrão SQL e suporte a transações ACID, características essenciais para garantir a confiabilidade das informações \cite{postgresql2024}.

A escolha de um banco relacional se mostrou adequada para a natureza estruturada dos dados do projeto, como os perfis sensoriais e as sugestões geradas pelo algoritmo. Isso permitiu a definição de restrições de integridade rigorosas e a execução de consultas eficientes. A comunicação com o banco é mediada pela API RESTful em Node.js, que centraliza e valida todas as operações de leitura e escrita, garantindo que o acesso às informações do usuário ocorra de forma segura.

\subsection{Ferramentas de Apoio à Escrita}

Para apoio na revisão textual, estruturação de seções e aprimoramento da clareza da redação, foram utilizadas ferramenta de inteligência artificial, como o ChatGPT e o Gemini. Sua aplicação restringiu-se a sugestões de melhoria linguística e organização de texto, sem interferir no conteúdo técnico ou nos resultados do projeto.


% Nomenclaturas de Siglas e Abreviações para este capítulo (coloque as do Capítulo 1 + as novas deste capítulo aqui, de forma única)
\nomenclature[A]{UNB}{Universidade de Brasília}
\nomenclature[A]{REST}{Representational State Transfer}
\nomenclature[A]{ACID}{Atomicity, Consistency, Isolation, Durability}

% Capítulo de resultados
% ----------------------------------------------------------
\chapter{Proposta de Solução}
% ----------------------------------------------------------

Este capítulo tem como objetivo apresentar os principais aspectos envolvidos no desenvolvimento da solução proposta: um aplicativo voltado à sugestão de trocas alimentares para pessoas com Transtorno do Espectro Autista (TEA). Inicialmente, é feita uma contextualização do problema da seletividade alimentar no TEA e uma descrição do público-alvo da aplicação. Em seguida, são apresentados os elementos que compõem o planejamento do produto, como o \textit{backlog} com as funcionalidades previstas, e as definições de identidade visual — incluindo logotipo, paleta de cores e tipografia.

Além disso, este capítulo traz os protótipos de alta fidelidade desenvolvidos para representar visualmente a interface do aplicativo, buscando garantir uma experiência acessível e agradável para os usuários. Por fim, são descritas a arquitetura do produto, que define a organização técnica da aplicação, e a lógica do algoritmo responsável por sugerir alternativas alimentares adequadas, respeitando preferências sensoriais e necessidades nutricionais. Esses componentes são fundamentais para garantir que a solução seja funcional, intuitiva e alinhada aos objetivos propostos neste trabalho.

\section{Sobre o Aplicativo}

A ideia do aplicativo foi concebida para apoiar principalmente cuidadores de pessoas com Transtorno do Espectro Autista (TEA) em suas rotinas alimentares, entendendo que cada pessoa possui preferências, necessidades e sensibilidades únicas. Através dele, sugerimos trocas de alimentos personalizadas, considerando as peculiaridades de cada indivíduo, para tornar a alimentação mais variada, nutritiva e prazerosa no dia a dia.

Além de beneficiar diretamente os usuários, o aplicativo também oferece suporte aos nutricionistas, permitindo o acompanhamento detalhado de seus pacientes. Dessa forma, contribui para um atendimento mais individualizado e eficiente, fortalecendo o vínculo profissional-paciente e auxiliando na construção de hábitos alimentares mais saudáveis, que impactam positivamente a saúde, a autonomia e a qualidade de vida de cada pessoa atendida.

\subsection{O Público Alvo}

Com o objetivo de entender melhor quem serão os usuários do aplicativo desenvolvido, foram criadas personas e uma antipersona para representar diferentes perfis de público \cite{ferreira2018}. A primeira persona definida é a Dra. Renata Lopes, nutricionista especialista em TEA, que já possui experiência no atendimento de pacientes com seletividade alimentar, mas busca ferramentas que tornem mais prático o processo de organização das orientações e sugestões de trocas alimentares individualizadas.

Já a segunda persona, Maria de Fátima Silva, representa mães cuidadoras de pessoas autistas com alto nível de suporte, que precisam de orientações seguras sobre alternativas alimentares para seus filhos, mas sentem medo de que mudanças possam gerar crises ou recusa alimentar.

A terceira persona, Gabriel Henrique Rocha, retrata pessoas autistas com baixo suporte, que possuem seletividade alimentar, mas têm interesse em diversificar sua dieta por conta própria, desde que encontrem sugestões práticas, alinhadas aos seus gostos e características sensoriais.

Além dessas, foi definida uma antipersona, Carlos Eduardo Almeida, que simboliza indivíduos que não fazem parte do público-alvo do aplicativo, como profissionais de tecnologia que não possuem interesse ou relação com o tema da seletividade alimentar no TEA.

\begin{table}[H]
\centering
\caption{Persona 1 – Nutricionista}
\begin{tabular}{|p{3cm}|p{10cm}|}
\hline
\textbf{Nome:} & Dra. Renata Lopes \\ \hline
\textbf{Idade:} & 34 anos \\ \hline
\textbf{Profissão:} & Nutricionista Clínica (Especialista em TEA) \\ \hline
\textbf{Localização:} & Brasília – DF \\ \hline
\textbf{Perfil:} & Profissional organizada, empática e atualizada com pesquisas sobre seletividade alimentar no TEA. Atende pacientes de diferentes idades em consultório e online. Utiliza aplicativos para agendamento, prontuário e planos alimentares. \\ \hline
\textbf{Objetivos no aplicativo:} &
- Cadastrar recomendações e trocas alimentares para cada paciente \\
& - Visualizar histórico alimentar e respostas de questionários \\
& - Gerenciar orientações de forma rápida e centralizada \\ \hline
\textbf{Dores e necessidades:} &
- Baixa adesão dos pacientes ao plano alimentar devido à seletividade \\
& - Falta de tempo para organizar individualmente as orientações para muitos pacientes \\ \hline
\end{tabular}
\\
    \caption*{Fonte: Autor, 2025.}
\end{table}

\begin{table}[H]
\centering
\caption{Persona 2 – Cuidadora de Autista (Alto Suporte)}
\begin{tabular}{|p{3cm}|p{10cm}|}
\hline
\textbf{Nome:} & Maria de Fátima Silva \\ \hline
\textbf{Idade:} & 58 anos \\ \hline
\textbf{Profissão:} & Cuidadora em tempo integral (mãe) \\ \hline
\textbf{Localização:} & Goiânia – GO \\ \hline
\textbf{Perfil:} & Mãe de João (15 anos, autista nível 3, totalmente dependente para alimentação). Ensino médio completo, utiliza o celular principalmente para WhatsApp, vídeos e receitas. Busca melhorar a alimentação do filho, mas tem receio de mudanças que causem crises. \\ \hline
\textbf{Objetivos no aplicativo:} &
- Consultar trocas alimentares sugeridas pela nutricionista \\
& - Responder questionários sobre as preferências e recusas do filho \\
& - Receber instruções práticas e visuais para implementar no dia a dia \\ \hline
\textbf{Dores e necessidades:} &
- Ansiedade ao introduzir novos alimentos \\
& - Sobrecarga emocional e falta de tempo para planejar refeições variadas \\ \hline
\end{tabular}
\\
    \caption*{Fonte: Autor, 2025.}
\end{table}

\begin{table}[H]
\centering
\caption{Persona 3 – Autista (Baixo Suporte)}
\begin{tabular}{|p{3cm}|p{10cm}|}
\hline
\textbf{Nome:} & Gabriel Henrique Rocha \\ \hline
\textbf{Idade:} & 22 anos \\ \hline
\textbf{Profissão:} & Estudante de Análise de Sistemas \\ \hline
\textbf{Localização:} & São Paulo – SP \\ \hline
\textbf{Perfil:} & Autista nível 1, independente, mora com os pais, mas gerencia sua própria alimentação. Dieta restrita a poucos alimentos (arroz, nuggets, batata frita, refrigerante). Usa intensamente apps de organização, saúde e estudos. Motivado a melhorar a alimentação por conta própria. \\ \hline
\textbf{Objetivos no aplicativo:} &
- Receber sugestões de trocas alimentares práticas e compatíveis com seus gostos \\
& - Registrar alimentos que aceita e aqueles que deseja tentar consumir \\
& - Acompanhar o progresso para manter motivação \\ \hline
\textbf{Dores e necessidades:} &
- Ansiedade ao experimentar novos alimentos sem suporte direto \\
& - Dificuldade de planejar refeições variadas \\ \hline
\end{tabular}
\\
    \caption*{Fonte: Autor, 2025.}
\end{table}

\begin{table}[H]
\centering
\caption{Antipersona – Profissional sem vínculo com alimentação}
\begin{tabular}{|p{3cm}|p{10cm}|}
\hline
\textbf{Nome:} & Carlos Eduardo Almeida \\ \hline
\textbf{Idade:} & 29 anos \\ \hline
\textbf{Profissão:} & Desenvolvedor Backend \\ \hline
\textbf{Localização:} & Curitiba – PR \\ \hline
\textbf{Perfil:} & Profissional de tecnologia que não possui interesse em nutrição ou saúde alimentar voltada ao TEA. Utiliza aplicativos apenas para trabalho, comunicação e lazer. Nunca atuou com pessoas autistas nem busca informações sobre alimentação terapêutica. \\ \hline
\textbf{Motivo para não utilizar o aplicativo:} &
- Não possui relação profissional ou pessoal com o tema \\
& - Não apresenta interesse em alimentação terapêutica \\ \hline
\end{tabular}
\\
    \caption*{Fonte: Autor, 2025.}
\end{table}

\section{\textit{Backlog} do produto}

Conforme introduzido no Capítulo 3, o Backlog do Produto é um artefato central em metodologias ágeis, funcionando como uma lista priorizada de todas as funcionalidades e requisitos do projeto. Para o desenvolvimento da solução proposta neste trabalho, foi criado um backlog específico que serviu como guia para a equipe de desenvolvimento.

Este backlog foi populado com itens na forma de Histórias de Usuário, detalhando as necessidades dos usuários finais. A priorização desses itens foi fundamental para definir o escopo da versão atual do projeto, garantindo que as funcionalidades de maior valor e impacto fossem desenvolvidas primeiro. Portanto, o sistema apresentado neste capítulo é o resultado tangível dos itens que foram selecionados, planejados e executados a partir do topo do nosso backlog.

As Histórias de Usuário foram utilizadas para descrever todas as funcionalidades do sistema sob a perspectiva de quem o utiliza. Adotando o formato padrão de mercado — "Como um [ator], eu quero [realizar uma ação] para que [possa obter um benefício]" —, conseguimos manter o foco na entrega de valor real. Cada história representava um requisito de negócio ou uma necessidade do usuário que o software deveria satisfazer. Em contrapartida, as Tarefas Técnicas foram criadas para registrar trabalhos essenciais que não entregam uma nova funcionalidade necessariamente visível ao usuário final, mas são cruciais para a saúde, qualidade e viabilidade do projeto.

\subsection*{Backlog do Aplicativo (Priorização MoSCoW)}

\begin{itemize}
    \item \textbf{US01 (Must Have, 3 pts)}: Como cuidador, quero criar uma conta para registrar e gerenciar dados dos supervisionados.
    
    \item \textbf{US02 (Must Have, 2 pts)}: Como nutricionista, quero criar uma conta para visualizar os dados compartilhados pelos cuidadores.
    
    \item \textbf{US03 (Must Have, 3 pts)}: Como usuário (cuidador ou nutricionista), quero fazer login para acessar a interface correspondente ao meu tipo de conta.
    
    \item \textbf{US04 (Must Have, 3 pts)}: Como cuidador, quero cadastrar múltiplos supervisionados para gerenciar cada um individualmente.
    
    \item \textbf{US05 (Must Have, 2 pts)}: Como cuidador, quero visualizar os dados de cada supervisionado de forma organizada, para facilitar o acompanhamento.
    
    \item \textbf{US06 (Should Have, 2 pts)}: Como cuidador, quero indicar o nível de suporte (1 a 3) de cada supervisionado para melhor caracterização e análise dos dados.
    
    \item \textbf{US07 (Must Have, 3 pts)}: Como cuidador, quero preencher os questionários de desenvolvimento do perfil alimentar para cada supervisionado.
    
    \item \textbf{US08 (Must Have, 4 pts)}: Eu, como sistema, devo processar os dados dos questionários para identificar o perfil de seletividade alimentar de cada supervisionado.
    
    \item \textbf{US09 (Should Have, 3 pts)}: Como cuidador, quero receber o perfil de seletividade alimentar de cada supervisionado em formato de relatório resumido.
    
    \item \textbf{US10 (Must Have, 5 pts)}: Eu, como sistema, devo gerar um relatório de trocas alimentares baseado no perfil e no grau de seletividade identificados.
    
    \item \textbf{US11 (Must Have, 3 pts)}: Como cuidador, quero visualizar o relatório de trocas alimentares com sugestões organizadas por categoria alimentar.
\end{itemize}

\subsection*{Épicos do Produto}

\begin{itemize}
    \item \textbf{Épico 1: Gestão de Usuários}
    \begin{itemize}
        \item (Must Have) \textbf{US01}: Como cuidador, quero criar uma conta para registrar e gerenciar dados dos supervisionados.
        \item (Must Have) \textbf{US02}: Como nutricionista, quero criar uma conta para visualizar os dados compartilhados pelos cuidadores.
        \item (Must Have) \textbf{US03}: Como usuário (cuidador ou nutricionista), quero fazer login para acessar a interface correspondente ao meu tipo de conta.
        \item (Must Have) \textbf{TK01}: Implementar autenticação com controle de tipo de usuário (guest, cuidador, nutricionista).
    \end{itemize}

    \vspace{0.3cm} % Adiciona um pequeno espaço entre os épicos

    \item \textbf{Épico 2: Cadastro e Gerenciamento de Supervisionados}
    \begin{itemize}
        \item (Must Have) \textbf{US04}: Como cuidador, quero cadastrar múltiplos supervisionados para gerenciar cada um individualmente.
        \item (Must Have) \textbf{US05}: Como cuidador, quero visualizar os dados de cada supervisionado de forma organizada, para facilitar o acompanhamento.
        \item (Should Have) \textbf{US06}: Como cuidador, quero indicar o nível de suporte (1 a 3) de cada supervisionado para melhor caracterização e análise dos dados.
        \item (Must Have) \textbf{TK02}: Criar base de dados para armazenar usuários, supervisionados, dados alimentares e respostas dos questionários.
    \end{itemize}

    \vspace{0.3cm}

    \item \textbf{Épico 3: Avaliação Alimentar e Sensorial}
    \begin{itemize}
        \item (Must Have) \textbf{US07}: Como cuidador, quero preencher os questionários de desenvolvimento do perfil alimentar para cada supervisionado.
        \item (Must Have) \textbf{US08}: Eu, como sistema, devo processar os dados dos questionários para identificar o perfil de seletividade alimentar de cada supervisionado.
        \item (Must Have) \textbf{TK03}: Desenvolver componentes para os questionários com processamento automático dos resultados.
        \item (Must Have) \textbf{TK04}: Implementar integração dos questionários ao banco de dados.
    \end{itemize}

    \vspace{0.3cm}

    \item \textbf{Épico 4: Geração de Perfil e Relatórios Personalizados}
    \begin{itemize}
        \item (Should Have) \textbf{US09}: Como cuidador, quero receber o perfil de seletividade alimentar de cada supervisionado em formato de relatório resumido.
        \item (Must Have) \textbf{US10}: Eu, como sistema, devo gerar um relatório de trocas alimentares baseado no perfil e no grau de seletividade identificados.
        \item (Must Have) \textbf{US11}: Como cuidador, quero visualizar o relatório de trocas alimentares com sugestões organizadas por categoria alimentar.
        \item (Should Have) \textbf{TK05}: Gerar PDF ou tela exportável com o relatório de trocas alimentares personalizado.
    \end{itemize}
\end{itemize}

\section{Identidade Visual}

A identidade visual do nosso projeto foi desenvolvida como um pilar estratégico, com o objetivo de criar uma experiência de usuário intuitiva e acolhedora. As escolhas estéticas, como a paleta de cores e a tipografia, foram intencionalmente selecionadas para refletir os valores do projeto e, posteriormente, validadas com a Product Owner (PO) para garantir seu alinhamento com a visão do produto.

A base da nossa identidade visual está na paleta de cores em tons pastéis e na fonte Baloo 2. A opção por essa estética se fundamenta na busca por conforto visual. Cores suaves e de baixa saturação são amplamente associadas a sensações de calma e tranquilidade, ajudando a criar um ambiente digital menos intimidante e mais convidativo \cite{heller2012}. A fonte Baloo 2, com suas formas arredondadas e amigáveis, complementa essa atmosfera, promovendo uma excelente legibilidade e transmitindo uma sensação de simplicidade e acessibilidade. A combinação desses elementos visa remeter a uma experiência leve e agradável, mesmo que o público-alvo seja geral.

A validação dessas escolhas foi um passo fundamental do processo. Conforme o framework Scrum, o Product Owner é o responsável por maximizar o valor do produto \cite{schwaber2020}. Nesse sentido, a aprovação da nossa PO confirmou que a identidade visual proposta não era apenas uma preferência da equipe, mas uma solução de design funcional que contribui diretamente para as metas de usabilidade e para a proposta de valor que desejamos entregar ao usuário final.

\begin{figure}[H]
    \centering
    \includegraphics[width=0.75\linewidth]{paleta.jpg}
    \caption{Paleta de Cores}
    \caption*{Fonte: Autor, 2025.}
    \label{fig:enter-label}
\end{figure}

\subsection{Logotipo}

A concepção da nossa identidade visual culminou em um logotipo que encapsula, de forma simbólica e integrada, os dois pilares fundamentais do projeto: as trocas alimentares e a conscientização sobre o Transtorno do Espectro Autista (TEA).

A análise da simbologia pode ser decomposta da seguinte forma:

A Dinâmica da Troca: Os elementos principais são representados por duas formas orgânicas que fluem uma em direção à outra. Esse movimento simboliza o ato de dar e receber, a base das trocas alimentares. Ele transmite dinamismo, cuidado e a conexão interpessoal que é o cerne da nossa proposta.

O Símbolo do Autismo: O espaço criado pelo encaixe preciso dessas duas formas gera, em seu contorno, a silhueta da peça de um quebra-cabeça. Este é um símbolo amplamente reconhecido pela conscientização do autismo e representa a complexidade, a diversidade de cada indivíduo no espectro e a busca por encaixes sociais e afetivos que promovam a compreensão e a inclusão.

\begin{figure}[H]
    \centering
    \includegraphics[width=0.5\linewidth]{logo.png}
    \label{fig:enter-label}
    \caption{Logo do Aplicativo}
    \caption*{Fonte: Autor, 2025.}
\end{figure}

\subsection{Protótipo de Alta Fidelidade}

Para materializar o conceito do aplicativo e validar seu design, foi desenvolvida uma etapa crucial no planejamento: a criação de um protótipo de alta fidelidade. A ferramenta escolhida para este trabalho foi o Figma, por sua flexibilidade e capacidade de simular interações complexas.

O principal objetivo deste protótipo foi traduzir os requisitos funcionais e o fluxo de navegação em uma representação visual e interativa. Ele permitiu que a equipe e as partes interessadas pudessem "sentir" como seria o produto final. Com ele, validamos o design da interface (UI) e a experiência do usuário (UX) antes mesmo de escrever a primeira linha de código.

O protótipo abrange todas as principais jornadas do usuário dentro do aplicativo. Foram desenhadas as telas de cadastro e login, a área de gerenciamento dos supervisionados, os questionários de avaliação e, principalmente, a tela de exibição dos relatórios com as sugestões de trocas alimentares.

As imagens de alguns exemplos das telas que compõem este protótipo de alta fidelidade estão disponíveis para consulta no Apêndice A deste documento.

\section{Organização dos Dados}

Para viabilizar as funcionalidades e suportar as regras de negócio do aplicativo de sugestões de trocas alimentares, foi concebido um modelo de dados relacional. A arquitetura do banco de dados foi projetada para ser robusta, escalável e capaz de gerenciar as complexas interações entre diferentes tipos de usuários, avaliações de pacientes e a lógica de recomendação nutricional. O modelo está estruturado em quatro domínios principais, conforme detalhado a seguir.

\subsection{Estrutura de Usuários, Pacientes e Acompanhamento Profissional}

Neste domínio, é feita uma distinção fundamental entre o usuário do sistema e o paciente que recebe a intervenção.
\begin{itemize}
    \item \textbf{Entidade \texttt{User}:} Esta entidade armazena as credenciais de acesso e os dados de identificação dos usuários que operam o sistema, os quais são categorizados em perfis de \texttt{nutricionista} ou \texttt{cuidador}.
    
    \item \textbf{Entidade \texttt{Patient}:} Representa o sujeito central da intervenção, contendo suas informações demográficas e clínicas. A entidade estabelece o relacionamento de dependência (1:N) com a entidade \texttt{User}, onde um cuidador pode ser responsável por um ou mais pacientes, através de uma chave estrangeira (\texttt{id\_user\_caregiver}).
    
    \item \textbf{Entidade \texttt{Nutri\_Patient}:} Implementada como uma entidade associativa, modela o relacionamento de muitos-para-muitos (N:M) entre nutricionistas e pacientes. Esta estrutura garante que um nutricionista possa acompanhar múltiplos pacientes e que um paciente possa, eventualmente, ser acompanhado por mais de um profissional.
\end{itemize}

\subsection{Armazenamento de Dados de Avaliação (Questionários)}

Para documentar o estado inicial e a evolução do paciente, o modelo permite o registro de avaliações contínuas.
\begin{itemize}
    \item \textbf{Entidade \texttt{Questionnaire}:} Funciona como um registro mestre para cada instância de um questionário aplicado, armazenando metadados como o tipo de instrumento (e.g., Questionário de Frequência Alimentar – QFA, BAMBI) e a data da aplicação.
    
    \item \textbf{Entidade \texttt{Questionnaire\_Response}:} Projetada com uma estrutura flexível de par chave-valor (\texttt{pergunta}, \texttt{resposta}), esta entidade armazena cada resposta individual de um questionário. Tal flexibilidade permite que o sistema acomode diversos instrumentos de avaliação sem a necessidade de alterações na estrutura do banco de dados.
\end{itemize}

\subsection{Base de Conhecimento Alimentar e Perfil Sensorial}

Este domínio estrutura a informação sobre os alimentos, sendo o pilar para o algoritmo de recomendação.
\begin{itemize}
    \item \textbf{Entidade \texttt{Food}:} Constitui o catálogo central de alimentos, contendo suas propriedades intrínsecas, como pertencimento a um grupo alimentar e uma classificação nutricional (e.g., caseiro, processado, frito).
    
    \item \textbf{Entidade \texttt{Food\_Sensory\_Profile}:} Vinculada à entidade \texttt{Food}, armazena os atributos extrínsecos e subjetivos de cada alimento, como textura, cor, formato e temperatura. A separação entre \texttt{Food} e \texttt{Food\_Sensory\_Profile} permite uma modelagem mais rica e é fundamental para a execução do algoritmo de similaridade.
    
    \item \textbf{Entidade \texttt{Safe\_Food}:} Representa o subconjunto de alimentos da base de conhecimento que são validados como seguros para um determinado paciente. Esta entidade funciona como o principal insumo para o motor de sugestões, ao definir os ``Alimentos Ponte'' para o processo de troca.
\end{itemize}

\subsection{Modelo de Sugestão, Opções e Feedback}

Este domínio gerencia o ciclo de vida de uma recomendação, desde sua geração até a resposta do usuário.
\begin{itemize}
    \item \textbf{Entidade \texttt{Exchange\_Suggestion}:} É uma entidade transacional que registra cada execução do algoritmo de recomendação. Ela armazena o contexto da sugestão: o paciente, a data, o grupo alimentar alvo (``grupo-meta'') e o \texttt{Safe\_Food} utilizado como referência.
    
    \item \textbf{Entidade \texttt{Exchange\_Option}:} Filha de \texttt{Exchange\_Suggestion}, esta entidade armazena cada uma das opções de troca geradas (a lista ordenada), vinculando um alimento sugerido a uma refeição específica (café, almoço, etc.) e registrando as pontuações calculadas (sensorial, nutricional e final).
    
    \item \textbf{Entidade \texttt{Feedback}:} Esta entidade fecha o ciclo de aprendizado do sistema. Ela armazena a interação do usuário final (cuidador ou paciente independente) com uma \texttt{Exchange\_Option} específica, registrando o status de \texttt{ACEITA} ou \texttt{REJEITADA}.
\end{itemize}

Em suma, esta modelagem de dados relacional provê a estrutura necessária para que a aplicação gerencie com segurança e consistência os dados de seus usuários, execute algoritmos de recomendação inteligentes baseados em perfis complexos e permita o acompanhamento contínuo da evolução dos pacientes, cumprindo assim os objetivos terapêuticos e funcionais do projeto.

\begin{figure}[H]
    \centering
    \includegraphics[width=1\linewidth]{modelorelacional.png}
    \caption{Modelo Relacional do Banco de Dados}
    \caption*{Fonte: Autor, 2025.}     
    \label{fig:enter-label}
\end{figure}

\section{Questionários e Métricas}

Para a condução deste estudo, será empregada uma metodologia de avaliação multifacetada, utilizando um conjunto de três instrumentos distintos e complementares para caracterizar o perfil inicial dos participantes e fundamentar a intervenção. A criação deste perfil inicial robusto será fundamentada em duas dimensões complementares: a comportamental e a dietética.

A dimensão comportamental, que foca na natureza e na severidade dos desafios alimentares, será avaliada através da aplicação do STEP-CHILD \cite{seiverling2011} e do BAMBI \cite{lukens2008}. O STEP-CHILD foi desenvolvido e validado nos EUA por Seiverling, Hendy, \& Williams em 2011 \cite{seiverling2011}, e serve como uma ferramenta de rastreamento para identificar problemas alimentares mais amplos. Já o BAMBI foi desenvolvido e validado nos EUA por Lukens \& Linscheid em 2008 \cite{lukens2008}, oferecendo uma avaliação focada nos comportamentos durante as refeições, especificamente em indivíduos com autismo. Ambos os instrumentos foram rigorosamente validados por seus respectivos autores, garantindo sua confiabilidade para as nossas avaliações. Juntos, eles fornecerão um registro detalhado e quantificável dos comportamentos problemáticos associados à alimentação, servindo para fins documentais. É importante destacar que a utilização desses instrumentos foi sugerida pela nossa Product Owner, fornecedora dos requisitos do projeto.

Paralelamente, a dimensão dietética será documentada pelo Questionário de Frequência Alimentar (QFA). Sua aplicação nos permitirá registrar quantitativamente o repertório alimentar de partida da criança, incluindo a variedade e a frequência de consumo dos alimentos. Assim como os demais, este questionário cumpre um papel documental crucial, pois estabelece a linha de base dietética a partir da qual qualquer progresso poderá ser mensurado futuramente.

É fundamental ressaltar, contudo, a dupla função do QFA nesta metodologia. Além de seu valor para a documentação do perfil inicial, o QFA desempenha o papel central e operacional para a intervenção tecnológica. Os dados coletados por ele, especificamente a lista de "alimentos seguros", são o insumo direto para o algoritmo do nosso sistema. É a partir desta informação que a regra de negócio do aplicativo será executada para gerar as sugestões de trocas alimentares personalizadas.

Em síntese, a abordagem metodológica utiliza o STEP-CHILD e o BAMBI para documentar a natureza do problema comportamental alimentar e o QFA para documentar suas consequências dietéticas. Subsequentemente, o mesmo QFA transcende sua função de registro para se tornar a base operacional da solução proposta, garantindo que a intervenção parta de uma compreensão holística e bem documentada do perfil de cada participante.


\section{Regra de negócio}

No contexto de desenvolvimento de sistemas, regras de negócio podem ser compreendidas como declarações que determinam restrições, condições ou políticas fundamentais para a execução de processos dentro de uma organização. Elas orientam comportamentos, cálculos, decisões e definem como as atividades devem ser conduzidas para alcançar os objetivos definidos \cite{vonhalle2001}. A aplicação dessas regras no desenvolvimento de software garante que a solução tecnológica esteja alinhada às práticas reais do negócio, assegurando que suas funcionalidades atendam de forma adequada às necessidades e diretrizes do domínio de atuação \cite{vonhalle2001}.

Para desenvolver um sistema de sugestões alimentares que seja realmente eficaz para crianças com seletividade alimentar no Transtorno do Espectro Autista (TEA), nosso projeto se baseia em uma ferramenta validada, o Questionário de Frequência Alimentar (QFA), mas com uma extensão crucial. Partimos do QFA por ele ser excelente para traçar um mapa geral do que a criança consome, nos permitindo identificar rapidamente os grupos alimentares que estão ausentes ou são pouco explorados em sua dieta. Contudo, ao lidar com as particularidades do autismo, essa visão geral, por si só, é insuficiente e pode levar a recomendações ineficazes.

A principal limitação do questionário padrão é que ele ignora o fator mais decisivo para a aceitação de alimentos nesta população: a experiência sensorial. A seletividade no TEA raramente se deve ao alimento em si, mas sim à sua textura, formato, cor, cheiro ou temperatura. Um QFA tradicional registra apenas que a criança aceita "batata", sem diferenciar se é uma batata frita — crocante, salgada e em formato de palito — ou um purê de batata — cremoso, macio e sem forma definida. Para uma criança com hipersensibilidade sensorial, estas não são simples variações, mas sim alimentos completamente distintos, e a aceitação de um não implica na do outro.

O presente trabalho propõe o desenvolvimento de um sistema de software cujo núcleo funcional é governado por uma Regra de Negócio explícita, denominada \textbf{Regra de Negócio para Sugestão de Trocas Alimentares}. Em conformidade com os princípios da Engenharia de Software, que recomendam a separação da lógica de negócio do código de implementação, esta regra foi meticulosamente projetada para operacionalizar a estratégia terapêutica de Encadeamento Alimentar (\textit{Food Chaining}). O objetivo precípuo do sistema é apoiar seus usuários --- sejam pessoas com Transtorno do Espectro Autista (TEA) com baixo nível de necessidade de suporte ou os cuidadores de outros usuários com TEA --- a promoverem a melhoria da qualidade nutricional de sua dieta, assegurando que todos os grupos alimentares essenciais sejam preenchidos por meio de trocas que são, simultaneamente, mais saudáveis e sensorialmente compatíveis.

\subsection{Fundamentos e Modelo de Dados de Suporte}

A operacionalização desta regra de negócio é dependente de um modelo de dados estruturado. Neste modelo, cada \textbf{Alimento Genérico} é classificado em um respectivo \textbf{Grupo Alimentar} (e.g., ``Proteínas'', ``Verduras e Legumes''). A lógica do sistema, entretanto, opera sobre as \textbf{Preparações Culinárias Específicas} (e.g., ``Nugget de frango industrializado'', ``Filé de frango grelhado''). Cada uma dessas preparações possui um conjunto de \textbf{Atributos Sensoriais} associados (textura, formato, cor, sabor, temperatura) e, de forma crucial, uma \textbf{Classificação Nutricional} (e.g., processado, caseiro, assado, frito). Esta modelagem resulta em um perfil detalhado para cada item, o qual serve de base para o processamento do algoritmo de recomendação.

\subsection{O Fluxo Operacional do Algoritmo de Sugestão}

O processo para a geração de uma sugestão de troca alimentar foi implementado por meio de um algoritmo que executa as seguintes etapas sequenciais:

\begin{enumerate}
    \item \textbf{Identificação do Ponto de Partida:} O sistema analisa os dados fornecidos pelo QFA para selecionar um ``Alimento Seguro'' do perfil do usuário, recuperando seu perfil sensorial completo, sua classificação nutricional e seu grupo alimentar.

    \item \textbf{Definição da Meta Nutricional:} O objetivo estratégico do sistema é garantir que a dieta do usuário contemple todos os grupos alimentares essenciais. Mediante análise do QFA, o algoritmo identifica os grupos alimentares com consumo ausente ou deficiente e seleciona um deles como o ``grupo-meta'' para a intervenção.

    \item \textbf{Busca e Pontuação de Candidatos:} O algoritmo executa uma busca restrita às preparações pertencentes ao ``grupo-meta''. Para cada candidato, é calculada uma ``Pontuação de Recomendação'', função composta por dois critérios ponderados: a \textbf{Similaridade Sensorial} com o ``Alimento Seguro'' (para maximizar a aceitabilidade) e a \textbf{Melhora Nutricional} (priorizando trocas que representem um avanço em qualidade).

    \item \textbf{Geração da Lista Ordenada de Opções:} Em vez de apresentar uma única sugestão, o sistema compila uma lista com os 3 a 4 candidatos que obtiveram as maiores Pontuações de Recomendação. Essa lista é apresentada de forma ordenada, da opção mais promissora para a menos.
\end{enumerate}

\subsection{Dinamismo e Adaptação: O Ciclo de Feedback}

O sistema foi projetado para ser dinâmico e evoluir com o uso, através de um ciclo de feedback contínuo:
\begin{description}
    \item[Cenário de Aceitação] Ao registrar que uma sugestão foi aceita, o sistema promove esta preparação ao status de um novo ``Alimento Seguro''. Este processo enriquece o perfil do usuário, criando uma base sensorial mais ampla e diversificada para futuras recomendações, estabelecendo um ciclo virtuoso de expansão do repertório alimentar.
    
    \item[Cenário de Rejeição] Caso a primeira sugestão da lista seja rejeitada, o sistema a marca para não ser oferecida novamente em um curto prazo e apresenta automaticamente a próxima opção da lista. Isso mantém o processo de interação fluido, reduz a frustração e permite que o usuário explore alternativas sem interrupção.
\end{description}

\subsection{Agência do Usuário e o Papel do Profissional}

É fundamental ressaltar que o aplicativo é uma ferramenta de suporte direto, projetada para conferir agência e autonomia a seus usuários principais: pessoas com TEA com independência para gerir sua alimentação e cuidadores. A decisão final sobre qual sugestão da lista tentar, quando e como, pertence inteiramente a eles. O papel de profissionais como nutricionistas e terapeutas ocupacionais é o de observação e acompanhamento do progresso documentado pelo sistema, utilizando os dados para enriquecer suas próprias estratégias terapêuticas, mas sem a necessidade de intervenção direta na operação do aplicativo.

Em suma, a regra de negócio do sistema traduz uma complexa intervenção terapêutica em um algoritmo adaptativo e centrado no usuário, que não apenas sugere trocas alimentares, mas o faz de maneira estratégica, nutricionalmente direcionada e que fomenta a autonomia no processo de construção de uma dieta mais completa e saudável.

\begin{figure}[H]
\centering
    \includegraphics[width=1.1\linewidth]{fluxograma-regra.png}
    \caption{Fluxograma da Regra de Negócio}
    \caption*{Fonte: Autor, 2025.} 
    \label{fig:enter-label}
\end{figure}

% Novas nomenclaturas para este capítulo
\nomenclature[A]{UI}{User Interface}
\nomenclature[A]{UX}{User Experience}
\nomenclature[A]{N}{Número} % Para "1:N" e "N:M"
\nomenclature[A]{M}{Número} % Para "N:M"
\nomenclature[A]{QFA}{Questionário de Frequência Alimentar}
\nomenclature[A]{STEP-CHILD}{Screening Tool for Feeding Problems - Questionário de Triagem para Problemas Alimentares na Infância}
\nomenclature[A]{DSM-5}{Diagnostic and Statistical Manual of Mental Disorders, Fifth Edition} % Adicionando a sigla do DSM-5 aqui.

% Capítulo de conclusões
% ----------------------------------------------------------
\chapter{Conclusões}
% ----------------------------------------------------------

\section{Introdução}

\section{Fragilidades e Propostas de evolução}


% ------------------------------------------------------------------------
% ELEMENTOS PÓS-TEXTUAIS
% ------------------------------------------------------------------------
\postextual
% ------------------------------------------------------------------------

% ---
% Referências bibliográficas
% ---
% Arquivo com as referências bibliográficas
\bibhang{2.2em} % Recuo da margem esquerda da lista de referências
\bibliography{unbtex-example/referencias} % O estilo de citação é selecionado automaticamente
% ---



% ---
% Apêndices
% ---
\begin{apendicesenv}

% Imprime uma página indicando o início dos apêndices
\partapendices

\begin{figure}
    \centering
    \includegraphics[width=0.3\linewidth]{unbtex-example/figuras/telas/tela_cadastro_assistido.jpg}
    \caption{Tela - Pré-Formulário}
    \label{fig:enter-label}
\end{figure}



% ----------------------------------------------------------
\chapter{Protótipo de Média Fidelidade}
\label{apendice_media_fidelidade}
% ----------------------------------------------------------

% Página em modo paisagem para o protótipo
\begin{landscape}
    \centering 
    \begin{figure}
        \centering
        % Ajuste o nome do arquivo abaixo
        \includegraphics[width=0.9\linewidth]{prototipo.png} 
        \caption{Protótipo de Média Fidelidade}
        \label{fig:prototipo_media}
    \end{figure}
\end{landscape}

\include{unbtex-example/apendice-c}

\end{apendicesenv}
% ---



\end{document}